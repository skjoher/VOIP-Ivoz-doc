% Generated by Sphinx.
\def\sphinxdocclass{report}
\newif\ifsphinxKeepOldNames \sphinxKeepOldNamestrue
\documentclass[letterpaper,10pt,english]{sphinxmanual}
\usepackage{iftex}

\ifPDFTeX
  \usepackage[utf8]{inputenc}
\fi
\ifdefined\DeclareUnicodeCharacter
  \DeclareUnicodeCharacter{00A0}{\nobreakspace}
\fi
\usepackage{cmap}
\usepackage[T1]{fontenc}
\usepackage{amsmath,amssymb,amstext}
\usepackage{babel}
\usepackage{times}
\usepackage[Sonny]{fncychap}
\usepackage{longtable}
\usepackage{sphinx}
\usepackage{multirow}
\usepackage{eqparbox}

\addto\captionsenglish{\renewcommand{\contentsname}{Basic Concepts}}

\addto\captionsenglish{\renewcommand{\figurename}{Fig.\@ }}
\addto\captionsenglish{\renewcommand{\tablename}{Table }}
\SetupFloatingEnvironment{literal-block}{name=Listing }

\addto\extrasenglish{\def\pageautorefname{page}}

\setcounter{tocdepth}{1}


\title{IvozProvider Documentation}
\date{Aug 30, 2019}
\release{Oasis}
\author{Irontec}
\newcommand{\sphinxlogo}{}
\renewcommand{\releasename}{Release}
\makeindex

\makeatletter
\def\PYG@reset{\let\PYG@it=\relax \let\PYG@bf=\relax%
    \let\PYG@ul=\relax \let\PYG@tc=\relax%
    \let\PYG@bc=\relax \let\PYG@ff=\relax}
\def\PYG@tok#1{\csname PYG@tok@#1\endcsname}
\def\PYG@toks#1+{\ifx\relax#1\empty\else%
    \PYG@tok{#1}\expandafter\PYG@toks\fi}
\def\PYG@do#1{\PYG@bc{\PYG@tc{\PYG@ul{%
    \PYG@it{\PYG@bf{\PYG@ff{#1}}}}}}}
\def\PYG#1#2{\PYG@reset\PYG@toks#1+\relax+\PYG@do{#2}}

\expandafter\def\csname PYG@tok@gd\endcsname{\def\PYG@tc##1{\textcolor[rgb]{0.63,0.00,0.00}{##1}}}
\expandafter\def\csname PYG@tok@gu\endcsname{\let\PYG@bf=\textbf\def\PYG@tc##1{\textcolor[rgb]{0.50,0.00,0.50}{##1}}}
\expandafter\def\csname PYG@tok@gt\endcsname{\def\PYG@tc##1{\textcolor[rgb]{0.00,0.27,0.87}{##1}}}
\expandafter\def\csname PYG@tok@gs\endcsname{\let\PYG@bf=\textbf}
\expandafter\def\csname PYG@tok@gr\endcsname{\def\PYG@tc##1{\textcolor[rgb]{1.00,0.00,0.00}{##1}}}
\expandafter\def\csname PYG@tok@cm\endcsname{\let\PYG@it=\textit\def\PYG@tc##1{\textcolor[rgb]{0.25,0.50,0.56}{##1}}}
\expandafter\def\csname PYG@tok@vg\endcsname{\def\PYG@tc##1{\textcolor[rgb]{0.73,0.38,0.84}{##1}}}
\expandafter\def\csname PYG@tok@vi\endcsname{\def\PYG@tc##1{\textcolor[rgb]{0.73,0.38,0.84}{##1}}}
\expandafter\def\csname PYG@tok@vm\endcsname{\def\PYG@tc##1{\textcolor[rgb]{0.73,0.38,0.84}{##1}}}
\expandafter\def\csname PYG@tok@mh\endcsname{\def\PYG@tc##1{\textcolor[rgb]{0.13,0.50,0.31}{##1}}}
\expandafter\def\csname PYG@tok@cs\endcsname{\def\PYG@tc##1{\textcolor[rgb]{0.25,0.50,0.56}{##1}}\def\PYG@bc##1{\setlength{\fboxsep}{0pt}\colorbox[rgb]{1.00,0.94,0.94}{\strut ##1}}}
\expandafter\def\csname PYG@tok@ge\endcsname{\let\PYG@it=\textit}
\expandafter\def\csname PYG@tok@vc\endcsname{\def\PYG@tc##1{\textcolor[rgb]{0.73,0.38,0.84}{##1}}}
\expandafter\def\csname PYG@tok@il\endcsname{\def\PYG@tc##1{\textcolor[rgb]{0.13,0.50,0.31}{##1}}}
\expandafter\def\csname PYG@tok@go\endcsname{\def\PYG@tc##1{\textcolor[rgb]{0.20,0.20,0.20}{##1}}}
\expandafter\def\csname PYG@tok@cp\endcsname{\def\PYG@tc##1{\textcolor[rgb]{0.00,0.44,0.13}{##1}}}
\expandafter\def\csname PYG@tok@gi\endcsname{\def\PYG@tc##1{\textcolor[rgb]{0.00,0.63,0.00}{##1}}}
\expandafter\def\csname PYG@tok@gh\endcsname{\let\PYG@bf=\textbf\def\PYG@tc##1{\textcolor[rgb]{0.00,0.00,0.50}{##1}}}
\expandafter\def\csname PYG@tok@ni\endcsname{\let\PYG@bf=\textbf\def\PYG@tc##1{\textcolor[rgb]{0.84,0.33,0.22}{##1}}}
\expandafter\def\csname PYG@tok@nl\endcsname{\let\PYG@bf=\textbf\def\PYG@tc##1{\textcolor[rgb]{0.00,0.13,0.44}{##1}}}
\expandafter\def\csname PYG@tok@nn\endcsname{\let\PYG@bf=\textbf\def\PYG@tc##1{\textcolor[rgb]{0.05,0.52,0.71}{##1}}}
\expandafter\def\csname PYG@tok@no\endcsname{\def\PYG@tc##1{\textcolor[rgb]{0.38,0.68,0.84}{##1}}}
\expandafter\def\csname PYG@tok@na\endcsname{\def\PYG@tc##1{\textcolor[rgb]{0.25,0.44,0.63}{##1}}}
\expandafter\def\csname PYG@tok@nb\endcsname{\def\PYG@tc##1{\textcolor[rgb]{0.00,0.44,0.13}{##1}}}
\expandafter\def\csname PYG@tok@nc\endcsname{\let\PYG@bf=\textbf\def\PYG@tc##1{\textcolor[rgb]{0.05,0.52,0.71}{##1}}}
\expandafter\def\csname PYG@tok@nd\endcsname{\let\PYG@bf=\textbf\def\PYG@tc##1{\textcolor[rgb]{0.33,0.33,0.33}{##1}}}
\expandafter\def\csname PYG@tok@ne\endcsname{\def\PYG@tc##1{\textcolor[rgb]{0.00,0.44,0.13}{##1}}}
\expandafter\def\csname PYG@tok@nf\endcsname{\def\PYG@tc##1{\textcolor[rgb]{0.02,0.16,0.49}{##1}}}
\expandafter\def\csname PYG@tok@si\endcsname{\let\PYG@it=\textit\def\PYG@tc##1{\textcolor[rgb]{0.44,0.63,0.82}{##1}}}
\expandafter\def\csname PYG@tok@s2\endcsname{\def\PYG@tc##1{\textcolor[rgb]{0.25,0.44,0.63}{##1}}}
\expandafter\def\csname PYG@tok@nt\endcsname{\let\PYG@bf=\textbf\def\PYG@tc##1{\textcolor[rgb]{0.02,0.16,0.45}{##1}}}
\expandafter\def\csname PYG@tok@nv\endcsname{\def\PYG@tc##1{\textcolor[rgb]{0.73,0.38,0.84}{##1}}}
\expandafter\def\csname PYG@tok@s1\endcsname{\def\PYG@tc##1{\textcolor[rgb]{0.25,0.44,0.63}{##1}}}
\expandafter\def\csname PYG@tok@dl\endcsname{\def\PYG@tc##1{\textcolor[rgb]{0.25,0.44,0.63}{##1}}}
\expandafter\def\csname PYG@tok@ch\endcsname{\let\PYG@it=\textit\def\PYG@tc##1{\textcolor[rgb]{0.25,0.50,0.56}{##1}}}
\expandafter\def\csname PYG@tok@m\endcsname{\def\PYG@tc##1{\textcolor[rgb]{0.13,0.50,0.31}{##1}}}
\expandafter\def\csname PYG@tok@gp\endcsname{\let\PYG@bf=\textbf\def\PYG@tc##1{\textcolor[rgb]{0.78,0.36,0.04}{##1}}}
\expandafter\def\csname PYG@tok@sh\endcsname{\def\PYG@tc##1{\textcolor[rgb]{0.25,0.44,0.63}{##1}}}
\expandafter\def\csname PYG@tok@ow\endcsname{\let\PYG@bf=\textbf\def\PYG@tc##1{\textcolor[rgb]{0.00,0.44,0.13}{##1}}}
\expandafter\def\csname PYG@tok@sx\endcsname{\def\PYG@tc##1{\textcolor[rgb]{0.78,0.36,0.04}{##1}}}
\expandafter\def\csname PYG@tok@bp\endcsname{\def\PYG@tc##1{\textcolor[rgb]{0.00,0.44,0.13}{##1}}}
\expandafter\def\csname PYG@tok@c1\endcsname{\let\PYG@it=\textit\def\PYG@tc##1{\textcolor[rgb]{0.25,0.50,0.56}{##1}}}
\expandafter\def\csname PYG@tok@fm\endcsname{\def\PYG@tc##1{\textcolor[rgb]{0.02,0.16,0.49}{##1}}}
\expandafter\def\csname PYG@tok@o\endcsname{\def\PYG@tc##1{\textcolor[rgb]{0.40,0.40,0.40}{##1}}}
\expandafter\def\csname PYG@tok@kc\endcsname{\let\PYG@bf=\textbf\def\PYG@tc##1{\textcolor[rgb]{0.00,0.44,0.13}{##1}}}
\expandafter\def\csname PYG@tok@c\endcsname{\let\PYG@it=\textit\def\PYG@tc##1{\textcolor[rgb]{0.25,0.50,0.56}{##1}}}
\expandafter\def\csname PYG@tok@mf\endcsname{\def\PYG@tc##1{\textcolor[rgb]{0.13,0.50,0.31}{##1}}}
\expandafter\def\csname PYG@tok@err\endcsname{\def\PYG@bc##1{\setlength{\fboxsep}{0pt}\fcolorbox[rgb]{1.00,0.00,0.00}{1,1,1}{\strut ##1}}}
\expandafter\def\csname PYG@tok@mb\endcsname{\def\PYG@tc##1{\textcolor[rgb]{0.13,0.50,0.31}{##1}}}
\expandafter\def\csname PYG@tok@ss\endcsname{\def\PYG@tc##1{\textcolor[rgb]{0.32,0.47,0.09}{##1}}}
\expandafter\def\csname PYG@tok@sr\endcsname{\def\PYG@tc##1{\textcolor[rgb]{0.14,0.33,0.53}{##1}}}
\expandafter\def\csname PYG@tok@mo\endcsname{\def\PYG@tc##1{\textcolor[rgb]{0.13,0.50,0.31}{##1}}}
\expandafter\def\csname PYG@tok@kd\endcsname{\let\PYG@bf=\textbf\def\PYG@tc##1{\textcolor[rgb]{0.00,0.44,0.13}{##1}}}
\expandafter\def\csname PYG@tok@mi\endcsname{\def\PYG@tc##1{\textcolor[rgb]{0.13,0.50,0.31}{##1}}}
\expandafter\def\csname PYG@tok@kn\endcsname{\let\PYG@bf=\textbf\def\PYG@tc##1{\textcolor[rgb]{0.00,0.44,0.13}{##1}}}
\expandafter\def\csname PYG@tok@cpf\endcsname{\let\PYG@it=\textit\def\PYG@tc##1{\textcolor[rgb]{0.25,0.50,0.56}{##1}}}
\expandafter\def\csname PYG@tok@kr\endcsname{\let\PYG@bf=\textbf\def\PYG@tc##1{\textcolor[rgb]{0.00,0.44,0.13}{##1}}}
\expandafter\def\csname PYG@tok@s\endcsname{\def\PYG@tc##1{\textcolor[rgb]{0.25,0.44,0.63}{##1}}}
\expandafter\def\csname PYG@tok@kp\endcsname{\def\PYG@tc##1{\textcolor[rgb]{0.00,0.44,0.13}{##1}}}
\expandafter\def\csname PYG@tok@w\endcsname{\def\PYG@tc##1{\textcolor[rgb]{0.73,0.73,0.73}{##1}}}
\expandafter\def\csname PYG@tok@kt\endcsname{\def\PYG@tc##1{\textcolor[rgb]{0.56,0.13,0.00}{##1}}}
\expandafter\def\csname PYG@tok@sc\endcsname{\def\PYG@tc##1{\textcolor[rgb]{0.25,0.44,0.63}{##1}}}
\expandafter\def\csname PYG@tok@sb\endcsname{\def\PYG@tc##1{\textcolor[rgb]{0.25,0.44,0.63}{##1}}}
\expandafter\def\csname PYG@tok@sa\endcsname{\def\PYG@tc##1{\textcolor[rgb]{0.25,0.44,0.63}{##1}}}
\expandafter\def\csname PYG@tok@k\endcsname{\let\PYG@bf=\textbf\def\PYG@tc##1{\textcolor[rgb]{0.00,0.44,0.13}{##1}}}
\expandafter\def\csname PYG@tok@se\endcsname{\let\PYG@bf=\textbf\def\PYG@tc##1{\textcolor[rgb]{0.25,0.44,0.63}{##1}}}
\expandafter\def\csname PYG@tok@sd\endcsname{\let\PYG@it=\textit\def\PYG@tc##1{\textcolor[rgb]{0.25,0.44,0.63}{##1}}}

\def\PYGZbs{\char`\\}
\def\PYGZus{\char`\_}
\def\PYGZob{\char`\{}
\def\PYGZcb{\char`\}}
\def\PYGZca{\char`\^}
\def\PYGZam{\char`\&}
\def\PYGZlt{\char`\<}
\def\PYGZgt{\char`\>}
\def\PYGZsh{\char`\#}
\def\PYGZpc{\char`\%}
\def\PYGZdl{\char`\$}
\def\PYGZhy{\char`\-}
\def\PYGZsq{\char`\'}
\def\PYGZdq{\char`\"}
\def\PYGZti{\char`\~}
% for compatibility with earlier versions
\def\PYGZat{@}
\def\PYGZlb{[}
\def\PYGZrb{]}
\makeatother

\renewcommand\PYGZsq{\textquotesingle}

\begin{document}

\maketitle
\tableofcontents
\phantomsection\label{index::doc}



\chapter{Introduction to IvozProvider}
\label{intro/index::doc}\label{intro/index:introduction-to-ivozprovider}\label{intro/index:ivozprovider-official-documentation}
The following sections will serve as general introduction to IvozProvider:


\section{About this document}
\label{intro/about:about-this-document}\label{intro/about::doc}
This document describes the process of installation and usage of
IvozProvider, the multi-tenant telephony platform for providers developed
by \href{http://irontec.com}{Irontec}.

This should be the starting point for anyone interested in this solution,
both  from the technical point of view and the user one and it's divided
in four blocks:
\begin{itemize}
\item {} 
The first block is about {\hyperref[index:concepts]{\sphinxcrossref{\DUrole{std,std-ref}{Basic Concepts}}}} where each element of the
product and its main function is decscribed.

\item {} 
The second block describes the {\hyperref[index:installation]{\sphinxcrossref{\DUrole{std,std-ref}{Minimum Configuration}}}} process with a fresh
installed platform, leaving the deep configuration details for the next
block.

\item {} 
The third block goes deeper into the most {\hyperref[index:advanced]{\sphinxcrossref{\DUrole{std,std-ref}{Advanced Configuration}}}} features like
\textbf{trarification, billing, PBX advanced options and every call details}
that were omitted in the previous block.

\item {} 
The fourth and last block describes the {\hyperref[index:security]{\sphinxcrossref{\DUrole{std,std-ref}{Security and Maintenance}}}} measures that
implements the solution.

\end{itemize}


\section{Getting help}
\label{intro/getting_help:getting-help}\label{intro/getting_help::doc}\label{intro/getting_help:id1}
IvozProvider is an alive and highly developed project. There are
multiple channels to get information or report bugs:
\begin{itemize}
\item {} 
GitHub: \url{https://github.com/irontec/ivozprovider}

\item {} 
Users mailing list: \href{mailto:users@lists-ivozprovider.irontec.com}{users@lists-ivozprovider.irontec.com}

\item {} 
Developers mailing list: \href{mailto:dev@lists-ivozprovider.irontec.com}{dev@lists-ivozprovider.irontec.com}

\item {} 
email: \href{mailto:vozip@irontec.com}{vozip@irontec.com}

\item {} 
Twitter: \href{https://twitter.com/irontec}{@irontec}

\item {} 
IRC Channel \href{https://webchat.freenode.net/?channels=ivozprovider}{\#ivozprovider} at irc.freenode.net

\end{itemize}

Don't hesitate to contact us for any kind of feedback :)


\section{What is IvozProvider?}
\label{intro/what_is_ivozprovider::doc}\label{intro/what_is_ivozprovider:what-is-ivozprovider}
IvozProvider is a {\hyperref[intro/what_is_ivozprovider:operator\string-oriented]{\sphinxcrossref{\DUrole{std,std-ref}{provider oriented}}}}
{\hyperref[intro/what_is_ivozprovider:multilevel]{\sphinxcrossref{\DUrole{std,std-ref}{multilevel}}}} {\hyperref[intro/what_is_ivozprovider:voip]{\sphinxcrossref{\DUrole{std,std-ref}{IP telephony}}}} solution
{\hyperref[intro/what_is_ivozprovider:exposed]{\sphinxcrossref{\DUrole{std,std-ref}{exposed to the public network}}}}.


\subsection{IP Telephony}
\label{intro/what_is_ivozprovider:ip-telephony}\label{intro/what_is_ivozprovider:voip}
IvozProvider supports telephony systems that use \emph{Session Initialitation
Protocol}, \textbf{SIP}, described in \href{https://tools.ietf.org/html/rfc3261}{RFC 3261} and any \href{https://www.packetizer.com/ipmc/sip/standards.html}{related RFCs} independent of
manufacturers.

This allows total freedom to choose \emph{softphones}, \emph{hardphones} and the
rest of elements that interact with IvozProvider, without any kind of
binding with a manufacturer.

Right now, IvozProvider supports the following \textbf{transport protocols}
for SIP:
\begin{itemize}
\item {} 
UDP

\item {} 
TCP

\item {} 
TLS

\item {} 
Websockets

\end{itemize}

This last transport protocol described in \href{https://tools.ietf.org/html/rfc7118}{RFC 7118} supports web intregrated
softphones, using the \href{https://webrtc.org/}{WebRTC} standard allowing
browsers to establish real-time \emph{peer-to-peer} connections.

The \textbf{supported audio codec} list is:
\begin{itemize}
\item {} 
PCMA (\emph{alaw})

\item {} 
PCMU (\emph{ulaw})

\item {} 
GSM

\item {} 
SpeeX

\item {} 
G.722

\item {} 
G.726

\item {} 
G.729

\item {} 
iLBC

\item {} 
\href{http://opus-codec.org/}{OPUS}

\end{itemize}


\subsection{Multilevel}
\label{intro/what_is_ivozprovider:multilevel}
The web portal design of IvozProvider allows \textbf{multiple actors within the
same infrastructure}:

\noindent\sphinxincludegraphics{{operator_levels}.png}

In {\hyperref[operation_roles/index:operation\string-roles]{\sphinxcrossref{\DUrole{std,std-ref}{Platform roles}}}} section, the different roles are deeply
described, but to sum up:
\begin{itemize}
\item {} 
\textbf{God Admin}: The administrator and maintainer of the solution. Provides
access to multiple brand operators.

\item {} 
\textbf{Brand Operator}: Responsible of giving access, tarificate and bill to
multiple company operators.

\item {} 
\textbf{Company Operator}: Responsible of its own PBX configuration and to
manage the final platform users.

\item {} 
\textbf{Users}: The last link of the chain, has SIP credentials and can access
its own portal for custom configurations.

\end{itemize}

\textbf{Each one} of this roles \textbf{has its own portal} that allows them to
fullfill their tasks. Each portal can be customized in the following
ways:
\begin{itemize}
\item {} 
Themes and \emph{skins} for corporative colours.

\item {} 
Company Logos.

\item {} 
Customized URLs with the Brand or Company domain.

\end{itemize}


\subsection{Provider oriented}
\label{intro/what_is_ivozprovider:operator-oriented}\label{intro/what_is_ivozprovider:provider-oriented}
IvozProvider is a telephony solution \textbf{designed with horizontal scaling
in mind}, what allows handling a great amount of \textbf{traffic and users}
only by increasing the machines and resources of them.

This are the main ideas that makes this product provider oriented:
\begin{itemize}
\item {} 
Despite the fact that all machine profiles can run in the same host,
whatmakes it easier for the initial testing, each profile of IvozProvider
can be splitted from the rest to make it run in its own machine.

\item {} 
A \textbf{distributed installation} allows to distribute the correct amount of
resources to each task, but also:
\begin{itemize}
\item {} 
Geographic distribution of elements to warranty high availability in
caseof CPD failure.

\item {} 
Setup of key elements near the final users, to minimize the communication
latencies.

\item {} 
Horizontal scaling of key profiles to handle hundred of thousands
concurrent calls.

\end{itemize}

\end{itemize}

The resource consuming elements that limit the service of VoIP solutions
use to be:
\begin{itemize}
\item {} 
Already established calls audio management.

\item {} 
Managing configuration for each company administrator (IVRs, conference
rooms, external call filters, etc.)

\item {} 
Databases of configuration and records.

\end{itemize}

IvozProvider was designed always keeping in mind the \textbf{horizontal
scaling} of each of its elements, so it \textbf{can handle hundred of
thousands concurrent calls} and what is more important, \textbf{adapt the
platform resources to the expected service quality}:
\begin{itemize}
\item {} 
\textbf{Media-relay} servers handle audio frames for the already established
calls:
\begin{itemize}
\item {} 
You can use as many media-relays as you need.

\item {} 
You can join media-relay in groups, and force some companies to use a
group if you want.

\item {} 
You can setup media-relays near the final users, to minimize network
latencies in the calls.

\end{itemize}

\item {} 
\textbf{Application servers} are in charge of processing the configurad logics:
\begin{itemize}
\item {} 
They scale horizontally: new Application Serves can be installed and
added to the pool if you feel the need.

\item {} 
Every call is handled by the least busy Appliction Server

\item {} 
By default, there is no static assigment * between Companies and
Application Servers. This way failure of any Application Server is not
critical: the platform will ignore the faulty Application Server while
distributing calls.

\end{itemize}

\end{itemize}


\subsection{Exposed to the public network}
\label{intro/what_is_ivozprovider:exposed-to-the-public-network}\label{intro/what_is_ivozprovider:exposed}
As showed in the installation proces, \textbf{IvozProvider is designed to serve
users directly from Internet}. Although it can be used in local
enviroments, IvozProvider is designed to use public IP addresses for its
services, removing the need of VPN or IPSec tunnels that connect the
infrastructure with the final users

Highlights:
\begin{itemize}
\item {} 
Only the required services will be exposed to Internet.

\item {} 
The unstrusted origins access can be filtered out by integrated firewall

\item {} 
Access from IP addresses or networks can be filtered to avoid any kind of
phishing.

\item {} 
There is also an anti-flood mechanism to avoid short-life Denial of
Service attacks.

\item {} 
Each company concurrent calls can be limited to a fixed amount.

\item {} 
IvozProvider supports connection from terminals behind
\href{https://es.wikipedia.org/wiki/Traducci\%C3\%B3n\_de\_direcciones\_de\_red}{NAT}.

\item {} 
IvozProvider keep track of those NAT windows and keep them alive with
\emph{nat-piercing} mechanisms.

\end{itemize}


\section{What is inside IvozProvider?}
\label{intro/what_is_inside::doc}\label{intro/what_is_inside:what-is-inside-ivozprovider}
IvozProvider uses well-known and stable \href{https://www.gnu.org/philosophy/free-sw.en.html}{Free Software} projects to fullfill
the different required task of the platform.

Nothing better than an image to show all the software that its integrated
into IvozProvider:

\noindent{\hspace*{\fill}\sphinxincludegraphics{{ivozprovider_logos}.png}\hspace*{\fill}}

\begin{notice}{note}{Note:}
We can not stress enough our gratitude to the developers and communities
of this projects.
\end{notice}

The task of each of this software will be deeply detailed in the block
{\hyperref[architecture/index:architecture]{\sphinxcrossref{\DUrole{std,std-ref}{Platform general architecture}}}}.


\section{Who should use IvozProvider?}
\label{intro/use_cases::doc}\label{intro/use_cases:who-should-use-ivozprovider}
IvozProvider is a good option for those insterested in having a telephony
platform that can provide service to \textbf{thousands concurrent calls}.

The greatest strengths of IvozProvide can help to decide if the solution
feeds your needs:
\begin{itemize}
\item {} 
VoIP: SIP

\item {} 
Multilevel, multitenant

\item {} 
Horizontal scaling

\item {} 
PseudoSBC: open to Internet

\item {} 
PBX Features

\end{itemize}

The installation process is so simple, that the best way to test if
IvozProvider fullfills yout needs is to test it!


\chapter{Platform general architecture}
\label{architecture/index:platform-general-architecture}\label{architecture/index::doc}\label{architecture/index:architecture}

\section{General diagram}
\label{architecture/index:general-diagram}
Following diagram shows the global architecture of IvozProvider solution,
with all its components:

\noindent\sphinxincludegraphics{{flows}.png}

This is a more conceptual diagram:

\noindent\sphinxincludegraphics{{conceptual}.png}


\section{SIP signalling flow}
\label{architecture/index:sip-signalling-flow}\label{architecture/index:signallingflow}
The first diagram shows the SIP signalling traffic involved in the
establishment, modification and termination of sessions following the SIP
\href{https://tools.ietf.org/html/rfc3261}{RFC 3261} and any \href{https://www.packetizer.com/ipmc/sip/standards.html}{related RFCs}.

These are the \textbf{external SIP entities} involved:
\begin{itemize}
\item {} 
UACs: users hardphones, softphones, SIP-capable gadget.

\item {} 
SIP carriers: carriers used to interconnect IvozProvider with external SIP
networks (and, probably, with PSTN).

\end{itemize}

All the SIP traffic (in any of the supported transports: TCP, UDP, TLS, WSS)
they send/receive is to/from this two \textbf{internal SIP entities} of IvozProvider:
\begin{itemize}
\item {} 
Users SIP Proxy (running \href{https://www.kamailio.org}{Kamailio}).

\item {} 
Trunks SIP Proxy (running \href{https://www.kamailio.org}{Kamailio}).

\end{itemize}

In fact, users UACs only talks to \emph{Users SIP Proxy} and `SIP carriers' only talks
to \emph{Trunks SIP Proxy}.

Inside IvozProvider these two proxies talk to \emph{Application Servers} running
\href{http://www.asterisk.org/}{Asterisk}, but \textbf{no external element is allowed
to talk to Application Servers directly}.


\section{RTP audio flow}
\label{architecture/index:rtp-audio-flow}\label{architecture/index:audioflow}
Sessions initiated by SIP signalling protocol imply media streams shared by
involved entities.

This media streams use \href{https://tools.ietf.org/html/rfc3550}{RTP} to send and
receive the media itself, usually using UDP as a transport protocol.

\textbf{External entities} involved in RTP sessions can be divided in:
\begin{itemize}
\item {} 
Users.

\item {} 
Carriers.

\end{itemize}

Both entities exchanges RTP with the same IvozProvider entity: \emph{media-relays}.

IvozProvider implements \emph{media-relays} using both \href{https://github.com/sipwise/rtpengine}{RTPengine} and
\href{http://www.rtpproxy.org/}{RTPproxy}.

Similar to SIP, these \emph{media-relays} exchanges RTP when is needed with
\emph{Application Servers}, but \textbf{external entities never talk directly to them}.


\section{HTTPS traffic}
\label{architecture/index:https-traffic}
HTTPS is the third traffic type exchanged between IvozProvider and \emph{external
world}.

HTTPS traffic is used for:
\begin{itemize}
\item {} 
\textbf{Terminal provisioning}: several hardphones ask for their configuration when
they wake up and this configuration files can be served through HTTPS.

\item {} 
\textbf{Web portals}: IvozProvider has 4-level web portals for all the
{\hyperref[operation_roles/index:operation\string-roles]{\sphinxcrossref{\DUrole{std,std-ref}{platform roles}}}}.

Both of these traffics are handled by \emph{Web portals} IvozProvider entity.

\end{itemize}


\section{Additional elements}
\label{architecture/index:additional-elements}
IvozProvider has multiple elements that are not exposed to the \emph{external world}
but play a crucial task.

The most remarkable profile is \textbf{database profile} that gathers all the
information of the platform and shares it between the majority of software packaged.
IvozProvider uses \href{https://www.mysql.com/}{MySQL database engine} for this task.

Another remarkable task is \textbf{asychronous tasks handler}: CDR must be parsed,
calls must be billed, recordings must be encoded, etc.


\section{Auxiliar elements}
\label{architecture/index:auxiliar-elements}
\textbf{Aux profile} runs software that, even though is not vital for calls placing,
makes IvozProvider mantainer's life much more easier.

In fact, without them, debugging problems would be much harder and the quality
of given service would be damaged.

IvozProvider ships:
\begin{itemize}
\item {} 
\textbf{Homer SIP capture}: This amazing software lets us capture all the SIP traffic
for later analysis, for obtaining statistics, call quality measuring, etc.
Visit \href{http://sipcapture.org/}{SIP Capture website} for more information.

\item {} 
\textbf{Graylog log viewer}: All logs of all IvozProvider profiles are stored and
shown with \href{https://www.graylog.org/}{Graylog} and divied in brands.

\item {} 
\textbf{Grafana graph dashboard}: \href{http://grafana.org/}{Grafana} lets us graph
everything. Literally.

\end{itemize}


\chapter{Platform roles}
\label{operation_roles/index:operation-roles}\label{operation_roles/index::doc}\label{operation_roles/index:platform-roles}
IvozProvider is a multilevel role provider solution.

The following images shows the different available levels and the
relation between them:

\noindent\sphinxincludegraphics{{operator_levels2}.png}

This section will explain each of the available roles, describing their
responsibilities and more important tasks.


\section{Global administator role}
\label{operation_roles/index:global-administator-role}
The global administator role (operator in the image) is usually done by
the installation responsible.

All options and platform features are visible to this role and usually is
in charge of its maintenance.

Their most important task is to \textbf{create Brands} and configure them so
they have the enough autonomy to properly use the platform:
\begin{itemize}
\item {} 
Configure their web access.

\item {} 
Configure their brand portal look and feel: themes, colors, etc.

\end{itemize}

Appart from their main task, their global visibility and total access
makes them responsible of:
\begin{itemize}
\item {} 
Monitor the platform so it keeps always UP \& RUNNING

\item {} 
Analyze platform logs to track possible errors.

\item {} 
Polish the security mechanisms to avoid external attacks.

\item {} 
Obtain global statistics of calls audio quality.

\item {} 
Increase the available resources of the platform as long as is needed:
\begin{itemize}
\item {} 
Increasing resources available in a standalone installation

\item {} 
Migrating, whenever required, to a distributed installation with multiple
AS, media relays, etc.

\end{itemize}

\end{itemize}

To sum up, \textbf{this role is the only one that has no limits within the
platform}, thats why \emph{God} is a term used in multiple places along this
documentation.

\begin{notice}{important}{Important:}
\emph{This role is responsible of maintain the platform*}, configuring
it for the correct behaviour. This role \textbf{doesn't have any kind
of limit} and \textbf{grants access} to the \textbf{brand operators}.
\end{notice}


\section{Brand andministator role}
\label{operation_roles/index:brand-andministator-role}
Brand operator can access a portal with less sections available compared
to the previous role. The general (God) administrator is in charge of
providing an URL with credentials for its brand portal.

The most important task for brand operator can be managed through this
portal: \textbf{create and configure companies so they can work properly}.

Due to brand operators are also resposible of billing their companies and
make sure the external cals are properly setup, it must also manage:
\begin{itemize}
\item {} 
Peering contracts with other IP providers for PSTN interconectivity.

\item {} 
Include all required company information for the billing process.

\item {} 
Pricing plans that will offer to their companies, that will determine how
match they pay for each call.

\item {} 
Setup the routes for each outgoing call types based on their final
destination

\item {} 
Create the invoices for each billing period and send them to their
clients.

\end{itemize}

As you can see, the task of brand operator has little in common with the
global operator, but their importance is vital so the final users can use
all the features includes in IvozProvider
\phantomsection\label{operation_roles/index:brand-responsibilities}
\begin{notice}{important}{Important:}
\textbf{To sum up}, the brand operators \textbf{grant access} to the
\textbf{company} administrators they serve and \textbf{configure the
platform to route, tarificate and bill their calls}.
\end{notice}


\section{Company administator role}
\label{operation_roles/index:company-administator-role}
The company administrator has access to the portal supplied by the brand
operator.

From its point of view, it has a virtual pbx in the cloud that must
configure for its users.

To accomplish that, it's required:
\begin{itemize}
\item {} 
Configure terminals, extensions and users.

\item {} 
Configure the DDI incoming process with the proper logic:
\begin{itemize}
\item {} 
Directly to an user

\item {} 
IVRs

\item {} 
Hunt groups

\item {} 
Faxes

\end{itemize}

\item {} 
Give access to the final users to their web portal, so they can configure
their profile options:
\begin{itemize}
\item {} 
Call forward

\item {} 
Do not disturb

\item {} 
Call waiting

\end{itemize}

\end{itemize}

\begin{notice}{important}{Important:}
To sum up, the company administrators are resposible for
\textbf{configuring the telephony system and make use of all the
features available in IvozProvider}.
\end{notice}


\section{Final user role}
\label{operation_roles/index:final-user-role}
The final user has two different kinds of credentials, both supplied by
its company administator:
\begin{itemize}
\item {} 
User portal access credetentials

\item {} 
SIP credentials used to register its terminal (or terminals) to IvozProvider

\end{itemize}

Through the user portal, it can browse their call registry and configure:
\begin{itemize}
\item {} 
Call forward

\item {} 
Do not disturb

\item {} 
Call waiting

\end{itemize}

On the other hand, the SIP creadentials allow the users to configure
their terminal (or terminals) to place and receive calls.

\begin{notice}{note}{Note:}
The same SIP credentials can be used in multiple devices at the same
time,generating what is known as \emph{parallel-forking}: whenever a call is
placed to an user, all the active devices will ring so the user can
answer the call from any of them.
\end{notice}

\begin{notice}{important}{Important:}
Final users are the ones that use and enjoy all the feature of
IvozProvider
\end{notice}


\chapter{Initial Installation}
\label{installation/index::doc}\label{installation/index:initial-installation}

\section{Installation Types}
\label{installation/install_types::doc}\label{installation/install_types:installation-types}

\subsection{Distributed Install}
\label{installation/install_types:distributed-install}
IvozProvider software is designed to run distributed between multiple systems
in what we call profiles:

Each profile is in charge of performing one of the platform functionalities:
\begin{itemize}
\item {} 
Data storage

\item {} 
SIP Proxy

\item {} 
Application Server

\item {} 
Web portal

\end{itemize}

For each of this profiles, there's a virtual package that will install all the
required dependencies (see {\hyperref[installation/debian_install:installing\string-profile\string-package]{\sphinxcrossref{\DUrole{std,std-ref}{Installing profile package}}}}).

You can install as many instances as you want for each profile, but take into
account, that while some of them are designed to scale horizontally (for
example: asterisk or media-relays) others will require aditional software so the
systems that have the same profile are syncronized (for example: database
replication or http request balancing).


\subsection{StandAlone Install}
\label{installation/install_types:standalone-install}
If you want a small installation to make a couple of tests or give a basic
service, we have designed all this configuration so they can work in a single
machine.

We have called this kind of installations \textbf{StandAlone} and we have also
created {\hyperref[installation/cd_install:automatic\string-iso\string-cd\string-image]{\sphinxcrossref{\DUrole{std,std-ref}{Automatic ISO CD image}}}} so you can install in a couple of minutes.


\section{Minimum requirements}
\label{installation/requirements:minimum-requirements}\label{installation/requirements::doc}

\subsection{System requirements}
\label{installation/requirements:system-requirements}
IvozProvider is designed to be installed using Debian GNU/Linux APT package
system.

\begin{notice}{important}{Important:}
It's recommended to install IvozProvider in a dedicated server
for the platform. Many of the installed software may not work properly with
other pre-installed services (like MySQL or DNS servers).
\end{notice}

For a StandAlone installation, we recommend at least:
\begin{itemize}
\item {} 
4 CPUs (x86\_64 or i386)

\item {} 
4 Gb memory

\item {} 
30GB HDD

\item {} 
1/2 public IP Addresses (read note behind)

\end{itemize}

\begin{notice}{note}{Note:}
Since version 1.2 it is possible to make both KamUsers and KamTrunks
share a unique public IP address. If so, \textbf{KamTrunks ports will be changed
from 5060 (TCP/UDP) to 7060 (TCP/UDP) and from 5061 (TCP) to 7061 (TCP)}.
\end{notice}

If you're not using a {\hyperref[installation/cd_install:automatic\string-iso\string-cd\string-image]{\sphinxcrossref{\DUrole{std,std-ref}{Automatic ISO CD image}}}} you will also need:
\begin{itemize}
\item {} 
Debian Jessie 8.0 base install

\item {} 
Internet access

\end{itemize}


\section{Debian packages install}
\label{installation/debian_install::doc}\label{installation/debian_install:debian-packages-install}
IvozProvider is designed to be installed and updated using Debian packages.
More exactly, the current release is ready to be installed on
\href{https://www.debian.org/releases/jessie}{Debian Jessie 8}.

It's recommended to use one of the \href{https://www.debian.org/releases/jessie/installmanual}{official installation guides} to install the minimum
base system. The rest of required  dependencies will be installed automatically
with IvozProvider meta packages.

No matter if you are installing a  {\hyperref[installation/install_types:standalone\string-install]{\sphinxcrossref{\DUrole{std,std-ref}{StandAlone Install}}}} or a
{\hyperref[installation/install_types:distributed\string-install]{\sphinxcrossref{\DUrole{std,std-ref}{Distributed Install}}}}, it's required to configure Irontec debian
repositories.


\subsection{APT Repository configuration}
\label{installation/debian_install:apt-repository-configuration}
Right now, two different repositories are used for the latest IvozProvider
release (called oasis) and it's frontend Klear release (called chloe).

\begin{Verbatim}[commandchars=\\\{\}]
\PYG{g+go}{cd /etc/apt/sources.list.d}
\PYG{g+go}{echo deb http://packages.irontec.com/debian oasis main extra \PYGZgt{} ivozprovider.list}
\PYG{g+go}{echo deb http://packages.irontec.com/debian chloe main \PYGZgt{} klear.list}
\end{Verbatim}

Optionally, we can add the repository key to check signed packages:

\begin{Verbatim}[commandchars=\\\{\}]
\PYG{g+go}{wget http://packages.irontec.com/public.key \PYGZhy{}q \PYGZhy{}O \PYGZhy{} \textbar{} apt\PYGZhy{}key add \PYGZhy{}}
\end{Verbatim}


\subsection{Installing profile package}
\label{installation/debian_install:installing-profile-package}
Once the repositories are configured, it will be required to select the proper
metapackage depending on the type of installation.
\begin{itemize}
\item {} \begin{description}
\item[{For a {\hyperref[installation/install_types:standalone\string-install]{\sphinxcrossref{\DUrole{std,std-ref}{StandAlone Install}}}}:}] \leavevmode\begin{itemize}
\item {} 
ivozprovider

\end{itemize}

\end{description}

\item {} 
For a {\hyperref[installation/install_types:distributed\string-install]{\sphinxcrossref{\DUrole{std,std-ref}{Distributed Install}}}}: one of the profile packages depending on the
role the machine will perform.
\begin{itemize}
\item {} 
ivozprovider-profile-data

\item {} 
ivozprovider-profile-proxy

\item {} 
ivozprovider-profile-portal

\item {} 
ivozprovider-profile-as

\end{itemize}

\end{itemize}

\begin{Verbatim}[commandchars=\\\{\}]
\PYG{g+go}{apt\PYGZhy{}get update}
\PYG{g+go}{apt\PYGZhy{}get install ivozprovider}
\end{Verbatim}


\subsection{Finish the installation}
\label{installation/debian_install:finish-the-installation}
Distributed installation require a couple manual configuration based on the
roles that are performing. Check \href{http://github.com/irontec/ivozprovider}{finishing role configuration} for more information.

Standalone installation have a menu that can be used to configure the basic
services used in IvozProvider. Most of the services are automatically configured
to work in the same machine with the default values.

This menu allows:
\begin{itemize}
\item {} 
Configure IP address(es) for SIP proxies

\item {} 
Default platform language

\item {} 
Administrator MySQL database password

\end{itemize}

It's possible to change any of this values anytime by running:

\begin{Verbatim}[commandchars=\\\{\}]
\PYG{g+go}{dpkg\PYGZhy{}reconfigure ivozprovider}
\end{Verbatim}

\begin{notice}{important}{Important:}
Any of the public IP addresses configured during the
installation will work to acces the web portal. Default credentials are
\textbf{admin / changeme}.
\end{notice}


\section{Automatic ISO CD image}
\label{installation/cd_install::doc}\label{installation/cd_install:automatic-iso-cd-image}
You can download one of the \href{https://github.com/irontec/ivozprovider}{IvozProvider Automatic ISO CD images} (generated using
\href{https://wiki.debian.org/Simple-CDD}{simplecdd}) in stable or nighly versions:

\begin{notice}{important}{Important:}
IMPORTANT: Automatic install CDs will format target machine disk!
\end{notice}
\begin{itemize}
\item {} 
Configure the target machine to boot from CD. It will display the Debian
GNU/Linux installation menu.

\end{itemize}

\begin{notice}{note}{Note:}
You can use graphic installation if you prefer, but the following
screenshots show the standard installation.
\end{notice}

\noindent\sphinxincludegraphics{{installcd-intro}.png}
\begin{itemize}
\item {} 
Choose installation language:

\end{itemize}

\noindent\sphinxincludegraphics{{installcd-language}.png}
\begin{itemize}
\item {} 
Choose location:

\end{itemize}

\noindent\sphinxincludegraphics{{installcd-location}.png}
\begin{itemize}
\item {} 
Choose date and time configuration:

\end{itemize}

\noindent\sphinxincludegraphics{{installcd-clock}.png}

\begin{notice}{note}{Note:}
At this point, a generic network configuration and disk partitioning
will be performed, and also a installation of base system.
\end{notice}
\begin{itemize}
\item {} 
Configure IvozProvider:

\end{itemize}

\noindent\sphinxincludegraphics{{installcd-ivozmenu}.png}

As mentioned in {\hyperref[installation/requirements:minimum\string-requirements]{\sphinxcrossref{\DUrole{std,std-ref}{Minimum requirements}}}} is required at least one public IP
address for User and Trunk SIP proxies. Remember that if you use only one,
KamTrunks will use different SIP ports to avoid collision.

You can set its addresses right now and configure the interfaces properly when
the system is fully installed. This menu can be displayed anytime after the
installation.

\noindent\sphinxincludegraphics{{installcd-proxyaddr}.png}

You can also configure default root MySQL password right now.

\begin{notice}{note}{Note:}
If you don't configure MySQL password, default password will be used
(changeme). You can still change it later.
\end{notice}

\noindent\sphinxincludegraphics{{installcd-mysql}.png}

And default language for portals:

\noindent\sphinxincludegraphics{{installcd-portallang}.png}

\begin{notice}{note}{Note:}
It is not require to configure all settings during initial
installation. In case any setting has been left without configuration a
warning dialog will be displayed.
\end{notice}

\noindent\sphinxincludegraphics{{installcd-warning}.png}

At last, select where the GRUB boot loader will be installed.

\noindent\sphinxincludegraphics{{installcd-grub}.png}

After the reboot, you are ready to access using the web portals!

\begin{notice}{important}{Important:}
Any of the public IP addresses configured during the
installation will work to acces the web portal. Default credentials are
\textbf{admin / changeme}.
\end{notice}


\section{Extra components}
\label{installation/extra_components::doc}\label{installation/extra_components:extra-components}

\subsection{G.729}
\label{installation/extra_components:g-729}
\begin{notice}{important}{Important:}
In some countries, you might have to pay royalty fees in order to
use G.729 codec to their patent holders. We're not legal advisors regarding
active or withdrawn world patents.
\end{notice}

You can use G.729 with IvozProvider, but installation must be done manually.
G.729 codec is optimized for each CPU type and version of asterisk, so each
installation may require a different codec module.

You can download codec from \href{http://asterisk.hosting.lv/}{here} under the
section Asterisk 13.

Once downloaded, move the \sphinxtitleref{.so} file to \textbf{/usr/lib/asterisk/modules/} and rename
it to \textbf{codec\_g729.so}

You can check the codec is valid by restarting asterisk and printing the
available codec translations using:

\begin{Verbatim}[commandchars=\\\{\}]
\PYG{g+go}{systemctl restart asterisk \PYGZsh{} Restart asterisk}
\PYG{g+go}{asterisk \PYGZhy{}rx \PYGZsq{}core show translation\PYGZsq{} \textbar{} grep 729}
\end{Verbatim}


\chapter{Making internal calls}
\label{internal_calls/index:making-internal-calls}\label{internal_calls/index::doc}
The goal of this block will be to configure IvozProvider in order to make
internal calls, using as the starting point the base installation described
in the previous step.

In order to archive a call between Alice and Bob, we have to do some task in
the three configuration levels described in {\hyperref[operation_roles/index:operation\string-roles]{\sphinxcrossref{\DUrole{std,std-ref}{Platform roles}}}}, and that
is why we have ordered the index in this blocks:


\section{Main Management}
\label{internal_calls/god_portal::doc}\label{internal_calls/god_portal:main-management}
\begin{notice}{important}{Important:}
Any of the 2 Public IP addresses configured during the
installation will work to acces the web portal. Default credentials are
\textbf{admin / changeme}.
\end{notice}

In this section will reference global administrator configuration options,
avaible in the menu (\textbf{Main management}) of the web portal (only visible to
God Admins):

\noindent{\hspace*{\fill}\sphinxincludegraphics{{bloque_god}.png}\hspace*{\fill}}


\subsection{Custom installation configuration}
\label{internal_calls/god_portal:custom-installation-configuration}
During the install process two IP addresses will be requested in order to run
the following two processes:


\subsubsection{Users SIP Proxy}
\label{internal_calls/god_portal:proxyusers}\label{internal_calls/god_portal:users-sip-proxy}
This is the SIP proxy exposed to the external world where users register their
terminals.

The value displayed in the section \textbf{Proxy users} will show the IP address
entered during the installation process.

\noindent\sphinxincludegraphics{{proxyusers}.png}


\subsubsection{Providers SIP proxy}
\label{internal_calls/god_portal:providers-sip-proxy}
This is the SIP proxy exposed to the external world in charge of connecting
the provider that brand aministrators will configure for \emph{peering}.

The value displayed in the section \textbf{Proxy trunk} will show the IP address
entered during the installation process.

\noindent\sphinxincludegraphics{{proxytrunks}.png}

\begin{notice}{note}{Note:}
Only the IP address will be entered as the port will be always 5060
(5061 for SIP over TLS).
\end{notice}

\begin{notice}{danger}{Danger:}
This 2 values can be changed from the portal, but they must always
have the same IP address that proxy process listen to requests.
\end{notice}


\subsection{Standard global configuration}
\label{internal_calls/god_portal:standard-global-configuration}
The installation process includes other global values that are the same in
any standalone IvozProvider installation and can also be checked from the
web interface.


\subsubsection{Application Servers}
\label{internal_calls/god_portal:application-servers}
The section \textbf{Application Servers} will list the IP address where the existing
Asterisk processes will listen for request, and like previously mentioned,
can scale horizontaly to adapt the platform for the required load.

Contrary to the Proxies, Asterisk is not exposed to the external world, so
for a standalone installation there will only be one listening at 127.0.0.1.

\noindent\sphinxincludegraphics{{app_servers}.png}

\begin{notice}{note}{Note:}
The listening port will not be displayed in the field because it will
always be 6060 (UDP).
\end{notice}

\begin{notice}{important}{Important:}
As soon as another Application Server is added, the proxies will
try to balance load using it. If no response is received from added
Application server, it will be disabled automatically.
\end{notice}


\subsubsection{Media relay}
\label{internal_calls/god_portal:media-relay}
Media relays are in charge of bridging RTP traffic of established calls. Like
the Application Servers, they can scale horizontally as much as required.

Media relays are organized in groups so they can be assigned to a company. Each
element of the group has a \textbf{metric} that allows non-equal load balancing
within the same group (i.e. media-relay1 metric 1; media-relay2 metric 2:
the second media relay will handle two times the calls than the first one).

\begin{notice}{hint}{Hint:}
The static assigment of media relay groups is not the common practice
but allow us to assign strategic resources to companies that need a warranted
service. The most common usage of this \textbf{groups of media relays} is to
place them near the geographic area of the company (usually far from the
rest of the platform systems) in order to reduce \textbf{latencies} in their
conversations.
\end{notice}

In a standalone installation, only one media relay group will be exist:

\noindent\sphinxincludegraphics{{media_relay_groups}.png}

By default this group only has a media server:

\noindent\sphinxincludegraphics{{media_relays}.png}

\begin{notice}{note}{Note:}
The address displayed is the control socket, not the SDP address that
will be included during SIP negociation. By default this alone media-relay
will share the same IP address that the User's SIP proxy.
\end{notice}


\subsubsection{SIP Domains}
\label{internal_calls/god_portal:sip-domains}\label{internal_calls/god_portal:god-sipdomains}
The section \textbf{Domains} will display the SIP domains that points to our two
public IP addresses.
\begin{itemize}
\item {} 
Users SIP Proxy IP address

\item {} 
Trunks SIP Proxy IP address

\end{itemize}

After the initial installation, there will be two domains, one for each address:

\noindent\sphinxincludegraphics{{domain_list_local}.png}

This domains will be used internally by a builtin DNS server included in the
solution.

\begin{notice}{attention}{Attention:}
As mentioned in the section {\hyperref[internal_calls/brand_portal:domain\string-per\string-company]{\sphinxcrossref{\DUrole{std,std-ref}{Company SIP Domain}}}}, each
company will require a DNS pointing to the users SIP proxy. Once configured,
the domain will be displayed in this list so global administrator can check
what domains are registered for each company.
\end{notice}


\subsubsection{Brands}
\label{internal_calls/god_portal:brands}
After the initial installation, the platform will have an already created brand
called DemoBrand. Let's inspect its parameters:
\begin{description}
\item[{Name\index{Name|textbf}}] \leavevmode\phantomsection\label{internal_calls/god_portal:term-name}
Sets the name for this brand.

\item[{NIF\index{NIF|textbf}}] \leavevmode\phantomsection\label{internal_calls/god_portal:term-nif}
Number used in this brand's invoices.

\item[{Logo\index{Logo|textbf}}] \leavevmode\phantomsection\label{internal_calls/god_portal:term-logo}
Used as default logo in invoices and in portals (if they don't specify
another logo).

\item[{Invoice data\index{Invoice data|textbf}}] \leavevmode\phantomsection\label{internal_calls/god_portal:term-invoice-data}
Data included in invoices created by this brand.

\item[{Mail data\index{Mail data|textbf}}] \leavevmode\phantomsection\label{internal_calls/god_portal:term-mail-data}
Display name and from address used in external emails for this brand
(faxes, voicemail, etc.)

\item[{SIP domain\index{SIP domain|textbf}}] \leavevmode\phantomsection\label{internal_calls/god_portal:term-sip-domain}
Introduced in 1.4. Domain pointing to Users SIP proxy used by all the
Retail Accounts of this brand.

\item[{Recordings\index{Recordings|textbf}}] \leavevmode\phantomsection\label{internal_calls/god_portal:term-recordings}
Configures a limit for the size of recordings of this brand. A
notification is sent to configured address when 80\% is reached and
older recordings are rotated when configured size is reached.

\item[{Features\index{Features|textbf}}] \leavevmode\phantomsection\label{internal_calls/god_portal:term-features}
Introduced in 1.3, lets god operator choose the features of the created
brand. An equivalent configuration is available in Companies, to choose
between the ones that god operator gave to your Brand. Related sections
are hidden consequently.

\end{description}

\begin{notice}{hint}{Hint:}
Some features (currently invoices and billing) are related to brand and
cannot be assigned to companies. Remaining features are related to
companies and lets the brand operator to assign them to its companies.
\end{notice}

\begin{notice}{warning}{Warning:}
Disabling billing hides all related sections and assumes that an
external element will set a price for calls (external tarification
module is needed, ask for it!).
\end{notice}

\begin{notice}{note}{Note:}
Disabling invoices hides related sections, assuming you will use an
external tool to generate them.
\end{notice}


\subsection{Emulate the Demo brand}
\label{internal_calls/god_portal:emulate-the-demo-brand}
As mentioned above, the initial installation will have an already created brand
called DemoBrand, that will be used for our goal: to have 2 telephones registered
that can call each other.

Before going to the next section, is quite important to understand how the
\textbf{emulation} works.
\begin{itemize}
\item {} 
As global operator, you have access to the menu \textbf{Main management} only
visible to \emph{God} administators.

\item {} 
Apart from that menu, you will also have access to the \textbf{Brand configuration}
and \textbf{Company configuration} that will look more or less like this:

\end{itemize}

\noindent{\hspace*{\fill}\sphinxincludegraphics{{emular_marca_prev}.png}\hspace*{\fill}}
\begin{itemize}
\item {} 
Check following button

\end{itemize}

\noindent{\hspace*{\fill}\sphinxincludegraphics{{emular_marca}.png}\hspace*{\fill}}
\begin{itemize}
\item {} 
When pressed, a popup will be displayed:

\end{itemize}

\noindent{\hspace*{\fill}\sphinxincludegraphics{{emular_marca2}.png}\hspace*{\fill}}
\begin{itemize}
\item {} 
After selecting the DemoBrand brand, the icon will change and shows the
emulted brand:

\end{itemize}

\noindent{\hspace*{\fill}\sphinxincludegraphics{{emular_marca3}.png}\hspace*{\fill}}
\begin{itemize}
\item {} 
The upper right corner of the portal will also display the brand that is being
emulted:

\end{itemize}

\noindent{\hspace*{\fill}\sphinxincludegraphics{{emular_marca4}.png}\hspace*{\fill}}


\subsubsection{What emulation means}
\label{internal_calls/god_portal:what-emulation-means}
Basically, that \textbf{everything in the menu `Brand configuration' will be relative
to the chosen brand} and is \textbf{exactly} the same menu entries that the brand
operator will see using its brand portal.

\begin{notice}{tip}{Tip:}
Ok, ok. maybe exactly is not totally accurate. The global operator is
able to see some fields in some screens that other admins cann't (i.e. On
Company edit screen, fields like `Media relays' or `Application server' are
only configurable by the global operator.
\end{notice}


\section{Brand Configuration}
\label{internal_calls/brand_portal::doc}\label{internal_calls/brand_portal:brand-configuration}
We need that the default DemoBrand have a company with at least 2 users. In
order to archive this we will require little configuration in this section.

In fact, if we check \textbf{Companies} in the brand menu, we'll discover that there
is already an existing \emph{DemoCompany} that we can use to fulfill our desired
goal :)

\noindent\sphinxincludegraphics{{company_list}.png}

Only a thing is required to configure for this company, marked as edit in the
previous image.


\subsection{Company SIP Domain}
\label{internal_calls/brand_portal:company-sip-domain}\label{internal_calls/brand_portal:domain-per-company}
As mentioned in the previous section, is \textbf{required} that each of the companies
have a public domain that resolves to the configured IP address for
{\hyperref[internal_calls/god_portal:proxyusers]{\sphinxcrossref{\DUrole{std,std-ref}{Users SIP Proxy}}}}.

\begin{notice}{note}{Note:}
DNS register can be type A (supported by all the hardphones/softphones
) or even NAPTR+SRV.
\end{notice}

Once the domain has been configured (by means that are out of scope of this
document), it will be enought to write it in our company configuration:

\noindent\sphinxincludegraphics{{set_domain}.png}

Once the company has been saved, the domain will be also displayed in the list
{\hyperref[internal_calls/god_portal:god\string-sipdomains]{\sphinxcrossref{\DUrole{std,std-ref}{previously mentioned}}}}:

\noindent\sphinxincludegraphics{{domain_list}.png}

\begin{notice}{attention}{Attention:}
It's important to understand this block. {\hyperref[internal_calls/brand_portal:dnshack]{\sphinxcrossref{\DUrole{std,std-ref}{Unless we've a
single company registered}}}}, wihout a DNS domain pointing to our
users proxy IP address, everything will fail.
\end{notice}

This is a good sign for the domain we have configured right now, replacing the
10.10.3.10 with the public address we have used to configure {\hyperref[internal_calls/god_portal:proxyusers]{\sphinxcrossref{\DUrole{std,std-ref}{Users SIP Proxy}}}}.

\noindent\sphinxincludegraphics{{dominio_bien_configurado}.png}

\begin{notice}{danger}{Danger:}
Have we stressed enough that without a properly configured DNS
pointing to the Users proxy IP address nothing will work?
\end{notice}


\subsubsection{I have no time for a DNS registry}
\label{internal_calls/brand_portal:dnshack}\label{internal_calls/brand_portal:i-have-no-time-for-a-dns-registry}
Everything we have said is true: as we create new brands and brands create new
companies, each of them will need a DNS registry.

But the first company of the platform is quite special and can take over the IP
address of the proxy to use it as a domain:

\noindent\sphinxincludegraphics{{fake_domain}.png}

Although it is not a domain, but being used like it was, it will be displayed
in Domain section:

\noindent\sphinxincludegraphics{{fake_domain2}.png}

\begin{notice}{tip}{Tip:}
It’s important to understand the this trick is only valid for the first
company of the platform ;)
\end{notice}


\subsection{Remaining parameters}
\label{internal_calls/brand_portal:remaining-parameters}
These are remaining relevant parameters configured in Companies section:
\begin{description}
\item[{Name\index{Name|textbf}}] \leavevmode\phantomsection\label{internal_calls/brand_portal:term-name}
Sets the name for this company.

\item[{NIF\index{NIF|textbf}}] \leavevmode\phantomsection\label{internal_calls/brand_portal:term-nif}
Number used in this company's invoices.

\item[{Invoice data\index{Invoice data|textbf}}] \leavevmode\phantomsection\label{internal_calls/brand_portal:term-invoice-data}
Data included in invoices created by this brand.

\item[{Outbound prefix\index{Outbound prefix|textbf}}] \leavevmode\phantomsection\label{internal_calls/brand_portal:term-outbound-prefix}
Some companies are used to dial an aditional prefix for outgoing calls.

\item[{Outgoing DDI\index{Outgoing DDI|textbf}}] \leavevmode\phantomsection\label{internal_calls/brand_portal:term-outgoing-ddi}
Introduced in 1.3, this setting selects a DDI for outgoing calls
of this company, if it is no overridden in a lower level (e.g. user level)

\item[{Media relay set\index{Media relay set|textbf}}] \leavevmode\phantomsection\label{internal_calls/brand_portal:term-media-relay-set}
As mentioned above, media-relay can be grouped in sets to reserve capacities
or on a geographical purpose. This section lets you assign them to companies.

\item[{Distribute Method\index{Distribute Method|textbf}}] \leavevmode\phantomsection\label{internal_calls/brand_portal:term-distribute-method}
`Hash based' distributes calls hashing a parameter that is unique per
company/retail, `Round robin' distributes calls equally between AS-es and
`static' is used for debugging purposes.

\item[{Application Server\index{Application Server|textbf}}] \leavevmode\phantomsection\label{internal_calls/brand_portal:term-application-server}
If `static' \emph{distribute method} is used, select an application server here.

\item[{Recordings\index{Recordings|textbf}}] \leavevmode\phantomsection\label{internal_calls/brand_portal:term-recordings}
Configures a limit for the size of recordings of this company. A
notification is sent to configured address when 80\% is reached and
older recordings are rotated when configured size is reached.

\item[{Features\index{Features|textbf}}] \leavevmode\phantomsection\label{internal_calls/brand_portal:term-features}
Introduced in 1.3, lets brand operator choose the features of the company.
Related sections are hidden consequently and the company cannot use them.

\end{description}

Most of the features are self-explanatory, but \textbf{voice notification} deserves
an explanation: if you enable them, when a call fails, the user will listen a
locution explaining what ocurred (``you have no permissions to place this call'',
``the call cannot be billed'', etc.)

\begin{notice}{warning}{Warning:}
Recordings rotation happens at two levels: brand and company. This
means that \textbf{a company's recordings can be rotated even though its limit
has not arrived (or even it has no limit) if brand's limit applies first}.
\end{notice}

\begin{notice}{error}{Error:}
Again: recordings rotation happens at two levels: brand and company. This
means that \textbf{a company's recordings can be rotated even though its limit
has not arrived (or even it has no limit) if brand's limit applies first}.
\end{notice}

\begin{notice}{hint}{Hint:}
To avoid this, make sure that the sum of all companies does not exceed
the size assigned to your brand and make sure that all companies has
a size configured (if 0, it has unlimited size).
\end{notice}

Both \textbf{Distribute method} and \textbf{Application Server} are only visible for God
Administrator.

\begin{notice}{warning}{Warning:}
`Round-robin' distribute method is reserved for huge companies/retails
whose calls cannot be handled in a single AS. \textbf{Use `Hash based'
for remaining ones}.
\end{notice}


\subsection{Emulate Demo company}
\label{internal_calls/brand_portal:emulate-company}\label{internal_calls/brand_portal:emulate-demo-company}
The company emulation process is the same as the brand emulation, with the
difference that it filters the block ‘Company Configuration’ insted of
‘Brand Configuration’.

\noindent{\hspace*{\fill}\sphinxincludegraphics{{emulate_company}.png}\hspace*{\fill}}

\noindent{\hspace*{\fill}\sphinxincludegraphics{{emulate_company2}.png}\hspace*{\fill}}

Once the company has been emulated, the top right corner of the portal will
show that we are in the right path :)

\noindent{\hspace*{\fill}\sphinxincludegraphics{{emular_empresa}.png}\hspace*{\fill}}


\section{Company Configuration}
\label{internal_calls/company_portal:company-configuration}\label{internal_calls/company_portal::doc}
We're close to make our fist call in our fresh installed IvozProvider, there
are only 6 steps to configure in our DemoCompany company.
\begin{itemize}
\item {} 
2 terminals

\item {} 
2 extensions

\item {} 
2 users

\end{itemize}


\subsection{Creating Terminals}
\label{internal_calls/company_portal:creating-terminals}
Go to the terminal section and... voilà! We already have 2 terminals created:

\noindent\sphinxincludegraphics{{terminals}.png}


\subsection{Creating Extensions}
\label{internal_calls/company_portal:creating-extensions}
Then we go to extensions, just to check that we have 2 extensions already
created for us:

\noindent\sphinxincludegraphics{{extensions}.png}

Nothing more to do in this section, let's go the next one!


\subsection{Creating Users}
\label{internal_calls/company_portal:creating-users}
As expected, we also have 2 created users:

\noindent\sphinxincludegraphics{{users}.png}

At this point, we have everthing ready make a call between this two users:
Alice and Bob.


\section{SIP Terminal configuration}
\label{internal_calls/configure_sipuacs:sip-terminal-configuration}\label{internal_calls/configure_sipuacs::doc}
The last thing we need is 2 SIP terminals (hardphones, softphones or even
mobile applications) and configure them as follows:

\textbf{ALICE}
\begin{itemize}
\item {} 
\textbf{User}: alice

\item {} 
\textbf{Password}: alice

\item {} 
\textbf{Domain}: users.democompany.com (or the IP if we are using {\hyperref[internal_calls/brand_portal:dnshack]{\sphinxcrossref{\DUrole{std,std-ref}{the DNS
trick}}}})

\end{itemize}

\textbf{BOB}
\begin{itemize}
\item {} 
\textbf{User}: bob

\item {} 
\textbf{Password}: bob

\item {} 
\textbf{Domain}: users.democompany.com (or the IP if we are using {\hyperref[internal_calls/brand_portal:dnshack]{\sphinxcrossref{\DUrole{std,std-ref}{the DNS
trick}}}})

\end{itemize}

\begin{notice}{tip}{Tip:}
Sometimes the user and domain is configured in a single option. In this
case we should enter \href{mailto:alice@users.democompany.com}{alice@users.democompany.com} and \href{mailto:bob@users.democompany.com}{bob@users.democompany.com}
(or the IP if we are using {\hyperref[internal_calls/brand_portal:dnshack]{\sphinxcrossref{\DUrole{std,std-ref}{the DNS trick}}}})
\end{notice}

After configuring the terminals, Alices should be able to call Bob only by
dialing 102 in her terminal.


\chapter{Receive external calls}
\label{external_incoming_calls/index::doc}\label{external_incoming_calls/index:receive-external-calls}
The goal of this block will be configure IvozProvider to receive incoming
external calls.

In order tho archive this, this steps will be followed:


\section{Numeric transformations}
\label{external_incoming_calls/numeric_transformations:id1}\label{external_incoming_calls/numeric_transformations::doc}\label{external_incoming_calls/numeric_transformations:numeric-transformations}

\subsection{Concept}
\label{external_incoming_calls/numeric_transformations:concept}
\textbf{IvozProvider} is designed to provide service \textbf{anywhere in the planet}, not
only the original country where the platform is installed.

A very important contept to archive this goal is the numeric transformation,
that \textbf{adapts the different number format systems of the countries of the world}
defined in \href{https://www.itu.int/rec/T-REC-E.164/es}{E.164} \textbf{to a neutral
format}.

The section that allows the brand operator to configure all the \textbf{numeric
transformations} is:

\noindent{\hspace*{\fill}\sphinxincludegraphics{{numeric_transformations_section}.png}\hspace*{\fill}}

There are two different transformation scenarios:


\subsubsection{Incoming transformations}
\label{external_incoming_calls/numeric_transformations:incoming-transformations}
When a new call is received in IvozProvider matching a provider that has been
configured for \emph{peering}, we must adapt the numbers that make reference to:
\begin{itemize}
\item {} 
Origin of the call

\item {} 
Destination of the call

\end{itemize}

Depending on the country of the provider, the international numbers will have
a format or another. In this case, the spanish provider will use, for example:
\begin{itemize}
\item {} 
00 + 33 + number belonging to France

\item {} 
It's possible that the international numbers came without the 00 code.

\item {} 
It's possible that, if the call comes from the same country that the provider,
the number comes without the calling code (911234567 insted of 00 + 34 +
911234567 for Spain).

\end{itemize}

For an Ukranian provider, that doesn't use the 00 as international code:
\begin{itemize}
\item {} 
It will use 810 + 33 + number belonging to France.

\item {} 
It's possible that even part of the internation code (00 in most of the
countries of the world) the provider use specific codes as prefix.

\end{itemize}

The goal of the incoming transformation is that, no matter what numeric system
the provider uses, the number will end in a general and common format.
\phantomsection\label{external_incoming_calls/numeric_transformations:e164}
\begin{notice}{important}{Important:}
This common format is usually called E.164 and shows the numbers
without internation code, but with country calling code: i.e. 34911234567
\end{notice}


\subsubsection{Outgoing transformations}
\label{external_incoming_calls/numeric_transformations:outgoing-transformations}
In the same whay the origin and destination must adaptat incoming numbers, it
will be required to adapt outgoing dialed numbers to properly work with each
of the providers that will route our call.

For example, for a number with spanish number system:
\begin{itemize}
\item {} 
\emph{Spanish provider}: Destination will come in E164 (34911234567) and for this
provider, we can remove the calling code (will understand it belongs to
its country), so the number sent to them will be 911234567.

\item {} 
\emph{French provider}: The destination will come in E164 (34911234567) and we must
add the international code for France, so the number sent to them will be
0034911234567.

\end{itemize}

\begin{notice}{note}{Note:}
To sum up, we aim to send the origin and destination in the format the
provider is expecting.
\end{notice}

\begin{notice}{tip}{Tip:}
Numeric transformation uses \href{https://es.wikipedia.org/wiki/Expresi\%C3\%B3n\_regular}{simple regular expresions} to describe the
changes done to the numbers. You can find multiple tutorials on net with the
basic regular expression format.
\end{notice}


\subsection{`National provider' transformations}
\label{external_incoming_calls/numeric_transformations:national-provider-transformations}
IvozProvider comes with an automatic transformation rules generator that fits
with most of the countries.

In order to create a new set of transformations for spanish provider:

\noindent{\hspace*{\fill}\sphinxincludegraphics{{numeric_transformations_auto}.png}\hspace*{\fill}}

The rules that has been auto-created will transform the numbers for spanish
providers that follow this rules:
\begin{itemize}
\item {} 
A spanish number: Neither international nor calling code (34).

\item {} 
Not a spanish number: International code (00) and calling code (34).

\end{itemize}

The numeric transformation \emph{sets} must be assigned to \textbf{Peering Contracts}, as
shown in the following section. This \emph{set} can be shared by multiple spanish
providers.

Let's check this \emph{set} to understand what transformation rule does:

\noindent\sphinxincludegraphics{{numeric_transformations}.png}

\begin{notice}{attention}{Attention:}
The automatic rule generation will create 8 common rules based on
the given parameters. This rules can be edited later to match the provider
requirements.
\end{notice}


\subsubsection{Spanish incoming transformation}
\label{external_incoming_calls/numeric_transformations:spanish-incoming-transformation}
Displayed in blue in the previous image:
\begin{itemize}
\item {} 
Left called/destination

\item {} 
Right callee/origin

\end{itemize}

The same rules will be applied for the origin and destination:

\noindent\sphinxincludegraphics{{numeric_transformations2}.png}
\begin{itemize}
\item {} 
The \textbf{metric} field will be used to order the rules (smaller first).
\begin{itemize}
\item {} 
If a rule doesnt \emph{match}, the next rule is evaluated.

\item {} 
If a rule \emph{matches}, no more rules are evaluated.

\item {} 
If no rule \emph{matches}, no change is applied.

\end{itemize}

\item {} 
The \textbf{Search} field is evaluated agains the number (depending of the
tranformation type it will be destination or origin).
\begin{itemize}
\item {} 
Metric 1: Starting with (\textasciicircum{}) 00 or `+', followed by a digit between 1 and 9
and only digits between 0 and 9 until the end (\$).

\item {} 
Metric 2: Starting with a digit bewteen 1 and 9, followed by eight digits
between 0 and 9 until the end(\$).

\end{itemize}

\item {} 
The \textbf{Replace} field will use the capture groups that matched the Search
field (displayed between brackets, 1 for the first one, 2 for the second
one, and so on) to determine how the number will end.
\begin{itemize}
\item {} 
Metric 1: The number will be converted just to the second capture (2)

\item {} 
Metric 2: The number will be converted to the first capture with 34 prefix

\end{itemize}

\item {} 
Or \emph{in other words}:
\begin{itemize}
\item {} 
Metric 1: Strip the international code (00 or `+')

\item {} 
Metric 2: Add 34 to spanish numbers that comes without calling code.

\end{itemize}

\end{itemize}


\subsubsection{Spanish outgoing transformation}
\label{external_incoming_calls/numeric_transformations:spanish-outgoing-transformation}
\noindent\sphinxincludegraphics{{numeric_transformations3}.png}

Following the same logic, this 2 rules make the change of the outgoig external
destination numbers:
\begin{itemize}
\item {} 
Metric 1: If the number starts with 34 prefix, remove it. This converts
E.164 spanish numbers to national format.

\item {} 
Metric 2: If the number doesn't start with 34 prefix, add 00. This converts
E.164 international numbers to spanish international numbers.

\end{itemize}

\begin{notice}{attention}{Attention:}
\textbf{To sum up}: numeric tranformation can adapt origin and
destination numbers to E.164 for the platform, and to providers expected
formats, based on regular expresions and metric that can be grouped in \emph{sets}
to be shared between multiple \textbf{Peering Contracts}.
\end{notice}


\section{Configuring a Peering Contract}
\label{external_incoming_calls/peering_contracts:configuring-a-peering-contract}\label{external_incoming_calls/peering_contracts::doc}\label{external_incoming_calls/peering_contracts:peering-contracts}
We understand a \textbf{Peering contract} the agreeming between a \textbf{Brand Operator}
and a VoIP Provider to make and receive calls.

IvozProvider is ready to integrate with IP providers created on the secteion
\textbf{Perring contracts}:

\noindent{\hspace*{\fill}\sphinxincludegraphics{{peeringcontract_section}.png}\hspace*{\fill}}


\subsection{Peering contract Basic data}
\label{external_incoming_calls/peering_contracts:peering-contract-basic-data}
These are the basic information of a Peering contract:

\noindent{\hspace*{\fill}\sphinxincludegraphics{{peeringcontract}.png}\hspace*{\fill}}

If we edit it, well see something like this:

\noindent{\hspace*{\fill}\sphinxincludegraphics{{peeringcontract_edit}.png}\hspace*{\fill}}
\begin{description}
\item[{Name\index{Name|textbf}}] \leavevmode\phantomsection\label{external_incoming_calls/peering_contracts:term-name}
Used to reference this Peering contract.

\item[{Description\index{Description|textbf}}] \leavevmode\phantomsection\label{external_incoming_calls/peering_contracts:term-description}
Optional field with any required extra information.

\item[{Numeric Transformation\index{Numeric Transformation|textbf}}] \leavevmode\phantomsection\label{external_incoming_calls/peering_contracts:term-numeric-transformation}
Transformation that will be applied to the origin and destination of the
incoming and outgoing numbers that use this Peering contact
(see {\hyperref[external_incoming_calls/numeric_transformations:numeric\string-transformations]{\sphinxcrossref{\DUrole{std,std-ref}{Numeric transformations}}}}).

\item[{External tarification\index{External tarification|textbf}}] \leavevmode\phantomsection\label{external_incoming_calls/peering_contracts:term-external-tarification}
This setting requires the external tarification module and allows
tarification on special numbers. This module is not standard so don't
hesitate in {\hyperref[intro/getting_help:getting\string-help]{\sphinxcrossref{\DUrole{std,std-ref}{contact us}}}} if you are interested.

\end{description}

\begin{notice}{important}{Important:}
Fields marked with a red start are mandatory.
\end{notice}


\subsection{Peer Servers}
\label{external_incoming_calls/peering_contracts:peer-servers}
A \textbf{Peer Server} is a SIP server associated to an IP Provider.
In order to add some Peer Servers for the Peer Contract that
have just created, we only have to use the following section:

\noindent{\hspace*{\fill}\sphinxincludegraphics{{peerservers}.png}\hspace*{\fill}}

As displayed by the zero counter, no Peer Serve has been created yet:

\noindent{\hspace*{\fill}\sphinxincludegraphics{{peerservers_add}.png}\hspace*{\fill}}
\begin{description}
\item[{Name\index{Name|textbf}}] \leavevmode\phantomsection\label{external_incoming_calls/peering_contracts:term-4}
Used to identify this Peer Server

\item[{Description\index{Description|textbf}}] \leavevmode\phantomsection\label{external_incoming_calls/peering_contracts:term-5}
Optional field with any required extra information.

\item[{SIP Proxy\index{SIP Proxy|textbf}}] \leavevmode\phantomsection\label{external_incoming_calls/peering_contracts:term-sip-proxy}
IP address (or DNS registry) of the Peer Server. You can also specify
a port if it's different from 5060.

\item[{URI Scheme\index{URI Scheme|textbf}}] \leavevmode\phantomsection\label{external_incoming_calls/peering_contracts:term-uri-scheme}
Supported schemes are sip and sips. Use `sip' in case of doubt.

\item[{Transport\index{Transport|textbf}}] \leavevmode\phantomsection\label{external_incoming_calls/peering_contracts:term-transport}
Supported transport protocols. Use `udp' in case of doubt.

\item[{Outbound Proxy\index{Outbound Proxy|textbf}}] \leavevmode\phantomsection\label{external_incoming_calls/peering_contracts:term-outbound-proxy}
Usually this is left empty. It can be filled with the IP address of the
\textbf{SIP Proxy} domain (to avoid DNS resolution, but keeping the domain
in the SIP messages). It works like a web proxy: instead of sending the
SIP messages to destination \textbf{SIP Proxy}, they will be sent to the
IP:PORT of this field.

\item[{Requires Authentication\index{Requires Authentication|textbf}}] \leavevmode\phantomsection\label{external_incoming_calls/peering_contracts:term-requires-authentication}
Some Peering contracts validate our platform by IP, others require
each session that we want to establish. For this last case, this section
allows to configure user and password for this authentication.

\item[{Call Origin Header\index{Call Origin Header|textbf}}] \leavevmode\phantomsection\label{external_incoming_calls/peering_contracts:term-call-origin-header}
Some Providers get origin from SIP From header. Others use the From
header for accounting and need extra headers to identify the origin.
In case of doubt leave \textbf{PAI} checked.

\item[{R-URI Transformations before numeric transformations\index{R-URI Transformations before numeric transformations|textbf}}] \leavevmode\phantomsection\label{external_incoming_calls/peering_contracts:term-r-uri-transformations-before-numeric-transformations}
This setting allow static changes to the destination of the calls before
applying numeric transformation rules mentioned in
{\hyperref[external_incoming_calls/numeric_transformations:numeric\string-transformations]{\sphinxcrossref{\DUrole{std,std-ref}{Numeric transformations}}}}. Some digits can be stripped from the
begining, add a prefix, or even, add extra parameters to the URI
followinging the given format. In case of doubt, leave empty.

\item[{From header customization\index{From header customization|textbf}}] \leavevmode\phantomsection\label{external_incoming_calls/peering_contracts:term-from-header-customization}
For those providers that show origin in other headers (PAI/RPID), it is
possible that request that From User have the account code being used
and from domain their SIP domain. In case of doubt, leave empty.

\end{description}

\begin{notice}{tip}{Tip:}
There are many fields to establish \emph{peering} with multiple kind of
providers, but usually with the name and SIP Proxy will be enough (for
those that validate our platform by IP) and Authentication (for those that
won't).
\end{notice}

\begin{notice}{warning}{Warning:}
In case of defining multiple Peering Servers for a single
Peering Contract, IvozProvider will balance and failover using all of them.
Like with Application Servers, it will disable those who doesn't respond to
our requests.
\end{notice}


\subsection{SIP Register}
\label{external_incoming_calls/peering_contracts:sip-register}
Some providerss require a \href{https://tools.ietf.org/html/rfc3261\#section-10}{SIP Register} active in order to receive
incoming calls to our DDIs. Some of them, even require this register in order
to process our outgoing calls through their services.

\begin{notice}{note}{Note:}
IvozProvider supports any kind of \emph{peering}, but we highly recomend
\emph{peer to peer peerings}: without authentication, without registry and
validated by IP. This will avoid unnecessary traffic (authentication in each
session and preriodic registers) and simplifies its configuration, just by
leaving most of the fields by default.
\end{notice}

For this reason, IvozProvider allows periodic SIP register configuration in the
following section:

\noindent\sphinxincludegraphics{{sip_registers}.png}

If we create a new one, the following screen will be displayed:

\noindent\sphinxincludegraphics{{sip_registers_add}.png}
\begin{description}
\item[{Username\index{Username|textbf}}] \leavevmode\phantomsection\label{external_incoming_calls/peering_contracts:term-username}
Account number or similar provider by the provider that requires SIP
register.

\item[{Domain\index{Domain|textbf}}] \leavevmode\phantomsection\label{external_incoming_calls/peering_contracts:term-domain}
Domain or IP of the registar server. Usually the same as the SIP proxy
of the Peer server.

\item[{DDI\index{DDI|textbf}}] \leavevmode\phantomsection\label{external_incoming_calls/peering_contracts:term-ddi}
This will be sent in the SIP Contact header and must be unique in all
the platform. For Peering contracts with an associated DDI, it is
recommended to enter that DDI. In case of multiples DDI for the same
Peering Contract, use any of them. If no DDI is associated with this
Peering Contract just enter an unique numeric value.

\item[{User\index{User|textbf}}] \leavevmode\phantomsection\label{external_incoming_calls/peering_contracts:term-user}
Authentication user. Most of the time it's the same as username, so
it's recommended to leave empty.

\item[{Register server URI\index{Register server URI|textbf}}] \leavevmode\phantomsection\label{external_incoming_calls/peering_contracts:term-register-server-uri}
Usually this can be left empty, as it can be obtained from the
Domain. If it is not the case, enter the IP address with the `sip:'
prefix.

\item[{Realm\index{Realm|textbf}}] \leavevmode\phantomsection\label{external_incoming_calls/peering_contracts:term-realm}
Leave empty to accept the authentication realm proposed by the provider.
Define only if you are familiar to the authentication mechanism used
in SIP.

\item[{Expire\index{Expire|textbf}}] \leavevmode\phantomsection\label{external_incoming_calls/peering_contracts:term-expire}
Default suggested register expire time.

\end{description}

\begin{notice}{tip}{Tip:}
Similar to the Peer Servers, there are lots of fields in the screen.
You must have into account that most of the provider doesn't require register
, and those who does, will only use user, domain and password.
\end{notice}

Once we have an agreement with a VoIP provider and we have configured it in
the \emph{peering} section, only two task are pending:


\section{Configuring an external DDI}
\label{external_incoming_calls/configure_ddi:settingup-ddi}\label{external_incoming_calls/configure_ddi:configuring-an-external-ddi}\label{external_incoming_calls/configure_ddi::doc}
The brand operator, responsible of this \emph{peering} agreements with VoIP providers
, has the task to create the DDIs for each provider.

To accomplish this, following section can be used:

\noindent{\hspace*{\fill}\sphinxincludegraphics{{ddis_company_section}.png}\hspace*{\fill}}

Notice that in order to access this section, the brand operator (or \emph{god})
must have emulated the proper company and access the menu section \textbf{Company
Configuration}.

\begin{notice}{attention}{Attention:}
Section \textbf{Company configuration \textgreater{} DDIs} is different when the
company administrator access than the displayed data when a global or brand
administrator does. Company administrator are unable to create or delete
DDIs, just edit the one created by the brand or god administrator.
\end{notice}

The section \textbf{Brand configuration \textgreater{} DDIs} is a \emph{read-only} display of all the
DDIs of the brand, associated with the different companies.

\noindent{\hspace*{\fill}\sphinxincludegraphics{{ddis_brand_section}.png}\hspace*{\fill}}

Taking into account this concepts, we create a new DDI and fill the required
fields:

\noindent{\hspace*{\fill}\sphinxincludegraphics{{ddis_add}.png}\hspace*{\fill}}
\phantomsection\label{external_incoming_calls/configure_ddi:bill-inbound}\begin{description}
\item[{Country\index{Country|textbf}}] \leavevmode\phantomsection\label{external_incoming_calls/configure_ddi:term-country}
The country of the new created DDI. Used for E164 standarization.

\item[{DDI\index{DDI|textbf}}] \leavevmode\phantomsection\label{external_incoming_calls/configure_ddi:term-ddi}
The number, without country code.

\item[{Peering contract\index{Peering contract|textbf}}] \leavevmode\phantomsection\label{external_incoming_calls/configure_ddi:term-peering-contract}
The peering contract that provides this number. This relation allow
the platform to apply the required {\hyperref[external_incoming_calls/numeric_transformations:numeric\string-transformations]{\sphinxcrossref{\DUrole{std,std-ref}{Numeric transformations}}}} in
order to determine its standard form.

\item[{External Call Filter\index{External Call Filter|textbf}}] \leavevmode\phantomsection\label{external_incoming_calls/configure_ddi:term-external-call-filter}
Allows configuration based on Calendars and Schedulers as shown in
{\hyperref[pbx_features/external_filters:external\string-filters]{\sphinxcrossref{\DUrole{std,std-ref}{External Call filters}}}}. Leave empty if you don't need to apply any
kind of filter.

\item[{Route\index{Route|textbf}}] \leavevmode\phantomsection\label{external_incoming_calls/configure_ddi:term-route}
A DDI can have different {\hyperref[pbx_features/external_ddis:routing\string-logics]{\sphinxcrossref{\DUrole{std,std-ref}{treatments}}}}. For our
current goul, set route to user and select \emph{Alice}.

\item[{Record calls\index{Record calls|textbf}}] \leavevmode\phantomsection\label{external_incoming_calls/configure_ddi:term-record-calls}
Can be used to record external calls (see {\hyperref[pbx_features/call_recording:call\string-recordings]{\sphinxcrossref{\DUrole{std,std-ref}{Call recording}}}}).

\item[{Tarificate incoming calls\index{Tarificate incoming calls|textbf}}] \leavevmode\phantomsection\label{external_incoming_calls/configure_ddi:term-tarificate-incoming-calls}
This setting requires the external tarification module and allows
tarification on special numbers. This module is not standard so don't
hesitate in {\hyperref[intro/getting_help:getting\string-help]{\sphinxcrossref{\DUrole{std,std-ref}{contact us}}}} if you are interested.

\end{description}


\section{Configure incoming routes}
\label{external_incoming_calls/configure_ddi:configure-incoming-routes}
In the previous section, we have created the DDI and configure it, but \textbf{the
most common procedure} is that the brand operator just create it while the
\textbf{company administator}, using the same saction \textbf{will configure} it choosing
the correct route (user, huntgroup, etc.), its filters with calendars and so on.

\begin{notice}{note}{Note:}
At this point, calling the number of the configured DDI will make the
\emph{Alice} phone ring.
\end{notice}


\chapter{Making external calls}
\label{external_outgoing_calls/index:making-external-calls}\label{external_outgoing_calls/index::doc}
The goal of this section is configuring IvozProvider to make external outgoing
calls, taking previous section configuration as a starting point.

We will follow this steps:


\section{Where do I call?}
\label{external_outgoing_calls/call_types::doc}\label{external_outgoing_calls/call_types:where-do-i-call}
At this point of the configuration, we have to configure IvozProvider to use the
already configured \emph{Contract Peering} to place the external calls we are making.

To achieve this, in first place, we need that the dialed external numbers fall
in an existing \textbf{target pattern}.


\subsection{Target patterns}
\label{external_outgoing_calls/call_types:target-patterns}\label{external_outgoing_calls/call_types:id1}
When a user dials an external phone number, IvozProvider tries to categorize
this call into a one of the target patterns defined in this section:

\noindent{\hspace*{\fill}\sphinxincludegraphics{{target_patterns_section}.png}\hspace*{\fill}}

Usually, it will we useful to have one target pattern for the 254 countries
defined in the \href{https://es.wikipedia.org/wiki/ISO\_3166}{ISO 3166}. That's why IvozProvider automatically
includes all this countries and their prefixes:

\noindent{\hspace*{\fill}\sphinxincludegraphics{{target_patterns_default}.png}\hspace*{\fill}}

Within this list we can find Spain's prefix, that will be the prefix of the test
call we are going to make in this section:

\noindent{\hspace*{\fill}\sphinxincludegraphics{{target_patterns_spain}.png}\hspace*{\fill}}

\begin{notice}{warning}{Warning:}
Brand operator can choose between keeping this target pattern if
finds them useful or deleting them an creating the ones that meet his needs. In
fact, apart from phone prefixes it is also possible to use regular expressions.
e.g. Unique target pattern that contains all possible targets: \textasciicircum{}{[}0-9{]}+\$
\end{notice}

\begin{notice}{danger}{Danger:}
Notice that using regular expressions instead of prefixes can make
a phone number to match more than one target pattern. Use with responsibility.
\end{notice}


\subsection{Target pattern groups}
\label{external_outgoing_calls/call_types:target-pattern-groups}
As we will see in {\hyperref[external_outgoing_calls/call_routing:outgoing\string-routes]{\sphinxcrossref{\DUrole{std,std-ref}{rutas salientes}}}} section, every target
pattern will be linked to a Peering Contract.

That's why it can be useful to group the target patterns in \textbf{target pattern group}
so that we can link a whole group to a Peering Contract more easily.

This is the goal of this section:

\noindent{\hspace*{\fill}\sphinxincludegraphics{{target_patterngroups_section}.png}\hspace*{\fill}}

By default we can see the 254 countries grouped in the continents defined in
\href{https://es.wikipedia.org/wiki/ISO\_3166}{ISO 3166}:

\noindent{\hspace*{\fill}\sphinxincludegraphics{{target_patterngroups_default}.png}\hspace*{\fill}}

\begin{notice}{important}{Important:}
\textbf{To sum up}, when a user dials an external number, IvozProvider
looks up a matching target pattern to decide which PeeringContract must be used
to place this call.
\end{notice}

To achive our goal of making an external call to a spanish number, we didn't have
to modify the initial contents of this two sections :)
\phantomsection\label{external_outgoing_calls/call_routing:outgoing-routes}
We already have our test call categorized as a call within the \textbf{Target pattern}
`Spain'. In addition, we also have a \textbf{Target pattern group} including `Spain',
called `Europe'.

Now we have to tell IvozProvider that calls to `Spain' or `Europe' should be
established through our \textbf{Contract Peering}.


\section{Outgoing Routing}
\label{external_outgoing_calls/call_routing::doc}\label{external_outgoing_calls/call_routing:outgoing-routing}
To make this assignment, we use the section \textbf{Outgoing routing}:

\noindent{\hspace*{\fill}\sphinxincludegraphics{{outgoing_routes_section}.png}\hspace*{\fill}}

If we choose routing `Spain' calls only through our \emph{Peering contract}, we will
make this configuration:

\noindent{\hspace*{\fill}\sphinxincludegraphics{{outgoing_routes_by_pattern}.png}\hspace*{\fill}}

On the other hand, if we are more generous and we decide to place calls to all
european countries, we would make this configuration:

\noindent{\hspace*{\fill}\sphinxincludegraphics{{outgoing_routes_by_patterngroup}.png}\hspace*{\fill}}
\phantomsection\label{external_outgoing_calls/call_routing:routes-metrics}
Two parameters deserve an explanation in this section:
\begin{description}
\item[{Priority\index{Priority|textbf}}] \leavevmode\phantomsection\label{external_outgoing_calls/call_routing:term-priority}
If a call matches several routes, it will be placed using the outgoing
route with lower priority, as long as it is available.

\item[{Metric\index{Metric|textbf}}] \leavevmode\phantomsection\label{external_outgoing_calls/call_routing:term-metric}
If a call matches several routes with equal priority, metric will determine
the proportion of calls that will use one route or another.

\end{description}

\begin{notice}{note}{Note:}
This are the key parameters to achieve two interesting features:
\textbf{load-balancing} and \textbf{failover-routes}.
\end{notice}


\subsection{Load balancing}
\label{external_outgoing_calls/call_routing:load-balancing}
\emph{Load-balancing} lets us distribute calls matching the same pattern using
several valid outgoing routes.
\paragraph{Example 1}
\begin{itemize}
\item {} 
Route A: priority 1, metric 1

\item {} 
Route B: priority 1, metric 1

\end{itemize}

Call matching these routes will use route A for \%50 of the calls and route B for
\%50 of the calls.
\paragraph{Example 2}
\begin{itemize}
\item {} 
Route A: priority 1, metric 1

\item {} 
Route B: priority 1, metric 2

\end{itemize}

Call matching these routes will use route A for \%33 of the calls and route B for
\%66 of the calls.


\subsection{Failover routes}
\label{external_outgoing_calls/call_routing:failover-routes}
Failover route lets us use another route whenever the main route fails.
\paragraph{Example}
\begin{itemize}
\item {} 
Route A: priority 1, metric 1

\item {} 
Route B: priority 2, metric 1

\end{itemize}

All calls matching these routes will try to use route A. In case the call fails,
the call will be placed using route B.

\begin{notice}{tip}{Tip:}
Although given examples use two routes, more routes can be chained and
failover and load-balancing estrategies can be combined.
\end{notice}


\section{Outgoing DDI configuration}
\label{external_outgoing_calls/external_ddi:external-ddi}\label{external_outgoing_calls/external_ddi::doc}\label{external_outgoing_calls/external_ddi:outgoing-ddi-configuration}
Before placing our first outgoing call, it would be desirable to choose the
number that the callee will see when the phone rings, so that he can return the
call easily.

To achieve this goal, we have to configure our DDI as \emph{Alice's} \textbf{outbound DDI},
because she will be the chosen one to place our first outgoing call:

\noindent\sphinxincludegraphics{{ddi_out}.png}

We can set this up editing \emph{Alice} in \textbf{Company Configuration} \textgreater{} \textbf{Users}. If
this change is made by brand operator or global operator, he must {\hyperref[internal_calls/brand_portal:emulate\string-company]{\sphinxcrossref{\DUrole{std,std-ref}{emulate
the corresponding company}}}} previously.

\begin{notice}{warning}{Warning:}
Calls from users without an outgoing DDI will be rejected by IvozProvider.
\end{notice}

At this point, we are looking forward to make our first outgoing call with our
new IvozProvider, we may have even tried to call with current configuration but...


\section{No pricing plan, no call}
\label{external_outgoing_calls/noplan_nocall:noplan-nocall}\label{external_outgoing_calls/noplan_nocall::doc}\label{external_outgoing_calls/noplan_nocall:no-pricing-plan-no-call}
Just the way we warned {\hyperref[operation_roles/index:brand\string-responsibilities]{\sphinxcrossref{\DUrole{std,std-ref}{when we described the duties of the brand operator}}}}, the brand operator is \textbf{responsible for making all the
needed setup so that IvozProvider is able to bill all external calls}.

\begin{notice}{note}{Note:}
\textbf{Billing a call} is the action of \textbf{assigning price} to a call that implies
cost.
\end{notice}

\textbf{IvozProvider checks live that a call can be billed when it is established} to avoid
placing calls that imply cost but won't be billed because Brand Operator, due to
a mistake, hasn't assigned a price.

\begin{notice}{error}{Error:}
If a call can't be billed, IvozProvider won't allow its establishment.
\end{notice}


\subsection{Creating a pricing pattern}
\label{external_outgoing_calls/noplan_nocall:price-pattern}\label{external_outgoing_calls/noplan_nocall:creating-a-pricing-pattern}
Just the way {\hyperref[external_outgoing_calls/call_types:target\string-patterns]{\sphinxcrossref{\DUrole{std,std-ref}{target patterns}}}} exist, \textbf{pricing patterns}
exist and are configured in this section:

\noindent\sphinxincludegraphics{{pricing_pattern_section}.png}

\begin{notice}{important}{Important:}
A call is considered billable if there is a pricing pattern that
matches this call.
\end{notice}

Pricing patterns section is empty by default, as opposed to target patterns section,
that has all the 254 countries of the world. The reason is that pricing pattern
will usually imply lots of pattern per country (GSM networks, especial numbers,
mobile numbers, fixed lines, etc.).

We will create the pricing plan `Spain' for our outgoing call:

\noindent\sphinxincludegraphics{{pricing_pattern_add}.png}


\subsection{Creating a pricing plan}
\label{external_outgoing_calls/noplan_nocall:price-plan}\label{external_outgoing_calls/noplan_nocall:creating-a-pricing-plan}
A \textbf{Pricing plan} determines the price of a type of call (of a pricing pattern)
and is configured in this section:

\noindent\sphinxincludegraphics{{pricing_plans_section}.png}

We create a \textbf{pricing plan} for our goal:

\noindent\sphinxincludegraphics{{pricing_plans_add1}.png}

And we add the \textbf{pricing patter} we have just created:

\noindent\sphinxincludegraphics{{pricing_plans_add_price1}.png}

\noindent\sphinxincludegraphics{{pricing_plans_add_price2}.png}

\begin{notice}{note}{Note:}
Floating number must use the ''.'' as decimal separator (e.g. 0.02)
\end{notice}
\paragraph{Finding a pricing plan for a specific destination}

To check the configuration so far we can \textbf{find a pricing plan} for a call
pressing:

\noindent\sphinxincludegraphics{{pricing_plans_find_plan}.png}

We introduce the destination number in {\hyperref[external_incoming_calls/numeric_transformations:e164]{\sphinxcrossref{\DUrole{std,std-ref}{E.164 format}}}}:

\noindent\sphinxincludegraphics{{pricing_plans_find_plan2}.png}

And we can check that it matches the \textbf{precing plan} we have just created:

\noindent\sphinxincludegraphics{{pricing_plans_find_plan3}.png}


\subsection{Assigning a pricing plan to a company}
\label{external_outgoing_calls/noplan_nocall:assigning-a-pricing-plan-to-a-company}\label{external_outgoing_calls/noplan_nocall:pricing-plan-to-company}
A specific \textbf{pricing plan} can be linked to `n' companies and the \emph{brand
operator} is responsible for this task.

In the section \textbf{Brand configuration} \textgreater{} \textbf{Companies} we select the \emph{demo}
company:

\noindent\sphinxincludegraphics{{pricing_plans_relate_company}.png}

The \textbf{Pricing plan} and \textbf{Companies} relationship is set for a determined
period of time, that's why we have to select \emph{Start time} and \emph{End time}:

\noindent\sphinxincludegraphics{{pricing_plans_relate_company2}.png}

The \textbf{metric} of the link lets you assign more than one \emph{pricing plan} for a
company, even though some destinations are included in more than one of those
pricing plans.

\begin{notice}{attention}{Attention:}
If a given call can be billed with more than one active pricing
plan, it will be billed using the pricing plan with lower metric.
\end{notice}

\begin{notice}{tip}{Tip:}
This allows having a general \emph{Pricing plan} and concrete the price of
a specific destination in another \emph{pricing plan} with lower metric (free cell
phone calls, for example).
\end{notice}
\paragraph{Simulating a call of a specific company}

We can simulate a call for a given company and check the price it will imply.
This way, we can be sure that the configuration is ok and that calls to that
destination will be billed using a specific \emph{Pricing plan}:

\noindent\sphinxincludegraphics{{pricing_plans_simulate_companycall}.png}

We introduce the destination number in {\hyperref[external_incoming_calls/numeric_transformations:e164]{\sphinxcrossref{\DUrole{std,std-ref}{E.164 format}}}}:

\noindent{\hspace*{\fill}\sphinxincludegraphics{{pricing_plans_simulate_companycall2}.png}\hspace*{\fill}}

And we confirm that it will be billed with the pricing plan that we have just
created and linked:

\noindent\sphinxincludegraphics{{pricing_plans_simulate_companycall3}.png}

\begin{notice}{note}{Note:}
At this point, \emph{Alice} should be able to make outgoing calls to
spanish destinations and this calls should be billed accordingly.
\end{notice}


\chapter{PBX Features}
\label{pbx_features/index:pbx-features}\label{pbx_features/index::doc}
In the previous block we have successfully make internal and external calls
with just a minimum configuration, based on the base data left by the installer
itself and ignoring a lot of the sections and features.

The goal of this section will be describe each one of those features that
are also included in IvozProvider.


\section{Extensions}
\label{pbx_features/extensions:extensions}\label{pbx_features/extensions::doc}
The base configuration includes 2 extensions (101 and 102) that route directly
to \emph{Alice} and \emph{Bob}, so we had almost nothing to do with the section
\textbf{Company configuration} \textgreater{} \textbf{Extensions}.

\begin{notice}{note}{Note:}
\textbf{An extensions is}, by definition, \textbf{an internal number with an
assigned logic}.
\end{notice}
\paragraph{Create a new extension}

The new extension window looks like this:

\noindent\sphinxincludegraphics{{extension_add}.png}
\begin{description}
\item[{Number\index{Number|textbf}}] \leavevmode\phantomsection\label{pbx_features/extensions:term-number}
The number that must be dialed by the internal user that will trigger
the configured logic. It must have a minimum length of 2 and must be
a number.

\item[{Route\index{Route|textbf}}] \leavevmode\phantomsection\label{pbx_features/extensions:term-route}
This select will allow us to choose the logic that will use this
extension when is dialed from an internal user. Depending on the selected
route, and additional select or input will be shown to select the
hungroup, conference room, user, etc.

\end{description}

\begin{notice}{warning}{Warning:}
If an extension has a number that conflicts with an external
number, this external number will be masked and, in practice, will be
unavailable for the whole company.
\end{notice}


\section{Terminals configuration}
\label{pbx_features/terminals:terminals}\label{pbx_features/terminals::doc}\label{pbx_features/terminals:terminals-configuration}
The section \textbf{Company configuration} \textgreater{} \textbf{Terminals} allows creating new
SIP credentials that can be used by multiple SIP devices to place and receive
calls from IvozProvider.

The best way to understand this section is creating a new item and see the
fields that must be filled.

\noindent\sphinxincludegraphics{{terminals_add}.png}
\begin{description}
\item[{Name\index{Name|textbf}}] \leavevmode\phantomsection\label{pbx_features/terminals:term-name}
Username that will use the terminal during the SIP authentication phase
with IvozProvider.

\item[{Password\index{Password|textbf}}] \leavevmode\phantomsection\label{pbx_features/terminals:term-password}
Password that will use the terminal to answer the SIP authentication
challenge. You can use the automatic password generator to fullfill the
secure password requirements.

\item[{Allowed/Disallowed codecs\index{Allowed/Disallowed codecs|textbf}}] \leavevmode\phantomsection\label{pbx_features/terminals:term-allowed-disallowed-codecs}
Determines what audio and video codecs will be used with the terminal.

\item[{CallerID update method\index{CallerID update method|textbf}}] \leavevmode\phantomsection\label{pbx_features/terminals:term-callerid-update-method}
Choose the SIP method the terminal prefers to received the session
update information: INVITE or UPDATE. The help hint can be used as
guide to configure different terminal manufacturers. Use \emph{INVITE} in
case of doubt.

\item[{Terminal model\index{Terminal model|textbf}}] \leavevmode\phantomsection\label{pbx_features/terminals:term-terminal-model}
Determines the provisioning type that will receive this terminal.
The section {\hyperref[provisioning/index:provisioning]{\sphinxcrossref{\DUrole{std,std-ref}{terminal provisioning}}}} will explain
in depth the different models for automatic provision. If your device
does not require provisioning, just select \emph{Generic}.

\item[{MAC\index{MAC|textbf}}] \leavevmode\phantomsection\label{pbx_features/terminals:term-mac}
Optional field that is only required if you plan to use IvozProvider
{\hyperref[provisioning/index:provisioning]{\sphinxcrossref{\DUrole{std,std-ref}{terminal provisioning}}}}. This is the \href{https://wikipedia.org/wiki/MAC\_Address}{phisical
address} of the network
adapter of the SIP device.

\end{description}

\begin{notice}{note}{Note:}
For \textbf{most of devices} that doesn't require provisioning just
filling \textbf{username} and \textbf{password} will be enough.
\end{notice}

\begin{notice}{hint}{Hint:}
Once the terminal has been created, most devices will only
require the name, password and {\hyperref[internal_calls/brand_portal:domain\string-per\string-company]{\sphinxcrossref{\DUrole{std,std-ref}{Company SIP domain}}}}
in order to place calls.
\end{notice}


\section{Call ACL Control}
\label{pbx_features/call_permissions:call-permissions}\label{pbx_features/call_permissions::doc}\label{pbx_features/call_permissions:call-acl-control}
The \textbf{Call ACLs} determines what users can call to external numbers.

\begin{notice}{attention}{Attention:}
The internal extensions are allowed to all users, the \textbf{Call
ACLs only apply to external numbers}
\end{notice}

The \textbf{Call ACL} setup has two different parts:
\begin{itemize}
\item {} 
Classify the call in different types based on \textbf{regular expressions}:
\begin{itemize}
\item {} 
Brand level: \textbf{Brand Configuration} \textgreater{} \textbf{Generic call ACL patterns}

\item {} 
Company level: \textbf{Company Configuration} \textgreater{} \textbf{Call ACL patterns}

\end{itemize}

\item {} 
Choose policies for groups of patterns: \textbf{Company Configuration} \textgreater{} \textbf{Call
ACLs}

\end{itemize}


\subsection{Call ACL patterns}
\label{pbx_features/call_permissions:call-acl-patterns}
The destination number is matched against the \textbf{Company ACL patterns} to determine
the destination type.

\begin{notice}{note}{Note:}
When a Brand operator creates a new company, all of the \textbf{Generic
ACL patterns} defined in the \textbf{Brand configuration} are copied to the
\textbf{Company configuration} \textgreater{} \textbf{Call ACLs}. This way, the brand operator can
define the most common patterns to speed up the company configuration.
\end{notice}

The patterns creation process is quite simple:

\noindent\sphinxincludegraphics{{permissions_patterns}.png}

This new ACL pattern includes the calls starting with the spanish country
prefix followed by 6 or 7 and 8 more digits between 0 and 9. This is the E.164
format for the spanish mobile numbers.

Following this spanish example, other formats will be:
\begin{itemize}
\item {} \begin{description}
\item[{Spanish Landline (including special numbers prefix: 902, etc.): \textasciicircum{}34{[}89{]}{[}0-9{]}\{8\}\$}] \leavevmode\begin{itemize}
\item {} 
34 (Spain country prefix), 8 or 9 followed by 8 digits

\end{itemize}

\end{description}

\item {} \begin{description}
\item[{Spanish Landline (excluding special number prefix: 902, etc.): \textasciicircum{}34{[}89{]}{[}1-9{]}{[}0-9{]}\{7\}\$}] \leavevmode\begin{itemize}
\item {} 
34 (Spain country prefix), 8 or 9, followed by one digit between 1 and 9, followed by 7 digits.

\end{itemize}

\end{description}

\item {} \begin{description}
\item[{United Kingdom Landline: \textasciicircum{}44{[}0-9{]}+\$}] \leavevmode\begin{itemize}
\item {} 
44 (UK country prefix), followed by more digits

\end{itemize}

\end{description}

\end{itemize}
\paragraph{External numbers format}

\begin{notice}{attention}{Attention:}
\textbf{Regular expressions of Call ACL patterns must be in E.164 format}.
\end{notice}

There are two main reasons for this decision:
\begin{itemize}
\item {} 
The same pattern will apply to all the users of the company,  no matter
what country the user is.

\item {} 
Brand-level ACL patterns will be inherited by all new companies. The only way
this inheritance could be useful was using an standard format, valid for all the
companies countries.

\end{itemize}

Besides, this way it is really easy to avoid (or allow) call to a country.


\subsection{Call ACL}
\label{pbx_features/call_permissions:call-acl}
The \textbf{Call ACL} configuration is easier to explain with an example:

Imagine the following \textbf{CALL ACL patterns}:

\noindent\sphinxincludegraphics{{permissions_patterns_list}.png}

We could create a \textbf{Call ACL} that only allow calling to this destinations:

\noindent\sphinxincludegraphics{{permissions_add}.png}

\begin{notice}{note}{Note:}
The default action determines what to do with the call when the
destination number \textbf{does not match any ACL patterns}.
\end{notice}

After creating the \textbf{Call ACL} we can edit it to add the required rules:

\noindent\sphinxincludegraphics{{permissions_add2}.png}
\begin{description}
\item[{The \textbf{metric} determines the evaulation order of the rules and the action that}] \leavevmode
that will be applied if it \emph{matches} the pattern (allow/deny).

\end{description}

\noindent\sphinxincludegraphics{{permissions_add3}.png}

Once we have added our two spanish \textbf{Call ACL patterns}, our \textbf{Call ACL} will
look like this:

\noindent\sphinxincludegraphics{{permissions_add4}.png}

We only have to assign this ACL to the users in the section \textbf{Company
configuration} \textgreater{} \textbf{Users}:

\noindent\sphinxincludegraphics{{permissions_add5}.png}

From this moment on, Alice will only be allowed to call internal extensions
(they are always allowed) and spanish numbers.


\section{Users configuration}
\label{pbx_features/users::doc}\label{pbx_features/users:users}\label{pbx_features/users:users-configuration}
The installation process creates \emph{Alice} and \emph{Bob} users, allowing us
to test internals calls between them without too much effort.

We skipped most of the settings in \textbf{Users} configuration that we will described
in this section.


\subsection{Personal data}
\label{pbx_features/users:personal-data}
\noindent\sphinxincludegraphics{{users_edit_personal_data}.png}
\begin{description}
\item[{Name\index{Name|textbf}}] \leavevmode\phantomsection\label{pbx_features/users:term-name}
Used to identify this user in most of the screens. This is also the
name that will be displayed in internal calls made from this user.

\item[{Lastname\index{Lastname|textbf}}] \leavevmode\phantomsection\label{pbx_features/users:term-lastname}
Most of the times this is used to complete the previous field.

\item[{Email\index{Email|textbf}}] \leavevmode\phantomsection\label{pbx_features/users:term-email}
Email used to send the user's received voicemails. This is also used to
identify the user in their portal.

\item[{Country code / Area code\index{Country code / Area code|textbf}}] \leavevmode\phantomsection\label{pbx_features/users:term-country-code-area-code}
Defines the way the user calls and the way the numbers are presented to
this user.

\item[{Language\index{Language|textbf}}] \leavevmode\phantomsection\label{pbx_features/users:term-language}
When a locution is played to this user, this language is used.

\item[{Timezone\index{Timezone|textbf}}] \leavevmode\phantomsection\label{pbx_features/users:term-timezone}
User portal call list times will use this timezone.

\end{description}


\subsection{Login Info}
\label{pbx_features/users:login-info}
\noindent\sphinxincludegraphics{{users_edit_login}.png}
\begin{description}
\item[{Active\index{Active|textbf}}] \leavevmode\phantomsection\label{pbx_features/users:term-active}
Allows administrators to grant or disable user's acces to the
{\hyperref[userportal/index:userportal]{\sphinxcrossref{\DUrole{std,std-ref}{user's portal}}}}.

\item[{Password\index{Password|textbf}}] \leavevmode\phantomsection\label{pbx_features/users:term-password}
Password used to access the {\hyperref[userportal/index:userportal]{\sphinxcrossref{\DUrole{std,std-ref}{user's portal}}}}.

\item[{QR Code\index{QR Code|textbf}}] \leavevmode\phantomsection\label{pbx_features/users:term-qr-code}
If enabled, a QR code for Grandstream Wave softphone configuration
will be shown.

\end{description}


\subsection{Basic Configuration}
\label{pbx_features/users:basic-configuration}
\noindent\sphinxincludegraphics{{users_edit_basic_config}.png}
\begin{description}
\item[{Terminal\index{Terminal|textbf}}] \leavevmode\phantomsection\label{pbx_features/users:term-terminal}
The available terminals created in {\hyperref[pbx_features/terminals:terminals]{\sphinxcrossref{\DUrole{std,std-ref}{Terminals configuration}}}} are listed here
for assignment.

\item[{Screen Extension\index{Screen Extension|textbf}}] \leavevmode\phantomsection\label{pbx_features/users:term-screen-extension}
One of the available {\hyperref[pbx_features/extensions:extensions]{\sphinxcrossref{\DUrole{std,std-ref}{Extensions}}}} that this user will display when
placing internal calls. While multiple extensions can be routed to the
user, only one of them will be presented when the user calls.

\item[{Outgoing DDI\index{Outgoing DDI|textbf}}] \leavevmode\phantomsection\label{pbx_features/users:term-outgoing-ddi}
As described in {\hyperref[external_outgoing_calls/external_ddi:external\string-ddi]{\sphinxcrossref{\DUrole{std,std-ref}{Outgoing DDI configuration}}}}, determines the number that will
present when placing external outgoing calls.

\item[{Outgoing DDI Rules\index{Outgoing DDI Rules|textbf}}] \leavevmode\phantomsection\label{pbx_features/users:term-outgoing-ddi-rules}
Manages exceptions to previous setting. Read {\hyperref[pbx_features/outgoingddi_rules:outgoingddi\string-rules]{\sphinxcrossref{\DUrole{std,std-ref}{Outgoing DDI Rules}}}}
for further reference.

\item[{Call ACL\index{Call ACL|textbf}}] \leavevmode\phantomsection\label{pbx_features/users:term-call-acl}
One of the created {\hyperref[pbx_features/call_permissions:call\string-permissions]{\sphinxcrossref{\DUrole{std,std-ref}{Call ACL}}}} groups, described
it the previous sections.

\item[{Do not disturb\index{Do not disturb|textbf}}] \leavevmode\phantomsection\label{pbx_features/users:term-do-not-disturb}
When this setting is enabled, the user won't receive any call but can
still place calls.

\item[{Max Calls\index{Max Calls|textbf}}] \leavevmode\phantomsection\label{pbx_features/users:term-max-calls}
Limits the number of received calls if the user is handling
simultaneously (inbound and outbound) more than the number set.
Set 0 for unlimited calls.

\item[{Calls from non-granted IPs:\index{Calls from non-granted IPs:|textbf}}] \leavevmode\phantomsection\label{pbx_features/users:term-calls-from-non-granted-ips}
Enable calling from non-granted IP addresses for this user.
It limits the number of outgoing calls to avoid toll-fraud.
`None' value makes outgoing calls unlimited as long as company IP
policy is fulfilled. Read {\hyperref[security/index:roadwarrior\string-users]{\sphinxcrossref{\DUrole{std,std-ref}{Roadwarrior users}}}} for further reference.

\end{description}


\subsection{Voicemail}
\label{pbx_features/users:voicemail}
\noindent\sphinxincludegraphics{{users_voicemail}.png}
\begin{description}
\item[{VoiceMail enabled\index{VoiceMail enabled|textbf}}] \leavevmode\phantomsection\label{pbx_features/users:term-voicemail-enabled}
Enables or disables the \textbf{existance} of a users voicemail.
This only makes the voicemail available to be routed as described in the
section {\hyperref[pbx_features/users:fwd\string-to\string-vm]{\sphinxcrossref{\DUrole{std,std-ref}{forward to voicemail}}}}.

\item[{Voicemail Locution\index{Voicemail Locution|textbf}}] \leavevmode\phantomsection\label{pbx_features/users:term-voicemail-locution}
If set, this locution is played as voicemail welcome message when a voicemail
for this user is going to be recorded. This only applies for call forwardings
to voicemail described in the section {\hyperref[pbx_features/users:fwd\string-to\string-vm]{\sphinxcrossref{\DUrole{std,std-ref}{forward to voicemail}}}}.

\item[{Email notification\index{Email notification|textbf}}] \leavevmode\phantomsection\label{pbx_features/users:term-email-notification}
Send an email to the configured user address when a new voicemail is
received.

\item[{Attach sounds:\index{Attach sounds:|textbf}}] \leavevmode\phantomsection\label{pbx_features/users:term-attach-sounds}
Attach the audio message to the sent email.

\end{description}

\begin{notice}{note}{Note:}
If voicemail locution is not assigned, default locution will be used as long as
the user has not recorded a custom message through the voicemail menu (calling to
voicemail service code).
\end{notice}


\subsection{Boss-Assistant}
\label{pbx_features/users:boss-assistant}
\noindent{\hspace*{\fill}\sphinxincludegraphics{{users_edit_boss}.png}\hspace*{\fill}}

This feature will turn the user into a boss that can only be directly call by:
\begin{itemize}
\item {} 
The selected assistant.

\item {} 
Any origin that matches the white list.

\end{itemize}

The rest of the calls to \emph{a bos} will be redirected to the assistant.
\begin{description}
\item[{Is boss\index{Is boss|textbf}}] \leavevmode\phantomsection\label{pbx_features/users:term-is-boss}
Determines if this user is a boss.

\item[{Assistant\index{Assistant|textbf}}] \leavevmode\phantomsection\label{pbx_features/users:term-assistant}
Who will receive the redirected calls of this boss.

\item[{Whitelist\index{Whitelist|textbf}}] \leavevmode\phantomsection\label{pbx_features/users:term-whitelist}
{\hyperref[pbx_features/match_lists:match\string-lists]{\sphinxcrossref{\DUrole{std,std-ref}{Match Lists}}}} with origins that are allowed to call directly to
the boss.

\end{description}

With the setup in the image, every call to \emph{Alice} will be redirected to \emph{Bob},
except the ones placed by \emph{Bob} itself and those coming from any origin that matches
\emph{Alice's friends} matchlist.


\subsection{Group Configuration}
\label{pbx_features/users:group-configuration}
\noindent\sphinxincludegraphics{{users_edit_groups}.png}

As described in the sections {\hyperref[pbx_features/huntgroups:huntgroups]{\sphinxcrossref{\DUrole{std,std-ref}{Hunt groups}}}} and {\hyperref[pbx_features/call_captures:capture\string-groups]{\sphinxcrossref{\DUrole{std,std-ref}{Call pickup}}}}, the
user can be part of one or more huntgroups and pickup groups.

Those groups can be configured from the sections {\hyperref[pbx_features/huntgroups:huntgroups]{\sphinxcrossref{\DUrole{std,std-ref}{Hunt groups}}}} and
{\hyperref[pbx_features/call_captures:capture\string-groups]{\sphinxcrossref{\DUrole{std,std-ref}{Call pickup}}}} or the user's screen if the groups already exists.

You can also configure the user's \textbf{hunt groups} from the icon in each user
line of the users list.

\noindent{\hspace*{\fill}\sphinxincludegraphics{{users_huntgroups}.png}\hspace*{\fill}}


\subsection{User Call Forward}
\label{pbx_features/users:user-call-forward}
The user's call forward can be configured in the following button:

\noindent{\hspace*{\fill}\sphinxincludegraphics{{users_call_fwd}.png}\hspace*{\fill}}
\phantomsection\label{pbx_features/users:fwd-to-vm}
These are the fields and available values:
\begin{description}
\item[{Enabled\index{Enabled|textbf}}] \leavevmode\phantomsection\label{pbx_features/users:term-enabled}
Determines if the forward must be applied or not. This way, you can have
most used call forward configured and toggle if they apply or not.

\item[{Call Type\index{Call Type|textbf}}] \leavevmode\phantomsection\label{pbx_features/users:term-call-type}
Determines if the forward must be applied to external, internal or any
type of call.

\item[{Forward type\index{Forward type|textbf}}] \leavevmode\phantomsection\label{pbx_features/users:term-forward-type}
When this forward must be applied:
\begin{itemize}
\item {} 
Inconditional: always

\item {} 
No answer: when the call is not answered in X seconds

\item {} 
Busy: When the user is talking to someone (and call waiting is
disabled), when \emph{Do not disturb} is enabled or when the user
rejects an incoming call.

\item {} 
Not registered: when the user SIP terminal is not registered
against IvozProvider.

\end{itemize}

\item[{Target type\index{Target type|textbf}}] \leavevmode\phantomsection\label{pbx_features/users:term-target-type}
What route will use the forwarded call.
\begin{itemize}
\item {} 
VoiceMail

\item {} 
Number (external)

\item {} 
Extension (internal)

\end{itemize}

\end{description}

\begin{notice}{hint}{Hint:}
If we want to forward to other process, we can create an extension
routed to that object and use the target type \emph{Extension}.
\end{notice}


\section{Friends}
\label{pbx_features/friends:friends}\label{pbx_features/friends::doc}
\textbf{Friends} section in the \textbf{Company configuration} allows interconnection of
IvozProvider with other SIP PBX systems through a SIP \emph{trunk}. The most typical
use case is when a company have multiple PBX systems that want to integrate in
a single flow.

\begin{notice}{warning}{Warning:}
It's important to understand the difference between \textbf{Contract peering}
defined by the \textbf{brand operator} to connect with the public network
and \textbf{Friends}, defined by \textbf{company administrators} to connect the
system with other PBXs.
\end{notice}


\subsection{What does this allow?}
\label{pbx_features/friends:what-does-this-allow}
This sections allows not just communication between users at boths ends of the
SIP \emph{trunk}, but also:
\begin{itemize}
\item {} 
Users ``from the other side'' can call to the public network just like native
Ivozprovider {\hyperref[pbx_features/users:users]{\sphinxcrossref{\DUrole{std,std-ref}{Users}}}}.

\item {} 
Public network calls can be routed to the other SIP \emph{trunk} end.

\end{itemize}


\subsection{Types of friends}
\label{pbx_features/friends:types-of-friends}
There are 2 main types of SIP PBX that can be integrate with IvozProvider:
\begin{itemize}
\item {} 
\textbf{Direct connection PBX}: IvozProvider must be able to talk SIP directly with
this kind of friends by just redirecting the traffic to the proper port of
the public IP address of the PBX.

\item {} 
\textbf{PBX behind NAT}: Not directly accesible. This kind of PBX must register at
IvozProvider (just like all the {\hyperref[pbx_features/terminals:terminals]{\sphinxcrossref{\DUrole{std,std-ref}{Terminals}}}} do).

\end{itemize}


\subsection{What kind of calls can be routed through a \emph{friend}?}
\label{pbx_features/friends:what-kind-of-calls-can-be-routed-through-a-friend}
IvozProvider must know what calls must be routed to the different defined \emph{friends}.
For that, \textbf{company administrator} will configure regular expressions that
describe the numbers that \emph{can be reached} through the \textbf{friend}.

\begin{notice}{note}{Note:}
Internal {\hyperref[pbx_features/extensions:extensions]{\sphinxcrossref{\DUrole{std,std-ref}{extensions}}}} have priority over any expression
defined in the \emph{friends}.
\end{notice}

To sum up, IvozProvider will route a call received by a {\hyperref[pbx_features/users:users]{\sphinxcrossref{\DUrole{std,std-ref}{user}}}} or
a {\hyperref[pbx_features/friends:friends]{\sphinxcrossref{\DUrole{std,std-ref}{friend}}}} following this logic:
\begin{enumerate}
\item {} 
Destination matches an existing IvozProvider extension?

\item {} 
If not: Destination matches any \emph{friend} regular expression?

\item {} 
If not: This is an external call.

\end{enumerate}


\subsection{Configuration}
\label{pbx_features/friends:configuration}
The \textbf{Friend} configuration is a merge between a \textbf{User} and a \textbf{Terminal}

\begin{notice}{hint}{Hint:}
\textbf{Friends} are so similar to \textbf{Users} that both talk SIP with the
{\hyperref[internal_calls/god_portal:proxyusers]{\sphinxcrossref{\DUrole{std,std-ref}{Users SIP Proxy}}}}.
\end{notice}

This are the configurable settings of \emph{friends}:
\begin{description}
\item[{Name\index{Name|textbf}}] \leavevmode\phantomsection\label{pbx_features/friends:term-name}
Name of the \textbf{friend}, like in \textbf{Terminals}. This will also be used
in SIP messanges (sent \textbf{From User}).

\item[{Description\index{Description|textbf}}] \leavevmode\phantomsection\label{pbx_features/friends:term-description}
Optional. Extra information for this \textbf{friend}.

\item[{Priority\index{Priority|textbf}}] \leavevmode\phantomsection\label{pbx_features/friends:term-priority}
Used to solve conflicts while routing calls through \textbf{friends}.
If a call destination \textbf{matches} more than one friend regular expresion
the call will be routed through the friend with \textbf{less priority value}.

\item[{Password\index{Password|textbf}}] \leavevmode\phantomsection\label{pbx_features/friends:term-password}
When the \emph{friend} send requests, IvozProvider will authenticate it using
this password. Like in terminals \textbf{using password IS A MUST}.

\item[{Direct connection\index{Direct connection|textbf}}] \leavevmode\phantomsection\label{pbx_features/friends:term-direct-connection}
If you choose `Yes' here, you'll have to fill the protocol, address and
port where this \emph{friend} can be contacted.

\item[{Call ACL\index{Call ACL|textbf}}] \leavevmode\phantomsection\label{pbx_features/friends:term-call-acl}
Similar to {\hyperref[pbx_features/users:users]{\sphinxcrossref{\DUrole{std,std-ref}{internal users}}}}, friends can place internal
company calls without restriction (including Extension or other Friends).
When calling to external numbers, this ACL will be checked if set.

\item[{Fallback Outgoing DDI\index{Fallback Outgoing DDI|textbf}}] \leavevmode\phantomsection\label{pbx_features/friends:term-fallback-outgoing-ddi}
External calls from this \emph{friend} will be presented with this DDI, \textbf{unless
the source presented by friend is a DDI that exists in DDIs section}.

\item[{Country and Area code\index{Country and Area code|textbf}}] \leavevmode\phantomsection\label{pbx_features/friends:term-country-and-area-code}
Used for number transformation from and to this friend.

\item[{Allowed codecs\index{Allowed codecs|textbf}}] \leavevmode\phantomsection\label{pbx_features/friends:term-allowed-codecs}
Like a terminal, \emph{friends} will talk the selected codec.

\item[{From domain\index{From domain|textbf}}] \leavevmode\phantomsection\label{pbx_features/friends:term-from-domain}
Request from IvozProvider to this friend will include this domain in
the From header.

\end{description}

\begin{notice}{note}{Note:}
Calls to \emph{friends} are considered internal. That means that ACLs won't
be checked when calling a friend, no matter if the origin of the call
is a user or another friend.
\end{notice}


\subsection{Asterisk as a friend}
\label{pbx_features/friends:asterisk-as-a-friend}
At the other end of a friend can be any kind of SIP entity. This section takes
as example an Asterisk PBX system using SIP channel driver that wants to connect
to IvozProvider.


\subsubsection{register}
\label{pbx_features/friends:register}
If the system can not be directly access, Asterisk will have to register in the
platform (like a terminal will do).

Configuration will be something like this:

\begin{Verbatim}[commandchars=\\\{\}]
register =\PYGZgt{} friend\PYGZhy{}name:friend\PYGZhy{}password@ivozprovider\PYGZhy{}company.sip\PYGZhy{}domain.com
\end{Verbatim}


\subsubsection{peer}
\label{pbx_features/friends:peer}
\begin{Verbatim}[commandchars=\\\{\}]
[nombre\PYGZhy{}friend]
type=peer
host=ivozprovider\PYGZhy{}company.sip\PYGZhy{}domain.com
context=XXXXXX
disallow=all
allow=alaw
defaultuser=friend\PYGZhy{}name
secret=friend\PYGZhy{}password
fromdomain=ivozprovider\PYGZhy{}company.sip\PYGZhy{}domain.com
insecure=port,invite
\end{Verbatim}

\begin{notice}{warning}{Warning:}
\emph{Friends}, like terminals, MUST NOT challenge IvozProvider. That's
why the \emph{insecure} setting is used here.
\end{notice}


\subsection{Summary}
\label{pbx_features/friends:summary}
The key point is understanding that a \emph{friend} has a direct relation with the
extension-user-terminal trio:
\begin{itemize}
\item {} 
Can place calls to all internal extensions and other friends.

\item {} 
Can place external calls that its ACL allows

\item {} 
Display their configured outgoing DDI when calling to external entities

\item {} 
Never challenge IvozProvider requests (don't request authentication on received requests)

\item {} 
Answers IvozProvider authentication challenges (All request from them to
IvozProvider must be autheticated for security reasons)

\item {} 
Only connects with \emph{Users SIP Proxy}, like terminals. In fact, SIP traffic from
friends are identical to any other user terminal traffic in format.

\end{itemize}


\section{Music on Hold}
\label{pbx_features/music_on_hold:music-on-hold}\label{pbx_features/music_on_hold::doc}\label{pbx_features/music_on_hold:musiconhold}
The music on hold will be played when the user holds the call and the other
member waits until the call is resumed.

IvozProvider can be configured at two levels what music will be played:
\begin{itemize}
\item {} 
At brand level: \textbf{Brand Configuration} \textgreater{} \textbf{Generic Music on Hold}

\item {} 
At company level: \textbf{Company Configuration} \textgreater{} \textbf{Music on Hold}

\end{itemize}

If a company has defined a music on hold, it will be played. Otherwise, the
one defined by the brand administrator. If none of this is configured, a global
music will be played.

\begin{notice}{note}{Note:}
Multiple files can be added to be played as Music on Hold. The system
will choose them randomly for each call.
\end{notice}
\paragraph{Add a new music on hold}

\noindent\sphinxincludegraphics{{moh_add}.png}

Once the music has been \emph{encoded} the \textbf{Status} fill will display \emph{ready} and
the music will be used for the next calls.

\noindent\sphinxincludegraphics{{moh_add2}.png}

\begin{notice}{tip}{Tip:}
IvozProvider supports most of the common audio formats and \emph{encodes}
them to the optimal format for the platform.
\end{notice}

After the \emph{encoding}, we can download both the original and the converted
version in the edit screen.

\noindent\sphinxincludegraphics{{moh_add3}.png}


\section{Locutions}
\label{pbx_features/sounds:locutions}\label{pbx_features/sounds::doc}
The locutions of the platform are created and uploaded just like the files of
{\hyperref[pbx_features/music_on_hold:musiconhold]{\sphinxcrossref{\DUrole{std,std-ref}{Music on Hold}}}}.

The section \textbf{Company configuration} \textgreater{} \textbf{Locutions}  allows the company admin
to choose the sounds that will be played in many configuration places (IVR, etc)
accross the platform.

\noindent\sphinxincludegraphics{{sounds}.png}

\begin{notice}{attention}{Attention:}
Locutions can be recorded from any terminal by dialing the
Recording extension displayed in their edit screen.
\end{notice}

\begin{notice}{hint}{Hint:}
The main difference between a \textbf{locution} and \textbf{music on hold} is
that the administrator chooses when the first one will be played (out of
schedule, IVRs, and so on) and the second one will be played when a call is
holded by an user.
\end{notice}


\section{Outgoing DDI Rules}
\label{pbx_features/outgoingddi_rules:outgoing-ddi-rules}\label{pbx_features/outgoingddi_rules:outgoingddi-rules}\label{pbx_features/outgoingddi_rules::doc}
Most calling entities in IvozProvider require an outgoing DDI when placing calls
to external numbers. This includes: Users, Friends, Faxes, Retail Accounts, and
so on..

But there are some cases when a single outgoing DDI is not enough, and the
presented DDI depends on the called number. To archive this dynamic outgoing DDI
selection you can use Outgoing DDI rules.

Before creating a new rule, it would be required to frist group the destination
numbers in {\hyperref[pbx_features/match_lists:match\string-lists]{\sphinxcrossref{\DUrole{std,std-ref}{Match Lists}}}}.

For this example, we will create a match list of corporative mobiles with all
the mobile numbers of our company workers. When we call to those numbers, we
will keep the original outgoing DDI assigned to the user, and for the rest of
the cases we will force the DDI to the main company outgoing DDI.
\paragraph{Create a new Outgoing DDI Rule}

The main creation screen defines the action that will take place when no rule
matches the dialed destination, so we define to force the main company DDI here.

\noindent{\hspace*{\fill}\sphinxincludegraphics{{outgoingddi_rules_edit}.png}\hspace*{\fill}}
\paragraph{Assign rule lists actions}

Now we add a new rule that will match our mobiles to make the user's outgoing
DDI be kept uptouched.

\noindent{\hspace*{\fill}\sphinxincludegraphics{{outgoingddi_rulepatterns_edit}.png}\hspace*{\fill}}
\paragraph{Assign rule to callers}

At last, we have to configure who will use this rule to dynamically change it's
presentation number. We can do this in the \textbf{Company's edit screen} or the
\textbf{Users's edit screen}.

\noindent{\hspace*{\fill}\sphinxincludegraphics{{user_outgoingddirule}.png}\hspace*{\fill}}

In this case, the User will present 777777777 DDI when calling Coporative mobiles
and 666666666 when calling the rest of the external numbers.

\begin{notice}{attention}{Attention:}
Current implementation of Outgoing DDI rules won't work for
diverted calls (out of schedule, holidays or user's call forward settings).
\end{notice}


\section{External Call filters}
\label{pbx_features/external_filters:external-call-filters}\label{pbx_features/external_filters:external-filters}\label{pbx_features/external_filters::doc}
One of the most common task a company's administrator will do is to
configure schedules and calendards to apply to existing {\hyperref[pbx_features/external_ddis:external\string-ddis]{\sphinxcrossref{\DUrole{std,std-ref}{External DDIs}}}}.

The first step is creating a schedule.


\subsection{Schedules}
\label{pbx_features/external_filters:schedules}
The section \textbf{Company configuration} \textgreater{} \textbf{Schedule} allows to configure
different time gaps when an {\hyperref[pbx_features/external_ddis:external\string-ddis]{\sphinxcrossref{\DUrole{std,std-ref}{external DDI}}}} will be available.

The screen displayed to the company administrator looks like this:

\noindent\sphinxincludegraphics{{timetables_add}.png}

With the above configuration, we have defined a morining schedule that will be
applied from Monday to Thursday.

We can also define an afternoon schedule for Monday to Thursday too:

\noindent\sphinxincludegraphics{{timetables_add2}.png}

And apply a different time gap for the Fridays:

\noindent\sphinxincludegraphics{{timetables_add3}.png}

We have the following time gaps that combined will determine our company
office schedule.

\noindent\sphinxincludegraphics{{timetables_add4}.png}

\begin{notice}{warning}{Warning:}
The schedule will be defined by combining the active time gaps:
Any time outside this grouped gaps will be considered out-of-schedule.
\end{notice}


\subsection{Calendars}
\label{pbx_features/external_filters:calendars}
Calenders are used to define what days are considered as holiday. Like
schedules, multiples calendars can be combined.

Let's imagine three calendars with the following configuration:

\noindent\sphinxincludegraphics{{calendars_list}.png}

Calendar creation process only requires a name. Once created, we can add what
days will be holidays using the buttons in its row:

\noindent\sphinxincludegraphics{{calendars_add_day}.png}

From this moment on, the calendar has the 1st of January of 2016 as holiday
date with the locution ``Happy New Year''.

\begin{notice}{warning}{Warning:}
Calendars logic is opposite to Schedulers: If a day is not defined
as holiday in any of the calendars, it will considered a normal day and no
filtering will be applied.
\end{notice}

\begin{notice}{hint}{Hint:}
Holidays without special locutions will apply the external call filter
holiday generic locution (see below).
\end{notice}
\paragraph{Create a new External call filter}

Once we have our new created schedules and calendars, it's time to apply them
in what we call \textbf{External call filter}.

The company admin can configure them in the following screen:

\noindent\sphinxincludegraphics{{external_call_filter}.png}
\begin{description}
\item[{Name\index{Name|textbf}}] \leavevmode\phantomsection\label{pbx_features/external_filters:term-name}
Descriptive name that will reference this filter in DDIs configuration.

\item[{Welcome locution\index{Welcome locution|textbf}}] \leavevmode\phantomsection\label{pbx_features/external_filters:term-welcome-locution}
This locution will be played if the call is not going to be
forwarded by out of schedule or holiday filtering (in other words if
the normal routing of the DDI is going to be applied).

\item[{Black list\index{Black list|textbf}}] \leavevmode\phantomsection\label{pbx_features/external_filters:term-black-list}
External origin will be checked against the associated {\hyperref[pbx_features/match_lists:match\string-lists]{\sphinxcrossref{\DUrole{std,std-ref}{Match Lists}}}},
if a coincidence is found, the call will be rejected immediately.

\item[{White list\index{White list|textbf}}] \leavevmode\phantomsection\label{pbx_features/external_filters:term-white-list}
External origin will be checked against the associated {\hyperref[pbx_features/match_lists:match\string-lists]{\sphinxcrossref{\DUrole{std,std-ref}{Match Lists}}}},
if a coincidence is found, the call will be directly routed to the DDI
destination, skipping the filter process. Take into account that black
listed are checked before white lists.

\item[{Holiday locution\index{Holiday locution|textbf}}] \leavevmode\phantomsection\label{pbx_features/external_filters:term-holiday-locution}
The locution will be  played when the day is maked as holiday in any
of the calendars associated with the filter \textbf{if the calendar entry has
no locution} for that day.

\item[{Holiday forward type\index{Holiday forward type|textbf}}] \leavevmode\phantomsection\label{pbx_features/external_filters:term-holiday-forward-type}
After playing the above locution (if configured), call can be forwarded
to a voicemail, external number or internal extension. For example, the
filter of the image will redirect calls during holidays to the external
number 676 676 676.

\item[{Out of schedule locution\index{Out of schedule locution|textbf}}] \leavevmode\phantomsection\label{pbx_features/external_filters:term-out-of-schedule-locution}
The locution will be played when, not being holiday, the current time
is not in any of the time gaps defined in the schedules assigned to the
filter.

\item[{Out of schedule forward type\index{Out of schedule forward type|textbf}}] \leavevmode\phantomsection\label{pbx_features/external_filters:term-out-of-schedule-forward-type}
Like in the holidays forward, but for out of schedule. The image above
won't apply any forward (and the call will be hanguped).

\item[{Calendars\index{Calendars|textbf}}] \leavevmode\phantomsection\label{pbx_features/external_filters:term-calendars}
One or more calendars can be associated with the filter. The combination
of all the callendars will be applied.

\item[{Schedules\index{Schedules|textbf}}] \leavevmode\phantomsection\label{pbx_features/external_filters:term-schedules}
One or more schedules can be applied. The combination of all the time
gaps defined in the schedules will be applied.

\end{description}

\begin{notice}{attention}{Attention:}
Holidays are processed \textbf{before} out of schedule events.
\end{notice}

In the next section we will use this new created filter with
{\hyperref[pbx_features/external_ddis:external\string-ddis]{\sphinxcrossref{\DUrole{std,std-ref}{External DDIs}}}} so we can configure a welcome locution for normal days,
and especial behaviours for hoildays and out of schedule events.


\section{Match Lists}
\label{pbx_features/match_lists:match-lists}\label{pbx_features/match_lists::doc}\label{pbx_features/match_lists:id1}
Mach Lists are designed to group well known numbers or patterns in order to use
them in specific treatments.

Depending on the section used, this numbers can be matched with the origin or
the destination of the call, so be sure to use distinctive names for your match
lists.

For example, like mentioned in the previous section {\hyperref[pbx_features/external_filters:external\string-filters]{\sphinxcrossref{\DUrole{std,std-ref}{External Call filters}}}},
white and black lists contain one or more match lists. In this case, the
\textbf{origin} of the call will be matched agains the list entries to determime if
the treatment of \textbf{skipping} the filter or \textbf{rejecting} the call will be applied.

\begin{notice}{note}{Note:}
Match lists themselfs have no behaviour associated, they only provide
a common way for all process to determine if a number has a treatment.
\end{notice}

\begin{notice}{attention}{Attention:}
Beware that numbers of a Match list are checked against origins
or destinations depending on the configuration section that use them.
\end{notice}

The section \textbf{Company configuration} \textgreater{} \textbf{Match Lists} allows to configure
different items that will group the numbers and patterns.

The screen displayed to the company administrator looks like this:

\noindent{\hspace*{\fill}\sphinxincludegraphics{{matchlist_add}.png}\hspace*{\fill}}

After creating a new Match list, you can include numbers and patterns.

\noindent{\hspace*{\fill}\sphinxincludegraphics{{matchlist_patterns_add}.png}\hspace*{\fill}}

As shown, the match list can contain specific numbers or groups using
\href{http://php.net/manual/en/reference.pcre.pattern.syntax.php}{Regular Expressions}


\section{Route locks}
\label{pbx_features/route_locks:id1}\label{pbx_features/route_locks::doc}\label{pbx_features/route_locks:route-locks}
Route locks are a simple but powerful way to fork route logics when delivering calls. This fork is done depending on the
state of the lock on a particular moment:
\begin{itemize}
\item {} 
\textbf{Opened}: green light, go ahead.

\item {} 
\textbf{Closed}: red light, no trespassing allowed.

\end{itemize}

They are used as conditional route rule criteria (see how in {\hyperref[pbx_features/conditional_routes:conditional\string-routes]{\sphinxcrossref{\DUrole{std,std-ref}{Conditional routes}}}}).


\subsection{Route lock creation}
\label{pbx_features/route_locks:route-lock-creation}
When you add a new route lock in \textbf{Route Locks} section, you are asked for the following fields:
\begin{description}
\item[{Name\index{Name|textbf}}] \leavevmode\phantomsection\label{pbx_features/route_locks:term-name}
This name will be used in conditional routes to identify the lock.

\item[{Description\index{Description|textbf}}] \leavevmode\phantomsection\label{pbx_features/route_locks:term-description}
Just a description.

\item[{Status\index{Status|textbf}}] \leavevmode\phantomsection\label{pbx_features/route_locks:term-status}
Set the initial status of the lock: opened or closed.

\end{description}


\subsection{Route locks service codes}
\label{pbx_features/route_locks:route-locks-service-codes}
Although you can set the initial lock status on creation and change it using the admin portal too, the usual way to
handle the status changes of a lock is to use the service codes listed in \textbf{Route locks} section.

These services codes have two parts:
\begin{itemize}
\item {} 
\textbf{Service code}: configured in \textbf{Services} section per brand/company.

\item {} 
\textbf{Lock id}: immutable numeric id assigned to each lock.

\end{itemize}

\begin{notice}{tip}{Tip:}
There are 3 service codes available for most common operations on locks:
\begin{itemize}
\item {} 
Open Lock

\item {} 
Close Lock

\item {} 
Toggle Lock.

\end{itemize}

Read {\hyperref[pbx_features/services:services]{\sphinxcrossref{\DUrole{std,std-ref}{Additional services}}}} for further details.
\end{notice}


\section{Conditional routes}
\label{pbx_features/conditional_routes::doc}\label{pbx_features/conditional_routes:conditional-routes}\label{pbx_features/conditional_routes:id1}
Conditional routes allows changing a call logic depending on:
\begin{itemize}
\item {} 
Who is calling.

\item {} 
What time is calling.

\item {} 
What day is calling.

\item {} 
Route lock status

\end{itemize}

These routes are electable in three sections:
\begin{itemize}
\item {} 
DDIs

\item {} 
Extensions

\item {} 
IVR custom options

\end{itemize}

\begin{notice}{tip}{Tip:}
Remaining sections could use conditional routes creating an extension
that point to a conditional route first, and routing to this extension.
\end{notice}


\subsection{Creating a conditional route}
\label{pbx_features/conditional_routes:creating-a-conditional-route}
First of all we create a conditional route in \textbf{Conditional routes} section:

\noindent\sphinxincludegraphics{{conditional_route_edit}.png}

On creation we define what should be done with a call that does not satisfy any
of the rules described below.


\subsection{Adding rules}
\label{pbx_features/conditional_routes:adding-rules}
Once created, we need to add rules, for example:
\paragraph{Calls from Japan and Germany received in the morning to an specific user}

\noindent\sphinxincludegraphics{{conditional_route_rule1}.png}
\paragraph{Calls from Japan and Germany received in the afternoon to another user}

\noindent\sphinxincludegraphics{{conditional_route_rule2}.png}
\paragraph{Override the reception IVR for summer days}

\noindent\sphinxincludegraphics{{conditional_route_rule3}.png}

With this example rules, our example conditional route will look like this:

\noindent\sphinxincludegraphics{{conditional_route_rules}.png}

\begin{notice}{note}{Note:}
Since IvozProvider 1.7 \textbf{Route locks} (see {\hyperref[pbx_features/route_locks:route\string-locks]{\sphinxcrossref{\DUrole{std,std-ref}{Route locks}}}}) can be used as a criteria too.
\end{notice}

Some notes about this example:
\begin{itemize}
\item {} 
Rules are evaluated following the metric parameter. Once a rule matches, its
logic is applied.

\item {} 
Rules may have from 1 to 4 criteria:
\begin{itemize}
\item {} 
None, one or more matchlist (pre-created, see {\hyperref[pbx_features/match_lists:match\string-lists]{\sphinxcrossref{\DUrole{std,std-ref}{Match Lists}}}})

\item {} 
None, one or more schedules (pre-created, see {\hyperref[pbx_features/external_filters:schedules]{\sphinxcrossref{\DUrole{std,std-ref}{Schedules}}}})

\item {} 
None, one or more calendar (pre-created, see {\hyperref[pbx_features/external_filters:calendars]{\sphinxcrossref{\DUrole{std,std-ref}{Calendars}}}})

\item {} 
None, one or more route locks (pre-created, see {\hyperref[pbx_features/route_locks:route\string-locks]{\sphinxcrossref{\DUrole{std,std-ref}{Route locks}}}})

\end{itemize}

\item {} 
These 4 criteria are combined (applying an AND logic).

\end{itemize}

\begin{notice}{important}{Important:}
When adding more than one route lock, if any of them is Opened, this criteria will be considered as fulfilled.
\end{notice}


\subsection{Using a conditional route}
\label{pbx_features/conditional_routes:using-a-conditional-route}
The behaviour when an IVR option or an extension is routed to a conditional
route is easy to understand, but using conditional routes with DDIs need an
additional explanation.

Imagine this scenario:

\noindent\sphinxincludegraphics{{conditional_route_ddi}.png}

DDI has an external call filter and is routed to the new conditional route.

When a call is received:
\begin{itemize}
\item {} 
External call filter is evaluated:
\begin{itemize}
\item {} 
If current day is marked in any calendar, the holiday logic applies.

\item {} 
If current time is not inside any time-gap, out-of-schedule logic applies.

\end{itemize}

\item {} 
If external call filter logics have not applied, conditional route is evaluated.

\end{itemize}

\begin{notice}{attention}{Attention:}
Conditional route is not intented as an external call filter
replacement. Filter is evaluated first, conditional route afterwards.
\end{notice}


\section{Hunt groups}
\label{pbx_features/huntgroups:hunt-groups}\label{pbx_features/huntgroups::doc}\label{pbx_features/huntgroups:huntgroups}
The hungroups allows configuring more complex \emph{ringing} process that the
traditional \textbf{call to a user}.

There are multiple types:
\begin{description}
\item[{Ring all\index{Ring all|textbf}}] \leavevmode\phantomsection\label{pbx_features/huntgroups:term-ring-all}
The call will make all the terminals of the group during a predefined
time.

\item[{Sequential\index{Sequential|textbf}}] \leavevmode\phantomsection\label{pbx_features/huntgroups:term-sequential}
The call will \emph{jump} from one user to another in a predefined order
ringing during the configured time. If the call is not answered by any
user of the group, it will be hanguped (or trigger the no answer logic).

\item[{Sequential (infinite)\index{Sequential (infinite)|textbf}}] \leavevmode\phantomsection\label{pbx_features/huntgroups:term-sequential-infinite}
The call will \emph{jump} from one user to another in a predefined order
ringing during the configured time. If the call is not answered by any
user of the group, the call will \emph{jump} again to the first member of the
group and keep looping.

\item[{Random\index{Random|textbf}}] \leavevmode\phantomsection\label{pbx_features/huntgroups:term-random}
The call will \emph{jump} from one user to another in a random order,
ringing during the configured time.  If the call is not answered by any
user of the group, it will be hanguped (or trigger the no answer logic).

\end{description}
\paragraph{Example 1: \emph{Ringall} hunt group}

The following example will show how to create a hunt group that will call our 2
users at the same time during 30 seconds:

\noindent\sphinxincludegraphics{{huntgroup_add}.png}

Pressing the proper icon, we can add Alice and Bob to the hunt group:

\noindent\sphinxincludegraphics{{huntgroup_add2}.png}
\paragraph{Example 2: \emph{Sequential} hunt group}

We will edit the hunt group to convert it into sequential: the call will ring
Alice during 10 seconds, then it will ring Bob 15 seconds, repeting this process
until one of them answers.

\noindent\sphinxincludegraphics{{huntgroup_add3}.png}

For this type of groups we have to configure priority (the call will \emph{jump}
from the users with lower number priority to the ones with higher number
priority) and a \emph{ringing} time for each user.

\noindent\sphinxincludegraphics{{huntgroup_add4}.png}

\begin{notice}{hint}{Hint:}
Hunt groups can be routed from any process of IvozProvider by simply
adding an extension that route to them.
\end{notice}

Let's create a new extension that routes to this hunt group:

\noindent\sphinxincludegraphics{{huntgroup_extension}.png}

\begin{notice}{hint}{Hint:}
When configuring a huntgroup, you can prevent missed calls on called members with \textbf{Prevent missed calls} setting:
\begin{itemize}
\item {} 
\textbf{Yes}: calls generated by the huntgroup will never generate missed calls on called members.

\item {} 
\textbf{No}: The behaviour of this setting depends on the huntgroup type:
\begin{itemize}
\item {} 
\textbf{RingAll}: calls generated by the huntgroup will generate missed calls on called members only if none of them answers the call.

\item {} 
\textbf{Remaining types}: calls generated by the huntgroup will generate missed calls on every called member that does not answer the call.

\end{itemize}

\end{itemize}
\end{notice}


\section{Queues}
\label{pbx_features/queues:queues}\label{pbx_features/queues::doc}
Easy queue behaviour was included in IvozProvider in 1.3 version. It is a simple
approach with \textbf{the unique goal to provide the capability to handle more calls
than users attending them}.

\begin{notice}{warning}{Warning:}
Queues and callcenter are close terms but different. \textbf{IvozProvider
is not a suitable product for callcenters}, as it does not provide
advanced features that are crucial to them (reports, RT visualization,
queue related stat, etc.).
\end{notice}

\textbf{In distributed installations} using Queues is only compatible with an static
assignment or `hash based' distribution (see \textbf{Distribute method} {\hyperref[internal_calls/brand_portal:remaining\string-parameters]{\sphinxcrossref{\DUrole{std,std-ref}{here}}}}).

\begin{notice}{hint}{Hint:}
Brand operators can choose which Companies have queues (see \textbf{Features}
in {\hyperref[internal_calls/brand_portal:brand\string-configuration]{\sphinxcrossref{\DUrole{std,std-ref}{Brand Configuration}}}} and {\hyperref[internal_calls/company_portal:company\string-configuration]{\sphinxcrossref{\DUrole{std,std-ref}{Company Configuration}}}}).
\end{notice}


\subsection{Queue configuration}
\label{pbx_features/queues:queue-configuration}
This are the settings related to a queue:
\begin{description}
\item[{Name\index{Name|textbf}}] \leavevmode\phantomsection\label{pbx_features/queues:term-name}
Use to reference this queue

\item[{Weight\index{Weight|textbf}}] \leavevmode\phantomsection\label{pbx_features/queues:term-weight}
Priorizes calls to an agent that attends calls in two (or more) calls. The
higher, the more priorized.

\item[{Strategy\index{Strategy|textbf}}] \leavevmode\phantomsection\label{pbx_features/queues:term-strategy}
How will the queue deliver the calls? Calling to all agents, calling to a
random one?

\item[{Member call seconds\index{Member call seconds|textbf}}] \leavevmode\phantomsection\label{pbx_features/queues:term-member-call-seconds}
Defines how long will a call to an agent last.

\item[{Member rest seconds\index{Member rest seconds|textbf}}] \leavevmode\phantomsection\label{pbx_features/queues:term-member-rest-seconds}
Seconds between calls for an agent.

\item[{Announce\index{Announce|textbf}}] \leavevmode\phantomsection\label{pbx_features/queues:term-announce}
Select a locution and its frequency. Caller waiting in the call will listen
to this locution.

\item[{Timeout configuration\index{Timeout configuration|textbf}}] \leavevmode\phantomsection\label{pbx_features/queues:term-timeout-configuration}
Limits the time that a call can wait in a queue and the following behaviour.

\item[{Full Queue configuration\index{Full Queue configuration|textbf}}] \leavevmode\phantomsection\label{pbx_features/queues:term-full-queue-configuration}
Limits the amount of people waiting in a call and the behaviour when this limit
it reached.

\end{description}

Apart from creating a queue, you have to assign users to it. This users will have
a \textbf{penalty: a user will not be selected to deliver a call if any user with lower
penalty is available}.

\begin{notice}{hint}{Hint:}
A call can be sent to a queue selecting it in the ``Route type'' selectors
available in multiple sections of IvozProvider (extension to queue, DDI
to queue, etc.)
\end{notice}


\subsection{Queue strategy}
\label{pbx_features/queues:queue-strategy}
The queue strategy \textbf{always applies to current penalty members} starting with
the smallest penalty value and only going to the next penalty if all members of
current one are busy or unavailable.
\begin{description}
\item[{Ring all\index{Ring all|textbf}}] \leavevmode\phantomsection\label{pbx_features/queues:term-ring-all}
The call will make all the members of the current priority during a
predefined time.

\item[{Least recent\index{Least recent|textbf}}] \leavevmode\phantomsection\label{pbx_features/queues:term-least-recent}
The call will \emph{jump} from one member to another in a predefined order
based on the last time the member attended a call. Members whose latest
call is older will be called first.

\item[{Fewer calls\index{Fewer calls|textbf}}] \leavevmode\phantomsection\label{pbx_features/queues:term-fewer-calls}
The call will \emph{jump} from one member to another in a predefined order
based on the number of atteded calls. Members that have attedended less
calls will be called first.

\item[{Random\index{Random|textbf}}] \leavevmode\phantomsection\label{pbx_features/queues:term-random}
The call will \emph{jump} from one member to another in a random order,
ringing during the configured time.

\item[{Round Robin memory\index{Round Robin memory|textbf}}] \leavevmode\phantomsection\label{pbx_features/queues:term-round-robin-memory}
The call will \emph{jump} from one member to another in a predefined order
starting past the last member that attended a call.

\item[{Linear\index{Linear|textbf}}] \leavevmode\phantomsection\label{pbx_features/queues:term-linear}
The call will \emph{jump} from one member to another in a predefined order
based on the creation time of the member.

\end{description}


\section{Interactive Voice Response (IVR)}
\label{pbx_features/ivrs::doc}\label{pbx_features/ivrs:interactive-voice-response-ivr}
IVRs are the most common way to make \textbf{audio menus} where the caller must
choose the destination of the call by \textbf{pressing codes} based on the locutions
instructions that will be played.


\subsection{Generic IVRs}
\label{pbx_features/ivrs:generic-ivrs}\label{pbx_features/ivrs:id1}
In this type of IVRs, the caller will directly press the extension that must
previously know (or the welcome locution suggests) and the system will
automatically connect with that extension:

Generic IVRs have the following fields:
\begin{description}
\item[{Name\index{Name|textbf}}] \leavevmode\phantomsection\label{pbx_features/ivrs:term-name}
Descriptive name of the IVR that will be used in other sections.

\item[{Timeout\index{Timeout|textbf}}] \leavevmode\phantomsection\label{pbx_features/ivrs:term-timeout}
Time that caller has to enter the digits of the target extension.

\item[{Welcome locution\index{Welcome locution|textbf}}] \leavevmode\phantomsection\label{pbx_features/ivrs:term-welcome-locution}
This locution will be played as soon as the caller enters the IVR.

\item[{Success locution\index{Success locution|textbf}}] \leavevmode\phantomsection\label{pbx_features/ivrs:term-success-locution}
In case the dialed extension exists in the company, this locution will
be played (usually something like `Connecting, please wait...').

\item[{No input process\index{No input process|textbf}}] \leavevmode\phantomsection\label{pbx_features/ivrs:term-no-input-process}
If the caller does not input any digit in the timeout value, the
no input process will trigger, playing the configured locution and
redirecting the call to another number, extension or voicemail.

\item[{Error process\index{Error process|textbf}}] \leavevmode\phantomsection\label{pbx_features/ivrs:term-error-process}
If the dialed extension is invalid, the error process will trigger,
playing the configured locution and redirecting the call to another
number, extension or voicemail.

\end{description}


\subsection{Custom IVRs}
\label{pbx_features/ivrs:id2}\label{pbx_features/ivrs:custom-ivrs}
Contrary to the generic IVRs where the caller can only dial internal
extensions, the custom IVRS can configure options that can be routed
in different ways.

\begin{notice}{hint}{Hint:}
The most common usage for this IVR is combining them with a welcome
locution that says something like `Press 1 to contact XXX, Press 2 to
contact YYY, ...''
\end{notice}

Most of the configurable fields are the same that generic IVR uses:

The process of each entry of the IVR can be defined in the following button:

\noindent\sphinxincludegraphics{{ivr_custom2}.png}

In this example, the caller can dial 1, 2 or 3 (the rest will be considered as
an error and will trigger the \textbf{Error process}):

\noindent\sphinxincludegraphics{{ivr_custom3}.png}
\begin{itemize}
\item {} 
1: Call to the internal extension 200, created in {\hyperref[pbx_features/huntgroups:huntgroups]{\sphinxcrossref{\DUrole{std,std-ref}{previous section}}}} that routes to hunt group \emph{Reception}.

\item {} 
2: Call to the internal extension 101.

\item {} 
3: Route this call to the external number 676 676 676.

\end{itemize}

\begin{notice}{note}{Note:}
Each of the Custom IVR entries supports a locution that, if set,
will be played instead of the IVR \textbf{success locution}. This way, you can
configure a generic locution (like `Connecting....') or a custom one for
a given entry (like `Connecting reception department, please wait...').
\end{notice}
\paragraph{Entries are regular expressions}

Although on the most typical usage of this IVRs options will be digits from 1
to 9, \textbf{entries are interpreted as regular expressions}. This way, you could
add an entry like ``\textasciicircum{}2{[}0-9{]}\{2\}\$'' to group the behaviour of all numbers from 200
to 299. With this usage, \textbf{Max digits} parameter is important too.

\begin{notice}{error}{Error:}
To avoid undesired behaviour, \textbf{if you use options out of 0-9, use
regular expression notation} (`\textasciicircum{}1\$' instead of `1', `\textasciicircum{}10\$' instead
of `10' and so on).
\end{notice}


\section{Conference rooms}
\label{pbx_features/conference_rooms::doc}\label{pbx_features/conference_rooms:conference-rooms}\label{pbx_features/conference_rooms:id1}
IvozProvider supports Conference rooms that can be configured in the section
\textbf{Company configuration} \textgreater{} \textbf{Conference rooms}.
\paragraph{Create a new audio conference}

The following image shows the process of creating a new confrence room:

\noindent\sphinxincludegraphics{{conference_room}.png}
\begin{description}
\item[{Name\index{Name|textbf}}] \leavevmode\phantomsection\label{pbx_features/conference_rooms:term-name}
Name that will used to identify this conference room in other sections

\item[{Max members\index{Max members|textbf}}] \leavevmode\phantomsection\label{pbx_features/conference_rooms:term-max-members}
Maximum number of participants in the conference. When this limit is
reached, join requests will be rejected.

\item[{Pin protected\index{Pin protected|textbf}}] \leavevmode\phantomsection\label{pbx_features/conference_rooms:term-pin-protected}
Conference rooms can be pin protected. The pin will be requested before
entering and must be numeric.

\end{description}

\begin{notice}{note}{Note:}
Member limit can be disabled by setting it to 0.
\end{notice}
\paragraph{Route an extension or DDI to the conference}

In order to enter a conference there must be a number that is route to them:

\noindent\sphinxincludegraphics{{conference_room_ext}.png}

In the following section we will see how to configure a {\hyperref[pbx_features/external_ddis:external\string-ddis]{\sphinxcrossref{\DUrole{std,std-ref}{external DDI}}}} to a conference room so it can be used by external callers.

\begin{notice}{hint}{Hint:}
There are other ways to make external callers join a conference room
without using a DDI: it can be assigned to an Extension. This way, any user
can transfer the call to the conference extension, or can be routed, for
exmaple using an IVR entry.
\end{notice}


\section{External DDIs}
\label{pbx_features/external_ddis:external-ddis}\label{pbx_features/external_ddis::doc}\label{pbx_features/external_ddis:id1}
In the previous section {\hyperref[external_incoming_calls/configure_ddi:settingup\string-ddi]{\sphinxcrossref{\DUrole{std,std-ref}{Configuring an external DDI}}}} we described in detail the required
configuration for an external DDI routed to user. This section will go a bit
further and explain the rest of configurable options of DDIs.


\subsection{DDI external filters}
\label{pbx_features/external_ddis:ddi-external-filters}
We can assign a \textbf{external call filter} configured in {\hyperref[pbx_features/external_filters:external\string-filters]{\sphinxcrossref{\DUrole{std,std-ref}{previous section}}}}.

\noindent\sphinxincludegraphics{{ddi_edit}.png}


\subsection{DDI routes}
\label{pbx_features/external_ddis:routing-logics}\label{pbx_features/external_ddis:ddi-routes}
Once the call has passed all the checks in the filter (schedules and calendars)
and after the welcome locution has been played (if there is any configured),
we can route the call to the following processes:
\begin{itemize}
\item {} 
{\hyperref[pbx_features/huntgroups:huntgroups]{\sphinxcrossref{\DUrole{std,std-ref}{Hunt groups}}}}

\end{itemize}

\noindent\sphinxincludegraphics{{ddi_edit2}.png}
\begin{itemize}
\item {} 
{\hyperref[pbx_features/ivrs:generic\string-ivrs]{\sphinxcrossref{\DUrole{std,std-ref}{Generic IVR}}}}

\end{itemize}

\noindent\sphinxincludegraphics{{ddi_edit3}.png}
\begin{itemize}
\item {} 
{\hyperref[pbx_features/ivrs:custom\string-ivrs]{\sphinxcrossref{\DUrole{std,std-ref}{Custom IVR}}}}

\end{itemize}

\noindent\sphinxincludegraphics{{ddi_edit4}.png}
\begin{itemize}
\item {} 
{\hyperref[pbx_features/conference_rooms:conference\string-rooms]{\sphinxcrossref{\DUrole{std,std-ref}{Conference rooms}}}}

\end{itemize}

\noindent\sphinxincludegraphics{{ddi_edit5}.png}

\begin{notice}{hint}{Hint:}
We can also route the DDI to a {\hyperref[faxing/index:faxing\string-system]{\sphinxcrossref{\DUrole{std,std-ref}{Virtual Fax}}}}, but
this is something we will explain in the following block.
\end{notice}


\section{Call pickup}
\label{pbx_features/call_captures:capture-groups}\label{pbx_features/call_captures::doc}\label{pbx_features/call_captures:call-pickup}
Call pickup is the process where a user can answer a call that is being ringing
in another terminal. No need to say that, somehow (sound, flashing lights,
notification, etc) the users must know that the call is ringing elsewhere.

IvozProvider supports two kind of call pickups:
\begin{description}
\item[{Direct pickup\index{Direct pickup|textbf}}] \leavevmode\phantomsection\label{pbx_features/call_captures:term-direct-pickup}
In this type of pickup, the user that is trying to capture the ringing
call must include the extension of the target phone after the service
code. For example, if the direct pickup code is *95, the user must
dial *95101 to capture a call that is ringing in the extension 101.

\item[{Group pickup\index{Group pickup|textbf}}] \leavevmode\phantomsection\label{pbx_features/call_captures:term-group-pickup}
In this type of pickup, the user that is trying to capture the ringing
call will just dial the service code. If anyone in any of the pickup
groups of the user has a ringing call, it will be answered by the
capturer.

\end{description}


\subsection{Call pickup groups}
\label{pbx_features/call_captures:call-pickup-groups}
In order to make \textbf{call group pickups}, the capturer user must be part of the
same group that the target user that wants to capture.

The section \textbf{Pickup groups} allows the company administrator to configure
what users will be in each group:

\noindent\sphinxincludegraphics{{capture_groups}.png}

As shown in the section {\hyperref[pbx_features/users:users]{\sphinxcrossref{\DUrole{std,std-ref}{Users configuration}}}}, we can add or edit the groups of a user
in the user's edit screen.

\begin{notice}{note}{Note:}
A user can be part of multiple pickup groups. The system will take
all of them into account when using the group pickup service.
\end{notice}


\subsection{Group pickup service code}
\label{pbx_features/call_captures:group-pickup-service-code}
IvozProvider supports 2 different configuration levels for defining the service
codes for pickup:
\begin{itemize}
\item {} 
At brand level: \textbf{Brand configuration} \textgreater{} \textbf{Services}.

\item {} 
At company level: \textbf{Company configuration} \textgreater{} \textbf{Services}.

\end{itemize}

The brand administrator can configure generic codes that all the companies will
use. Companies can customize this codes if they are used to another ones.

The {\hyperref[pbx_features/services:services]{\sphinxcrossref{\DUrole{std,std-ref}{following section}}}} explains the services in depth, with
all the additional services that can be accessed by dialing codes starting with
*.


\section{Additional services}
\label{pbx_features/services:services}\label{pbx_features/services:additional-services}\label{pbx_features/services::doc}
There are \textbf{special services} that can be accessed by calling to some codes
\textbf{from the terminal}.

\begin{notice}{danger}{Danger:}
Services defined in this section \textbf{are not accessible during a
conversation}. They are activated by \textbf{calling the codes}, not using
DTMF codes while talking.
\end{notice}


\subsection{Global Services codes}
\label{pbx_features/services:global-services-codes}
There are the folowing \textbf{special services} avaialble in the section \textbf{Global
configuration} \textgreater{} \textbf{Services}:

\noindent\sphinxincludegraphics{{services_god}.png}
\begin{description}
\item[{Direct pickup\index{Direct pickup|textbf}}] \leavevmode\phantomsection\label{pbx_features/services:term-direct-pickup}
This service allows capturing a ringing call from another terminal by
calling the code followed by the extension from the target user.

\item[{Group pickup\index{Group pickup|textbf}}] \leavevmode\phantomsection\label{pbx_features/services:term-group-pickup}
This service allows capturing a ringing call for any terminal whose user
is part of one of the capturer pickup groups.

\item[{Check voicemail\index{Check voicemail|textbf}}] \leavevmode\phantomsection\label{pbx_features/services:term-check-voicemail}
This service allows checking the user's voicemail using an interactive
menu from which new voicemails can be listen, deleted, etc. This is an
active alternative to receive voicemails via the email. Since 1.4, this
service allows optional extension after the service code to check
another users voicemails. Users can protect their voicemail using the
internal menu options.

\item[{Record locution\index{Record locution|textbf}}] \leavevmode\phantomsection\label{pbx_features/services:term-record-locution}
This service allows any user to record their company's locutions by
dialing an special code. Voice instructions will be provided in the
user's language.

\item[{Open Lock\index{Open Lock|textbf}}] \leavevmode\phantomsection\label{pbx_features/services:term-open-lock}
Calling this service code will set route lock status to `Opened' (see {\hyperref[pbx_features/route_locks:route\string-locks]{\sphinxcrossref{\DUrole{std,std-ref}{Route locks}}}}).

\item[{Close Lock\index{Close Lock|textbf}}] \leavevmode\phantomsection\label{pbx_features/services:term-close-lock}
Calling this service code will set route lock status to `Closed' (see {\hyperref[pbx_features/route_locks:route\string-locks]{\sphinxcrossref{\DUrole{std,std-ref}{Route locks}}}}).

\item[{Toggle Lock\index{Toggle Lock|textbf}}] \leavevmode\phantomsection\label{pbx_features/services:term-toggle-lock}
Calling this service code will change the current status of the lock (see {\hyperref[pbx_features/route_locks:route\string-locks]{\sphinxcrossref{\DUrole{std,std-ref}{Route locks}}}}).

\end{description}

As soon as new services are implemented into IvozProvider, they will be listed
in this section.

\begin{notice}{attention}{Attention:}
This section lists the available services and the default codes
when a \textbf{new brand} is created.
\end{notice}

\begin{notice}{hint}{Hint:}
Changing the default code in this section will only affect new
created brands.
\end{notice}


\subsection{Services and codes for Brands}
\label{pbx_features/services:services-and-codes-for-brands}
The section \textbf{Brand configuration} \textgreater{} \textbf{Services} allows the brand operator to:
\begin{itemize}
\item {} 
Change the default service code for all the brand companies (assuming the
company hasn't already customized the code)

\item {} 
Delete services so it won't be available for the companies.

\end{itemize}

By default this list has all the services and codes from the global
configuration:

\begin{notice}{attention}{Attention:}
This section lists the available services and the default codes
when a \textbf{new company} is created.
\end{notice}

\begin{notice}{hint}{Hint:}
Changing the default code in this section will only affect new
created companies.
\end{notice}


\subsection{Services and codes for Companies}
\label{pbx_features/services:services-and-codes-for-companies}
Each company can \emph{customize} the default values assigned by the \emph{brand operator}
using the section \textbf{Company configuration} \textgreater{} \textbf{Services} and changing the codes
listed there.
\paragraph{Company that wants to capture using ** instead of the default *95:}

\noindent\sphinxincludegraphics{{services_company_edit}.png}

\begin{notice}{hint}{Hint:}
Services deleted by the \emph{company admin} will not available to users.
\end{notice}


\section{Call recording}
\label{pbx_features/call_recording:call-recording}\label{pbx_features/call_recording::doc}\label{pbx_features/call_recording:call-recordings}
\begin{notice}{attention}{Attention:}
Beware that local legislation may enforce to announce that the
call is being recorded (sometimes to both parties). You should include
a recording disclaimer in your welcome locutions for DDIs with automatic
recording enabled.
\end{notice}

IvozProvider supports two different ways of recording calls:
\begin{itemize}
\item {} 
\textbf{Automatic recordings} for the incoming/outgoing calls that use a
{\hyperref[pbx_features/external_ddis:external\string-ddis]{\sphinxcrossref{\DUrole{std,std-ref}{External DDI}}}}.

\item {} 
\textbf{On demand recordings} requested by a user during a call.

\end{itemize}


\subsection{Automatic DDI recordings}
\label{pbx_features/call_recording:automatic-ddi-recordings}
In this type of recording, \textbf{the whole conversation will be recorded}: from
the start until it finishes.

Two different scenarios:
\begin{itemize}
\item {} 
\textbf{Incoming calls to a DDI}: The call will continue until the external
dialer hangups (no matter whom is talking to).

\item {} 
\textbf{Outgoing calls using a DDI} as {\hyperref[pbx_features/external_ddis:external\string-ddis]{\sphinxcrossref{\DUrole{std,std-ref}{Outgoing DDI}}}}: the
recording will continue as long as the external destination keeps in the
conversation.

\end{itemize}

\begin{notice}{attention}{Attention:}
Take into account that the call will be recorded while the
external entity is present, even it the call is being transfered between
multiple users of the platform.
\end{notice}
\paragraph{Record all the calls of a DDI}

To enable this feature, edit the DDI and configure the field under the section
recording data:

\noindent\sphinxincludegraphics{{recordings_ddi}.png}

There are 4 available options:
\begin{itemize}
\item {} 
Disable recordings

\item {} 
Enable incoming recordings

\item {} 
Enable outgoing recordings

\item {} 
Enable all call recordings

\end{itemize}


\subsection{On demand recordings}
\label{pbx_features/call_recording:on-demand-recordings}
The \emph{on-demand} recordings must be enabled by the \emph{brand administator} for the
companies that request it. This can be done in the companey edit screen:

\noindent\sphinxincludegraphics{{recordings_ondemand}.png}

\begin{notice}{warning}{Warning:}
Contrary to the {\hyperref[pbx_features/services:services]{\sphinxcrossref{\DUrole{std,std-ref}{Services}}}} mentioned in the
previous section, the on demand record are actived within a conversation.
\end{notice}

Contrary to automatic ones, on demand recording can be stoped using the same
process that started them.


\subsubsection{Activated using the \emph{Record} key}
\label{pbx_features/call_recording:activated-using-the-record-key}
Some terminals (for example, \emph{Yealink}) support sending a \href{https://tools.ietf.org/html/rfc6086}{SIP INFO} message during the conversation with a
special \emph{Record} header (see \href{http://www.yealink.com/Upload/document/UsingCallRecordingFeatureonYealinkPhones/UsingCallRecordingFeatureonYealinkSIPT2XPphonesRev\_610-20561729764.pdf}{reference}).
This is not a standard for the protocol, but being Yealink one of the supported
manufacturers of the solution, we include this kind of on-demand recording.

\begin{notice}{important}{Important:}
For this recording requests, the configured code doesn't matter
but the company still must have on demand records enabled.
\end{notice}

To start or stop this kind of recordings, just press the Record key in the
terminal and the system will handle the sent message.


\subsubsection{Activated using \emph{DTMF} codes}
\label{pbx_features/call_recording:activated-using-dtmf-codes}
The more traditional approach for this feature is to press a combination of
keys during the call. Some notification will be played and the recording will
start or stop. This combination is sent to the system using \href{https://es.wikipedia.org/wiki/Marcaci\%C3\%B3n\_por\_tonos}{DTMF tones} using the same audio
stream that the conversation (as mentioned in \href{https://tools.ietf.org/html/rfc4733}{RFC 4733}).

IvozProvider supports this kind of on demand record activation but with an
important downside. In order to capture this codes, the pbx must process each
audio packet to detect the code, avoiding the direct flow of media between the
final endpoints.

\begin{notice}{important}{Important:}
Enabling this record mode highly affects the performance of the
platform. Use at your own risk.
\end{notice}


\subsubsection{Activated using a frustated blind transfer}
\label{pbx_features/call_recording:activated-using-a-frustated-blind-transfer}
There is a tricky way to access this feature for terminals that does not support
the INFO message and don't want its audio to be parsed:

\begin{notice}{danger}{Danger:}
This method is a workaround for those terminals that does not support
the native \emph{Record} key activation (recommended). Take into account that not
all terminals will behave the same way while performing the transfer described
in this section.
\end{notice}

The keys for this methods are:
\begin{itemize}
\item {} 
It's not activated using a code during the conversation.

\item {} 
It's activated making a \textbf{blind transfer} to the on demand record code.

\item {} 
The system will understand this as a request to record and will reject the transfer.

\item {} 
The user will continue with the existing call and keep talking.

\end{itemize}
\paragraph{Why this method does even exist?}

The reason behind this tricky method is based, as explained in the previous
block, on the design of the {\hyperref[architecture/index:architecture]{\sphinxcrossref{\DUrole{std,std-ref}{Platform general architecture}}}} and the {\hyperref[architecture/index:audioflow]{\sphinxcrossref{\DUrole{std,std-ref}{RTP audio
flow}}}}.


\subsection{Recordings list}
\label{pbx_features/call_recording:recordings-list}
The \emph{company administrator} can access to all the recordings in the section
\textbf{Company configuration} \textgreater{} \textbf{Recordings}:

\noindent\sphinxincludegraphics{{recordings_list}.png}

Recordings can be heard from the \emph{web} or downloaded in MP3 format:

\noindent\sphinxincludegraphics{{recordings_list2}.png}

If the recording has been started on demand, it will also include the user
that requested it:

\noindent\sphinxincludegraphics{{recordings_list3}.png}


\chapter{Retail Clients}
\label{retail/index:retail-clients}\label{retail/index::doc}
Retail clients are a special type of company that only provide a connectivity
service with contract peerings through retail accounts.

\begin{notice}{attention}{Attention:}
Contrary to the Virtual PBX companies, all Retail clients use the
brand domain to unequivocally identify their accounts. You'll need to configure
Brand's domain to use this feature.
\end{notice}

\begin{notice}{hint}{Hint:}
Retail clients can be enabled per Brand basis via Features.
\end{notice}

The goal of this section will be describe each of the configuration settings
associated with Retail clients included in IvozProvider.


\section{Retail Accounts}
\label{retail/retail_accounts::doc}\label{retail/retail_accounts:retail-accounts}\label{retail/retail_accounts:id1}
Retail Accounts are the main routable option in Retail clients.
More or less like {\hyperref[pbx_features/friends:friends]{\sphinxcrossref{\DUrole{std,std-ref}{Friends}}}} are to Virtual PBX Companies, accounts
contain the required configurable options to provide a SIP connectivity
service with IvozProvider and an external SIP entity.

\begin{notice}{warning}{Warning:}
Although both \textbf{Contract peering} and \textbf{Retail accounts} are defined by the
\textbf{brand operator}, the first ones are designed to connect with the public network
while the second ones connect the system with other SIP agents.
\end{notice}


\subsection{Types of retail accounts}
\label{retail/retail_accounts:types-of-retail-accounts}
There are 2 main types of SIP PBX that can use retail with IvozProvider:
\begin{itemize}
\item {} 
\textbf{Direct connection PBX}: IvozProvider must be able to talk SIP directly with
this kind of accounts by just redirecting the traffic to the proper port of
the public IP address of the PBX.

\item {} 
\textbf{PBX behind NAT}: Not directly accesible. This kind of PBX must register at
IvozProvider (just like all the {\hyperref[pbx_features/terminals:terminals]{\sphinxcrossref{\DUrole{std,std-ref}{Terminals}}}} do).

\end{itemize}


\subsection{What kind of calls can be routed through a \emph{Retail Account}?}
\label{retail/retail_accounts:what-kind-of-calls-can-be-routed-through-a-retail-account}
Contrary to Friends, \textbf{Retail Accounts} have some simplifications and limitations.
\begin{itemize}
\item {} 
Retail Accounts only route their assigned DDIs

\item {} 
Retail Accounts only place externals calls to Contract Peerings

\item {} 
Retail Accounts only receive external calls from Contract Peerings

\end{itemize}


\subsection{Retail Accounts Configuration}
\label{retail/retail_accounts:retail-accounts-configuration}
These are the configurable settings of \emph{Retail accounts}:
\begin{description}
\item[{Name\index{Name|textbf}}] \leavevmode\phantomsection\label{retail/retail_accounts:term-name}
Name of the \textbf{retail account}. This name must be unique in the whole brand so
it's recommended to use some kind of secuential identifier. This will also be used
in SIP messages (sent \textbf{From User}).

\item[{Description\index{Description|textbf}}] \leavevmode\phantomsection\label{retail/retail_accounts:term-description}
Optional. Extra information for this \emph{retail account}.

\item[{Password\index{Password|textbf}}] \leavevmode\phantomsection\label{retail/retail_accounts:term-password}
When the \emph{retail account} send requests, IvozProvider will authenticate it using
this password. Like in other SIP agents in IvozProvider \textbf{using password IS A MUST}.

\item[{Direct connection\index{Direct connection|textbf}}] \leavevmode\phantomsection\label{retail/retail_accounts:term-direct-connection}
If you choose `Yes' here, you'll have to fill the protocol, address and
port where this \emph{retail account} can be contacted.

\item[{Fallback Outgoing DDI\index{Fallback Outgoing DDI|textbf}}] \leavevmode\phantomsection\label{retail/retail_accounts:term-fallback-outgoing-ddi}
External calls from this \emph{retail account} will be presented with this DDI, \textbf{unless
the source presented matches a DDI belonging to the account}.

\item[{Country and Area code\index{Country and Area code|textbf}}] \leavevmode\phantomsection\label{retail/retail_accounts:term-country-and-area-code}
Used for number transformation from and to this retail account.

\item[{Allowed codecs\index{Allowed codecs|textbf}}] \leavevmode\phantomsection\label{retail/retail_accounts:term-allowed-codecs}
Like a other SIP entities, \emph{retail accounts} will talk the selected codec.

\item[{From domain\index{From domain|textbf}}] \leavevmode\phantomsection\label{retail/retail_accounts:term-from-domain}
Request from IvozProvider to this account will include this domain in
the From header.

\end{description}


\subsubsection{Voicemail service}
\label{retail/retail_accounts:voicemail-service}
Each retail account has a separate voicemail mailbox to forward received calls using Call Forward Settings below.

\begin{notice}{important}{Important:}
They can use \textbf{Voicemail service code} defined at brand level to listen to messages and to
record a greeting locution different from the standard one.
\end{notice}


\subsubsection{Call Forward Settings}
\label{retail/retail_accounts:call-forward-settings}
Call forward settings can be configured per retail account. These are the fields and available values:
\begin{description}
\item[{Enabled\index{Enabled|textbf}}] \leavevmode\phantomsection\label{retail/retail_accounts:term-enabled}
Determines if the forward must be applied or not. This way, you can have
most used call forward configured and toggle if they apply or not.

\item[{Call Forward type\index{Call Forward type|textbf}}] \leavevmode\phantomsection\label{retail/retail_accounts:term-call-forward-type}
When this forward must be applied:
\begin{itemize}
\item {} 
Inconditional: always.

\item {} 
No answer: when the call is not answered in X seconds.

\item {} 
Busy: When the retail account is busy (486 response code).

\item {} 
Not registered: when the retail account is not registered
against IvozProvider.

\end{itemize}

\item[{Target type\index{Target type|textbf}}] \leavevmode\phantomsection\label{retail/retail_accounts:term-target-type}
What route will use the forwarded call.
\begin{itemize}
\item {} 
Retail account voicemail

\item {} 
Number (external)

\end{itemize}

\end{description}

\begin{notice}{attention}{Attention:}
Calls forwarded to an external number by a call forward setting will keep the original
caller identification, adding the forwarding info in a SIP \emph{Diversion} header.
\end{notice}


\subsection{Asterisk as an account client}
\label{retail/retail_accounts:asterisk-as-an-account-client}
At the other end of a account can be any kind of SIP entity. This section takes
as example an Asterisk PBX system using SIP channel driver that wants to connect
to IvozProvider.


\subsubsection{Account register}
\label{retail/retail_accounts:account-register}
If the system can not be directly access, Asterisk will have to register in the
platform (like a terminal will do).

Configuration will be something like this:

\begin{Verbatim}[commandchars=\\\{\}]
register =\PYGZgt{} account\PYGZhy{}name:account\PYGZhy{}password@ivozprovider\PYGZhy{}brand.sip\PYGZhy{}domain.com
\end{Verbatim}


\subsubsection{Account peer}
\label{retail/retail_accounts:account-peer}
\begin{Verbatim}[commandchars=\\\{\}]
[name\PYGZhy{}peer]
type=peer
host=ivozprovider\PYGZhy{}brand.sip\PYGZhy{}domain.com
context=XXXXXX
disallow=all
allow=alaw
defaultuser=account\PYGZhy{}name
secret=account\PYGZhy{}password
fromdomain=ivozprovider\PYGZhy{}brand.sip\PYGZhy{}domain.com
insecure=port,invite
\end{Verbatim}

\begin{notice}{warning}{Warning:}
\emph{Account clients} MUST NOT challenge IvozProvider. That's
why the \emph{insecure} setting is used here.
\end{notice}


\section{Retail DDI filters}
\label{retail/retail_filters:retail-ddi-filters}\label{retail/retail_filters::doc}\label{retail/retail_filters:retail-filters}
Retail External Filters can be assigned to DDIs to temporary
forward calls to an external number.


\subsection{Filters Configuration}
\label{retail/retail_filters:filters-configuration}
This are the configurable settings of \emph{Retail external filters}:
\begin{description}
\item[{Name\index{Name|textbf}}] \leavevmode\phantomsection\label{retail/retail_filters:term-name}
Name of the filter.

\item[{Number\index{Number|textbf}}] \leavevmode\phantomsection\label{retail/retail_filters:term-number}
External Destination for this filter.

\end{description}

\begin{notice}{attention}{Attention:}
Calls forwarded by a filter will keep the original
caller identification, adding the forwarding info in a SIP
\emph{Diversion} header.
\end{notice}

\begin{notice}{error}{Error:}
Retail DDI filters have precedence over retail account call forward settings (as they apply before).
\end{notice}


\section{Retail DDIs}
\label{retail/retail_ddis::doc}\label{retail/retail_ddis:retail-ddis}\label{retail/retail_ddis:id1}
DDIs are the external entry point from Contract Peerings to Retail Clients that
can be routed through Retail Accounts.


\subsection{DDI filters}
\label{retail/retail_ddis:ddi-filters}
We can assign an \textbf{external call filter} configured in {\hyperref[retail/retail_filters:retail\string-filters]{\sphinxcrossref{\DUrole{std,std-ref}{previous section}}}}. Contrary to Virtual PBX External Call fiters, Retail DDIs
filters only allow static redirection to another external number.


\subsection{Retail DDI routes}
\label{retail/retail_ddis:retail-ddi-routes}
Retail DDIs can only be routed to a {\hyperref[retail/retail_accounts:retail\string-accounts]{\sphinxcrossref{\DUrole{std,std-ref}{Retail Accounts}}}}
or {\hyperref[faxing/index:faxing\string-system]{\sphinxcrossref{\DUrole{std,std-ref}{Virtual Fax}}}}.

\begin{notice}{hint}{Hint:}
Routing a DDI through a Retail account will allow to place external calls
from that account presenting that DDI as origin.
\end{notice}


\subsection{Recordings}
\label{retail/retail_ddis:recordings}
If Retail Client has \emph{Recordings} feature enabled, DDIs can also record incoming and/or
outgoing calls.


\section{Virtual Fax System}
\label{faxing/index:virtual-fax-system}\label{faxing/index::doc}\label{faxing/index:faxing-system}
IvozProvider includes a simple but efficient \emph{faxing} solution that allows:
\begin{itemize}
\item {} 
Sending PDF files via Fax.

\item {} 
Receiving faxes through email or check them through the web portal.

\end{itemize}

\begin{notice}{error}{Error:}
IvozProvider uses
\href{http://www.voip-info.org/wiki/view/T.38}{T.38} for both sending and receiving
faxes. Brand Operator must use \emph{peering contracts} that have support for it.
\end{notice}


\subsection{Creating a virtual fax}
\label{faxing/index:creating-a-virtual-fax}
This is the interface that turns up when we create a new fax in section
\textbf{Company configuration} \textgreater{} \textbf{Virtual Faxes}:

\noindent{\hspace*{\fill}\sphinxincludegraphics{{fax_add}.png}\hspace*{\fill}}

Fields are nearly self-explanatory:
\begin{description}
\item[{Name\index{Name|textbf}}] \leavevmode\phantomsection\label{faxing/index:term-name}
Used by remaining section to reference a fax

\item[{Email\index{Email|textbf}}] \leavevmode\phantomsection\label{faxing/index:term-email}
Email address when we want to receive incoming faxes (if we check `Send
by email')

\item[{Outbound DDI\index{Outbound DDI|textbf}}] \leavevmode\phantomsection\label{faxing/index:term-outbound-ddi}
DDI used as source number for outgoing faxes

\end{description}

To receive faxes in this DDI, we need to point it to our new fax in the section
\textbf{DDIs}:

\noindent{\hspace*{\fill}\sphinxincludegraphics{{fax_ddi}.png}\hspace*{\fill}}

Brand Operator can choose one or more \emph{Outgoing Routes} for sending faxes:

\noindent{\hspace*{\fill}\sphinxincludegraphics{{fax_routes}.png}\hspace*{\fill}}

This route applies to all faxes sent by selected company (or for all companies).

\begin{notice}{note}{Note:}
\emph{load-balancing} y \emph{failover} logics described in {\hyperref[external_outgoing_calls/call_routing:routes\string-metrics]{\sphinxcrossref{\DUrole{std,std-ref}{previous sections}}}}
apply to faxes too.
\end{notice}

\begin{notice}{important}{Important:}
If no fax-specific route is defined, faxes will be routed using
standard call routes.
\end{notice}


\subsection{Sending a fax}
\label{faxing/index:sending-a-fax}
Sending a fax is an easy task. First, we upload de PDF file and set the destination:

\noindent{\hspace*{\fill}\sphinxincludegraphics{{fax_send}.png}\hspace*{\fill}}

The list shows the fax and its status:

\noindent{\hspace*{\fill}\sphinxincludegraphics{{fax_send2}.png}\hspace*{\fill}}


\subsection{Incoming faxes display}
\label{faxing/index:incoming-faxes-display}
Apart from being received by mail, faxes can be watched and downloaded within
the web portal too:

\noindent\sphinxincludegraphics{{fax_list}.png}


\chapter{Billing and Invoices}
\label{billing_and_invoices/index::doc}\label{billing_and_invoices/index:billing-and-invoices}
This block is essential for brand administrators as we will explain how to:
\begin{itemize}
\item {} 
Create pricing plans to assign a price to calls made by final users.

\item {} 
Create invoices that show call details and global costs of each of their companies.

\end{itemize}

We will cover this topics:


\section{Call Registry}
\label{billing_and_invoices/call_registry::doc}\label{billing_and_invoices/call_registry:call-registry}\label{billing_and_invoices/call_registry:id1}
\emph{Call Registry} section shows the calls of the whole telephony platform and it
is present in three different levels:
\begin{itemize}
\item {} 
\textbf{Main management} \textgreater{} \textbf{Call Registry}

\item {} 
\textbf{Brand configuration} \textgreater{} \textbf{Call Registry}

\item {} 
\textbf{Company configuration} \textgreater{} \textbf{Call Registry}

\end{itemize}

Each section filters calls conveniently.
\paragraph{Main management (god)}

Lists all the calls of the platform, showing which brand and company they belong to:

\noindent\sphinxincludegraphics{{call_registry_god}.png}
\paragraph{Brand level}

Lists all the calls of the currently emulated brand, showing which company they
belong to:

\noindent\sphinxincludegraphics{{call_registry_brand}.png}
\paragraph{Company level}

Lists all the calls of the currently emulated company:

\noindent\sphinxincludegraphics{{call_registry_company}.png}

\begin{notice}{note}{Note:}
\href{https://es.wikipedia.org/wiki/CSV}{CSV} export makes possible to
download the list for its later analysis.
\end{notice}

\begin{notice}{hint}{Hint:}
If you choose one specific call, additional information is shown. This
additional information is dependant of the current level (\emph{god}, brand or
company) and shows details about call transfers, call forwards, etc.
\end{notice}


\section{Billable calls}
\label{billing_and_invoices/billable_calls::doc}\label{billing_and_invoices/billable_calls:billable-calls}\label{billing_and_invoices/billable_calls:id1}
\textbf{Billable calls} sections only list \textbf{calls that imply cost}.

\begin{notice}{important}{Important:}
{\hyperref[billing_and_invoices/call_registry:call\string-registry]{\sphinxcrossref{\DUrole{std,std-ref}{Call Registry}}}} sections, on the other hand, show all calls,
even the ones that do not imply cost, such as internal calls, incoming calls,
etc.
\end{notice}

These lists therefore include the price of each call (once it is calculated). Since
companies are notified about its call's price via invoices issued by \textbf{brand operator},
this section is only available at two levels:
\begin{itemize}
\item {} 
Main level (god level)

\item {} 
Brand level

\end{itemize}

Each entry shows this information:
\begin{description}
\item[{Date\index{Date|textbf}}] \leavevmode\phantomsection\label{billing_and_invoices/billable_calls:term-date}
Date and time of the call establishment.

\item[{Brand\index{Brand|textbf}}] \leavevmode\phantomsection\label{billing_and_invoices/billable_calls:term-brand}
Only visible for \emph{god}, shows the brand of each call.

\item[{Company\index{Company|textbf}}] \leavevmode\phantomsection\label{billing_and_invoices/billable_calls:term-company}
Visible for \emph{god} and \emph{brand operator}, show the company of each call.

\item[{Destination\index{Destination|textbf}}] \leavevmode\phantomsection\label{billing_and_invoices/billable_calls:term-destination}
External number dialed.

\item[{Pricing pattern\index{Pricing pattern|textbf}}] \leavevmode\phantomsection\label{billing_and_invoices/billable_calls:term-pricing-pattern}
Shows the {\hyperref[external_outgoing_calls/noplan_nocall:price\string-pattern]{\sphinxcrossref{\DUrole{std,std-ref}{pricing pattern}}}} used to set the price.

\item[{Duration\index{Duration|textbf}}] \leavevmode\phantomsection\label{billing_and_invoices/billable_calls:term-duration}
Shows how long the call lasted.

\item[{Metered (yes/no)\index{Metered (yes/no)|textbf}}] \leavevmode\phantomsection\label{billing_and_invoices/billable_calls:term-metered-yes-no}
Shows if the asynchronous task that sets the price of each call has
parsed each call.

\item[{Price\index{Price|textbf}}] \leavevmode\phantomsection\label{billing_and_invoices/billable_calls:term-price}
The cost of the call.

\item[{Pricing plan\index{Pricing plan|textbf}}] \leavevmode\phantomsection\label{billing_and_invoices/billable_calls:term-pricing-plan}
Shows the {\hyperref[external_outgoing_calls/noplan_nocall:price\string-plan]{\sphinxcrossref{\DUrole{std,std-ref}{Pricing plan}}}} used to set the price.

\item[{Peering contract\index{Peering contract|textbf}}] \leavevmode\phantomsection\label{billing_and_invoices/billable_calls:term-peering-contract}
Shows which {\hyperref[external_incoming_calls/peering_contracts:peering\string-contracts]{\sphinxcrossref{\DUrole{std,std-ref}{Peering contract}}}} was used for
each call.

\item[{Invoice\index{Invoice|textbf}}] \leavevmode\phantomsection\label{billing_and_invoices/billable_calls:term-invoice}
Show if a call is already included in any {\hyperref[billing_and_invoices/invoices:invoices]{\sphinxcrossref{\DUrole{std,std-ref}{invoice}}}}.

\item[{Type (inbound/outbound)\index{Type (inbound/outbound)|textbf}}] \leavevmode\phantomsection\label{billing_and_invoices/billable_calls:term-type-inbound-outbound}
Since some incoming calls can imply cost (see
{\hyperref[external_incoming_calls/configure_ddi:bill\string-inbound]{\sphinxcrossref{\DUrole{std,std-ref}{billing of inbound calls}}}}), show if the call is an
incoming call or an outgoung one.

\end{description}

\begin{notice}{note}{Note:}
As soon as the call is hung up, they appear in this list. In some minutes
time the asynchronous process will set \emph{Metered} to `yes' and will assign a
\textbf{price}.
\end{notice}


\section{Pricing plans}
\label{billing_and_invoices/pricing_plans:pricing-plans}\label{billing_and_invoices/pricing_plans::doc}
In {\hyperref[external_outgoing_calls/noplan_nocall:noplan\string-nocall]{\sphinxcrossref{\DUrole{std,std-ref}{No pricing plan, no call}}}} section the process of creating a \emph{pricing plan} was
described quite thoroughly and main concepts were introduced:
\begin{itemize}
\item {} 
A pricing plan groups some pricing patterns (prefixes) with their cost details:
\begin{itemize}
\item {} 
Cost per minute

\item {} 
Call establishment cost

\item {} 
Bill by seconds, by minutes, etc.

\end{itemize}

\item {} 
A pricing plan is linked to an specific company with a period of time in which
this asociation is valid.

\item {} 
One company may have more than one valid pricing plan for an specific call in
an specific moment.

\item {} 
In such cases, the call price will be calculated using the price detail of the
matching pricing plan with minor metric.

\end{itemize}


\subsection{Manual creation}
\label{billing_and_invoices/pricing_plans:manual-creation}
{\hyperref[external_outgoing_calls/noplan_nocall:price\string-plan]{\sphinxcrossref{\DUrole{std,std-ref}{Manual creation of a pricing plan}}}} implied the previous
creation of at least one {\hyperref[external_outgoing_calls/noplan_nocall:price\string-pattern]{\sphinxcrossref{\DUrole{std,std-ref}{price pattern}}}}.

At this point, the future brand operator may have noticed that creating thousands
of pricing patterns would be a really annoying and time consuming task, as there
are 254 countries, each of them with their mobile networks, landline networks,
special service numbers, etc.

That's why the creation of pricing patterns and pricing plans is done using a
\href{https://es.wikipedia.org/wiki/CSV}{CSV} file.


\subsection{Importing a CSV file}
\label{billing_and_invoices/pricing_plans:importing-a-csv-file}
The first step is creating an empty pricing plan to import the prices in (section
\textbf{Brand configuration} \textgreater{} \textbf{Pricing plans}):

\noindent\sphinxincludegraphics{{pricing_plans_add}.png}

We enter the empty pricing plan we have just created:

\noindent\sphinxincludegraphics{{pricing_plans_add_price}.png}

This is the key button for the massive pricing pattern import process:

\noindent\sphinxincludegraphics{{pricing_plan_csv}.png}

Once chosen the CSV file to import, this window turns up:

\noindent\sphinxincludegraphics{{pricing_plan_csv3}.png}

We can select which column contains which field, in case we want to import a
\href{https://es.wikipedia.org/wiki/CSV}{CSV} file in a non-recommended format. We
can also decide whether to import the first line or discard it as it may have
titles instead of data.

\begin{notice}{hint}{Hint:}
The importing process is done in background, letting the brand operator
continue doing other stuff while it is finished.
\end{notice}


\subsubsection{CSV format}
\label{billing_and_invoices/pricing_plans:csv-format}
Although the above window allowed importing non-recommended format \href{https://es.wikipedia.org/wiki/CSV}{CSV}  files, we encourage you to import a file
in the proposed format, as it will make this process much easier.

The recommended \href{https://es.wikipedia.org/wiki/CSV}{CSV} format is described
in the contextual help section, that includes even a link to download an example
file:

\noindent\sphinxincludegraphics{{pricing_plan_csv2}.png}

The order of the columns should be:
\begin{itemize}
\item {} 
Pricing pattern name

\item {} 
Pricing pattern description

\item {} 
Prefix

\item {} 
Price per minute

\item {} 
Establishment cost

\item {} 
Billing period

\end{itemize}

\begin{notice}{note}{Note:}
It is recommended to double quote alphanumeric entries, though
it is not compulsory for single word entries (or entries without odd symbols).
\textbf{If they contain any comma, they MUST be quoted}.
\end{notice}

\begin{notice}{error}{Error:}
Floating numbers \textbf{MUST use point as decimal separator}.
\end{notice}

\begin{notice}{note}{Note:}
Numeric entries can be quoted with double quotes, but it is
not mandatory.
\end{notice}

\begin{notice}{important}{Important:}
The importing system will just bind the price to an existing
price pattern or, if prefix doesn't match any existing pricing pattern, it will
create one.
\end{notice}

\begin{notice}{warning}{Warning:}
The price of the call will be increased every billing period unit.
\begin{itemize}
\item {} 
If \emph{billing period} is set to 1, every second the price will be increased
\emph{price per minute} divided by 60 (bill by seconds).

\item {} 
If \emph{billing period} is set to 60, every minute the price will be increased
\emph{price per minute} (bill by minutes).

\end{itemize}
\end{notice}

Once the import process is over, we just have to bind the pricing plan to
the companies we want following {\hyperref[external_outgoing_calls/noplan_nocall:pricing\string-plan\string-to\string-company]{\sphinxcrossref{\DUrole{std,std-ref}{the procedure explained in
the previous block}}}}.


\section{Call billing}
\label{billing_and_invoices/bill_a_call::doc}\label{billing_and_invoices/bill_a_call:call-billing}
Billing a call is the \textbf{action of setting a price} to a call that implies cost.


\subsection{Automatic billing}
\label{billing_and_invoices/bill_a_call:automatic-billing}
As exposed previously, billing calls depends upon an automatic process:
\begin{itemize}
\item {} 
When a call is about to be established, IvozProvider verifies that it will be
able to bill it.
\begin{itemize}
\item {} 
If with the current configuration (active and applicable pricing plans for
a given company and for the specific destination) it won't be possible to
bill the call, IvozProvider will prevent its establishment.

\end{itemize}

\item {} 
Once a call that implies cost is hung up, it is listed in {\hyperref[billing_and_invoices/billable_calls:billable\string-calls]{\sphinxcrossref{\DUrole{std,std-ref}{Billable calls}}}}.

\item {} 
Some minutes later, the billing task will evaluate each unbilled call and will
update this fields:
\begin{itemize}
\item {} 
Price

\item {} 
Pricing plan

\item {} 
Pricing pattern

\item {} 
Set Metered to `yes'

\end{itemize}

\end{itemize}


\subsection{Manual re-billing}
\label{billing_and_invoices/bill_a_call:manual-re-billing}
It may happen that the brand operator needs to re-bill an specific call due to
multiple reasons:
\begin{itemize}
\item {} 
Mistake on imported pricing plan/pattern.

\item {} 
Multiple pricing plans with incorrect metric value.

\item {} 
Not asociated pricing plan.

\item {} 
Etc.

\end{itemize}

In these cases, the \emph{brand operator} can re-bill wrongly billed calls.

\begin{notice}{important}{Important:}
Billing a call again means setting the price as it is made right
now. It uses, therefore, the current configuration of pricing plans
not the configuration of the moment the call was made.
\end{notice}

In order to re-bill some calls (or just one), select them in \textbf{Brand configuration}
\textgreater{} \textbf{Billable calls} and press the button \textbf{Bill calls}.

\noindent\sphinxincludegraphics{{retarificate}.png}

\begin{notice}{error}{Error:}
\textbf{It is not possible to re-bill a call that is currently included in
an existing invoice}. In other words, if a selected call has a non-empty
\textbf{Invoice} field, this invoice must be deleted before. The reason behind this
logic is that we don't want and invoice containing calls with a wrong price.
\end{notice}


\section{Invoice generation}
\label{billing_and_invoices/invoices:invoices}\label{billing_and_invoices/invoices::doc}\label{billing_and_invoices/invoices:invoice-generation}
The final goal of this section is to generate invoices with the call that imply
cost of a given company.


\subsection{Invoice templates}
\label{billing_and_invoices/invoices:invoice-templates}
Before generating an example invoice, it is important to understand that invoice
creation process uses templates.

\begin{notice}{hint}{Hint:}
This way, every \textbf{brand operator} can adapt which information
is shown and how this information is shown, add logos, graphs, etc..
\end{notice}

Templates are parsed by \href{https://github.com/XaminProject/handlebars.php}{handlebars} and rendered
using \href{https://wkhtmltopdf.org/}{wkhtmltopdf} library.

The helper in the section \textbf{Brand configuration} \textgreater{} \textbf{Invoice templates} include
a summarized explanation of the creation of templates. In the \href{https://wkhtmltopdf.org/usage/wkhtmltopdf.txt}{official site of wkhtmltopdf} there is plenty additional information.
You can delve into template expressions \href{http://handlebarsjs.com/expressions.html}{here} as well.

By default, this section provides some basic example templates:

\noindent\sphinxincludegraphics{{invoices_templates}.png}


\subsection{Fixed costs}
\label{billing_and_invoices/invoices:fixed-costs}
Fixed cost are a constant concept that can be added to invoices that use invoice
templates that take into account these fixed costs.

Take this image as an example (section \textbf{Fixed costs}):

\noindent\sphinxincludegraphics{{fixed_costs}.png}

We will explain afterwords how these fixed costs can be added to a invoice and
in what amount.


\subsection{Invoice creation}
\label{billing_and_invoices/invoices:invoice-creation}
\textbf{Invoices} section lets \textbf{brand operator} to generate invoices to issue to its
clients.

This is the process to add a create a new invoice:

\noindent{\hspace*{\fill}\sphinxincludegraphics{{invoice_add}.png}\hspace*{\fill}}
\begin{description}
\item[{Number\index{Number|textbf}}] \leavevmode\phantomsection\label{billing_and_invoices/invoices:term-number}
Will be included in the invoice and shows the invoice number

\item[{Company\index{Company|textbf}}] \leavevmode\phantomsection\label{billing_and_invoices/invoices:term-company}
The company whose calls will be invoiced

\item[{Start/End date\index{Start/End date|textbf}}] \leavevmode\phantomsection\label{billing_and_invoices/invoices:term-start-end-date}
The time period of the calls that will be invoiced

\item[{Taxes\index{Taxes|textbf}}] \leavevmode\phantomsection\label{billing_and_invoices/invoices:term-taxes}
Taxes to add to the final cost (e.g. VAT)

\item[{Template\index{Template|textbf}}] \leavevmode\phantomsection\label{billing_and_invoices/invoices:term-template}
Invoice template that will be used

\end{description}

Let's add some fixed costs to this invoice:

\noindent{\hspace*{\fill}\sphinxincludegraphics{{invoice_add2}.png}\hspace*{\fill}}

Select previously defined fixed costs and their amounts:

\noindent{\hspace*{\fill}\sphinxincludegraphics{{invoice_add3}.png}\hspace*{\fill}}

At this point, we can generate the invoice pressing this button:

\noindent{\hspace*{\fill}\sphinxincludegraphics{{invoice_add4}.png}\hspace*{\fill}}

Pressing this button we can see which calls have been included in the invoice:

\noindent{\hspace*{\fill}\sphinxincludegraphics{{invoice_add5}.png}\hspace*{\fill}}

And pressing this one we can download the invoice in PDF format:

\noindent{\hspace*{\fill}\sphinxincludegraphics{{invoice_add6}.png}\hspace*{\fill}}

\begin{notice}{warning}{Warning:}
End date must be a past date. In other words, it is not allowed to
generate invoices for future dates o dates including today.
\end{notice}

\begin{notice}{error}{Error:}
All the calls of the selected period must be billed.
\end{notice}


\chapter{Terminal provisioning}
\label{provisioning/index:terminal-provisioning}\label{provisioning/index::doc}\label{provisioning/index:provisioning}

\section{Overview}
\label{provisioning/index:overview}
IvozProvider supports provisioning of terminals via HTTP/HTTPS that fulfill the
following requirements:
\begin{itemize}
\item {} 
Assuming a just unboxed terminal, just plugged and connected to the network:
\begin{itemize}
\item {} 
Ask IP address via DHCP.

\item {} 
DCHP has enabled the option 66 that points to the platform portal

\item {} 
The first requested provisioning file is a static file (different for each
model) prefixed with the previous step URL.

\item {} 
The served file can redefine the URL for further requests

\end{itemize}

\end{itemize}

Any terminal model that can adapt to this provisioning way can be added into
the section \textbf{Platform Configuration \textgreater{} Terminal manufacturers}.
\paragraph{Example Cisco SPA504G}
\begin{itemize}
\item {} 
Cisco SPA504G is turned on and requests an IP address to DHCP

\item {} 
Receives “\url{http://provision.example.com/provision}” as DHCP option 66

\item {} 
Request HTTP configuration from \url{http://provision.example.com/provision/spa504g.cfg}

\item {} 
All 504G request the same file (spa504.cfg), prefixed with the given URL

\item {} 
This file only contain basic configuration settings for the model and the URL
for the next request (p.e. \url{https://provision.example.com/provision/\$MAC.cfg})

\item {} 
This way, each terminal (MAC should be unique) request a specific file
(and different) after the generic one has been served.

\item {} 
This file will contain the specific configuration for the terminal:
\begin{itemize}
\item {} 
User

\item {} 
Password

\item {} 
SIP Domain

\end{itemize}

\end{itemize}

\begin{notice}{note}{Note:}
IvozProvider provisioning system, right now, only has one goal:
provide credentials and language settings for the terminals.
\end{notice}


\section{Configuration of supported models}
\label{provisioning/index:configuration-of-supported-models}
IvozProvider uses a template system that allows global operator (God) to
define new models and configure what files will be served.

The help section of \textbf{Terminal manufacturers} has examples for some models
that work (in the moment of writting this) with IvozProvider provisioning system.

\begin{notice}{hint}{Hint:}
These models will be available after the initial installation, but
you must edit them and load the default configuration before
you can use the provisioning system (option \textbf{Restore default template}).
\end{notice}

\begin{notice}{error}{Error:}
UACs firmware changes may cause that given examples stop working. We
will try to keep templates updated, but we can't guarantee this point.
\end{notice}

Analyzing the suggested templates you can have a basic idea of the flexibility of
the system to configure any existing terminal model in the market and to adapt
them to eventual changes in given examples.


\section{Getting technical}
\label{provisioning/index:getting-technical}
Imagine an environment with this configuration:
\begin{itemize}
\item {} 
Provisioning URLs:
\begin{itemize}
\item {} 
Generic file: \url{http://PROV\_IP/provision}

\item {} 
Specific file: \url{https://PROV\_IP:PROV\_PORT/provision}

\end{itemize}

\item {} 
TerminalModels.genericUrlPattern: y000000000044.cfg

\end{itemize}

Which requested URLs will be valid?

For generic file, just one: \url{http://PROV\_IP/provision/y000000000044.cfg}

For specific file, requests are right as long as this rules are fulfilled:
\begin{itemize}
\item {} 
All HTTP requests are wrong.

\item {} 
HTTPS requests to 443 are wrong (PROV\_PORT must be used).

\item {} 
Subpaths after provisioning URL are ignored, both in request and in
specificUrlPattern.

\item {} 
On specific file request, extension must match as long as extension is used
in specificUrlPattern.

\item {} 
On specific file request, the filename must match exactly once \{mac\} is replaced.

\item {} 
MAC address is case insensitive and can contain colons or not (`:').

\end{itemize}

Let's analyze the examples below to understand this rules better:
\paragraph{Example 1 - TerminalModels.specificUrlPattern: \{mac\}.cfg}

Working requests:

\begin{Verbatim}[commandchars=\\\{\}]
\PYG{g+go}{https://PROV\PYGZus{}IP:PROV\PYGZus{}PORT/provision/aabbccddeeff.cfg}
\PYG{g+go}{https://PROV\PYGZus{}IP:PROV\PYGZus{}PORT/provision/aa:bb:cc:dd:ee:ff.cfg}
\PYG{g+go}{https://PROV\PYGZus{}IP:PROV\PYGZus{}PORT/provision/aabbccdd:ee:ff.cfg}
\PYG{g+go}{https://PROV\PYGZus{}IP:PROV\PYGZus{}PORT/provision/aabbccddeeff.cfg}
\PYG{g+go}{https://PROV\PYGZus{}IP:PROV\PYGZus{}PORT/provision/AABBCCDDEEFF.cfg}
\PYG{g+go}{https://PROV\PYGZus{}IP:PROV\PYGZus{}PORT/provision/subpath1/aabbccddeeff.cfg}
\PYG{g+go}{https://PROV\PYGZus{}IP:PROV\PYGZus{}PORT/provision/subpath1/subpath2/aabbccddeeff.cfg}
\end{Verbatim}

Wrong requests:

\begin{Verbatim}[commandchars=\\\{\}]
\PYG{g+go}{https://PROV\PYGZus{}IP:PROV\PYGZus{}PORT/provision/aabbccddeeff.boot}
\PYG{g+go}{https://PROV\PYGZus{}IP:PROV\PYGZus{}PORT/provision/subpath1/subpath2/aabbccddeeff.boot}
\end{Verbatim}

This example is identical to `t23/\{mac\}.cfg', as subpaths are ignored.
\paragraph{Example 2 - TerminalModels.specificUrlPattern: \{mac\}}

All previous examples are ok, as extension is ignored if no extension is found
in specificUrlPattern.

This example is identical to `t23/\{mac\}', as subpaths are ignored.
\paragraph{Example 3 - TerminalModels.specificUrlPattern: yea-\{mac\}.cfg}

All previous examples are wrong, as no `yea-` is found (`yea' match is case
sensitive).

Working requests:

\begin{Verbatim}[commandchars=\\\{\}]
\PYG{g+go}{https://PROV\PYGZus{}IP:PROV\PYGZus{}PORT/provision/subpath1/yea\PYGZhy{}aabbccdd:ee:ff.cfg}
\end{Verbatim}

Wrong requests:

\begin{Verbatim}[commandchars=\\\{\}]
\PYG{g+go}{https://PROV\PYGZus{}IP:PROV\PYGZus{}PORT/provision/subpath1/yea\PYGZhy{}aabbccdd:ee:ff.boot}
\PYG{g+go}{https://PROV\PYGZus{}IP:PROV\PYGZus{}PORT/provision/subpath1/YEA\PYGZhy{}aabbccdd:ee:ff.cfg}
\end{Verbatim}

This example is identical to `t23/yea-\{mac\}.cfg', as subpaths are ignored.
\paragraph{Example 4 - TerminalModels.specificUrlPattern: yea-\{mac\}}

As no extension is given:

\begin{Verbatim}[commandchars=\\\{\}]
\PYG{g+go}{https://PROV\PYGZus{}IP:PROV\PYGZus{}PORT/provision/subpath1/yea\PYGZhy{}aabbccdd:ee:ff.cfg}
\PYG{g+go}{https://PROV\PYGZus{}IP:PROV\PYGZus{}PORT/provision/subpath1/yea\PYGZhy{}aabbccdd:ee:ff.boot}
\end{Verbatim}

Wrong requests:

\begin{Verbatim}[commandchars=\\\{\}]
\PYG{g+go}{https://PROV\PYGZus{}IP:PROV\PYGZus{}PORT/provision/subpath1/YEA\PYGZhy{}aabbccdd:ee:ff.cfg}
\end{Verbatim}

This example is identical to `t23/yea-\{mac\}', as subpaths are ignored.


\chapter{User Portal}
\label{userportal/index:userportal}\label{userportal/index:user-portal}\label{userportal/index::doc}
IvozProvider provides a web portal where final users can do the following
actions:
\begin{itemize}
\item {} 
See all calls he or she has been involved.

\item {} 
Configure call forwards:
\begin{itemize}
\item {} 
To voicemail

\item {} 
To an internal extension

\item {} 
To an external number

\end{itemize}

\item {} 
Enable functionalities:
\begin{itemize}
\item {} 
Call waiting

\item {} 
Do Not Disturb

\end{itemize}

\item {} 
See the state of his or her SIP device registration

\end{itemize}


\section{Access URLs}
\label{userportal/index:access-urls}
Prior to accessing to user portal, the URL addresses must be configured (domains
in these URLs must point to any of the public IP addresses of the platform).

2 roles can perform this task:


\subsection{God operator}
\label{userportal/index:god-operator}
In the section \textbf{Platform configuration \textgreater{} Brands} you can configure as many
user URLs as you wish, using the button \textbf{Portal list} of each brand.

\begin{notice}{note}{Note:}
URLs are linked to brands and god operator may choose where to create
one shared user portal URL for all the companies of a brand or creating
one per company.
\end{notice}

\begin{notice}{warning}{Warning:}
URLs MUST be HTTPS.
\end{notice}

This section also allows setting a logo per URL, a theme and a phrase to use as
the title of user portal.

\begin{notice}{hint}{Hint:}
This allows creating corporative user portals.
\end{notice}


\subsection{Brand Operator}
\label{userportal/index:brand-operator}
Brand Operator can also perform this same task in order to configure the user
portal URLs of his companies.

This way, he can choose whether to configure one URL per Company (with custom
domains, logos, theme and title) or sharing a global URL for all of them.

The section to do this is \textbf{Brand configuration \textgreater{} Portal URLs}.


\section{Access credentials}
\label{userportal/index:access-credentials}
Access credentials to user portal is configured in \textbf{Company configuration \textgreater{}
Users} section.

Specifically:
\begin{itemize}
\item {} 
\textbf{Login information} block, the access of each user is enabled or disabled.

\item {} 
You can set the \textbf{Password} too.

\item {} 
To log in the user portal, the user must use his/her email address.

\end{itemize}

\begin{notice}{warning}{Warning:}
The \textbf{email} of each user MUST be \textbf{globally unique}.
\end{notice}


\chapter{Security elements}
\label{security/index:security-elements}\label{security/index::doc}

\section{Firewall}
\label{security/index:firewall}
\textbf{IvozProvider does not currently include a firewall} but...

\begin{notice}{danger}{Danger:}
We \textbf{strongly encourage any production installation to implement
a firewall} to protect the platform from the wild Internet.
\end{notice}

The protection method could be:
\begin{itemize}
\item {} 
Local firewall based on \href{https://www.netfilter.org/}{iptables}

\item {} 
External firewall

\item {} 
Both

\end{itemize}


\subsection{Exposed ports/services}
\label{security/index:exposed-ports-services}
These are the \textbf{ports IvozProvider needs to expose} to work properly:

\textbf{SIP signalling}:
\begin{itemize}
\item {} 
Port 5060 (TCP/UDP)

\item {} 
Port 5061 (TCP)

\item {} 
Port 7060 (TCP/UDP) y 7061 TCP (just in case both ProxyUsers and ProxyTrunks share IP)

\end{itemize}

\textbf{RTP audioflow}:
\begin{itemize}
\item {} 
Port range 13000-19000 UDP

\end{itemize}

\textbf{Web portal and provisioning}:
\begin{itemize}
\item {} 
Ports TCP 443, 1443 y 2443

\end{itemize}

\begin{notice}{hint}{Hint:}
We recommend using \textbf{iptables geoIP module} to drop connections from
countries where we don't have any users.
\end{notice}


\section{Authorized company IP ranges}
\label{security/index:authorized-company-ip-ranges}
During the Company creating process, we skipped the security mechanism that
\textbf{limits the IP addresses or ranges that the company terminals can use
in their terminals}.

This can be activated in the section \textbf{Brand configuration} \textgreater{} \textbf{Company}:

\noindent{\hspace*{\fill}\sphinxincludegraphics{{authorized_ips2}.png}\hspace*{\fill}}

Rest of the users won't be allowed to connect from another network, even if the
credentials are valid.

\begin{notice}{warning}{Warning:}
Once the filter has been activated you \textbf{MUST} add networks or
valid IP addresses, otherwise, all the calls will be rejected.
\end{notice}

\noindent\sphinxincludegraphics{{authorized_ips}.png}

Both IP addresses or ranges can be used, in CIDR format (IP/mask):

\noindent\sphinxincludegraphics{{authorized_ips3}.png}

\begin{notice}{important}{Important:}
This mechanism limits the origin of the users of a company, it
doesn't filter origin from \textbf{Contract Peerings}.
\end{notice}


\subsection{Roadwarrior users}
\label{security/index:roadwarrior-users}\label{security/index:id1}
Some companies have roadwarrior users that travel often and connect from external
networks, forcing Companies to disable the IP filter security mechanism.

To solve this issue, there is a user option called \textbf{Calls for non-granted IPs}
that enables these users to call from non-granted IPs while their companies are
still protected with IP filter mechanism.

When users like theese call from non-granted IPs, their amount of concurrent
outgoing calls are limited to 1, 2 or 3 to avoid being a security breach.

\begin{notice}{warning}{Warning:}
Only calls generated by this kind of user (both internals and
externals) are counted and limited, received calls are not affected
by this setting (they are controlled with \textbf{MaxCalls} setting).
\end{notice}

To sum up, with this feature:
\begin{itemize}
\item {} 
There are users that are allowed to make a fixed amount of calls from
non-granted IPs.

\item {} 
This calls from non-granted IPs are counted and limited.

\end{itemize}
\paragraph{Example 1 - Company without IP check}

It doesn't matter if the user is allowed to make calls from non-granted IPs,
as there are no non-granted IPs.
\paragraph{Example 2 - Company with IP check}
\begin{itemize}
\item {} 
If the user is calling from one of the allowed IPs,
it doesn't matter if the user is allowed to make calls from non-granted IPs:
this calls are not counted nor limited.

\item {} 
If the user is NOT calling from one of the allowed IPs, it is verified the
amount of calls that this user is allowed to make. If the user is allowed to
make calls from non-granted IPs and has not exceeded his limit, the call is
granted and counted.

\end{itemize}

\begin{notice}{note}{Note:}
If \textbf{Calls for non-granted IPs} is set to \emph{None} the user must fulfill
the IP policy of his company.
\end{notice}


\section{Anti-flooding}
\label{security/index:anti-flooding}
IvozProvider comes with an \emph{anti-flooding} mechanism to avoid that a single
sender can deny the platform service by sending lots of requests. Both \emph{proxies}
(users and trunks) use this mechanism, that \textbf{limits the number of requests
from an origin address in a time lapse}.

\begin{notice}{warning}{Warning:}
When an origin reaches this limit, the proxy will stop sending
responses for a period of time. After this time, the requests will be again
handled normally.
\end{notice}

Some origins are automatically excluded from this \emph{anti-flooding} mechanism:
\begin{itemize}
\item {} 
Application Servers from the platform.

\item {} 
Company authorized IP addresses or ranges (see previous section).

\end{itemize}

Global operator of the platform can also add exceptions to this mechanism in
the section \textbf{Global configuration} \textgreater{} \textbf{Antiflood trusted IPs}.

\noindent\sphinxincludegraphics{{trusted_ips}.png}


\section{Concurrent call limit}
\label{security/index:concurrent-call-limit}
This mechanism limits the number of concurrent external calls for a company/retail account. It can also be configured
at brand level.

\begin{notice}{warning}{Warning:}
Brand and company/retail limits are not related anyhow. The one that is reached first will prevent new calls.
\end{notice}

\begin{notice}{tip}{Tip:}
To disable this mechanism, set its value to 0.
\end{notice}


\chapter{Maintenance and troubleshooting}
\label{maintenance/index::doc}\label{maintenance/index:maintenance-and-troubleshooting}
This section described the tools included in IvozProvider to troubleshot any
problems you may have:


\section{Analyzing SIP traffic}
\label{maintenance/sip_captures:analyzing-sip-traffic}\label{maintenance/sip_captures::doc}
Although all production IvozProvider installations mantained by
\href{https://www.irontec.com}{Irontec} include a \href{https://www.sipcapture.org/}{Homer SIP Capture Server}, it is not installed in the standlone version
of IvozProvider. The reason behind this is that we prefer awesome SIPCAPTURE
stack running on an additional machine.

\href{https://github.com/irontec/sngrep}{sngrep Ncurses SIP Messages flow viewer developed by Irontec} is currently
the preferred tool to inspect SIP traffic included in IvozProvider.

\noindent\sphinxincludegraphics{{sngrep_sample}.png}


\subsection{sngrep}
\label{maintenance/sip_captures:sngrep}
See live SIP traffic (all):

\begin{Verbatim}[commandchars=\\\{\}]
\PYG{g+go}{sngrep}
\end{Verbatim}

See live SIP traffic related to calls:

\begin{Verbatim}[commandchars=\\\{\}]
\PYG{g+go}{sngrep \PYGZhy{}c}
\end{Verbatim}

See live SIP traffic and capture RTP too:

\begin{Verbatim}[commandchars=\\\{\}]
\PYG{g+go}{sngrep \PYGZhy{}c \PYGZhy{}r}
\end{Verbatim}

For more reference, visit \href{https://github.com/irontec/sngrep}{sngrep official site}.


\subsection{Other capturing tools}
\label{maintenance/sip_captures:other-capturing-tools}
Although sngrep is our preferred capturing tool, IvozProvider ships other tools
to capture SIP/RTP traffic, such as \href{http://www.tcpdump.org}{tcpdump} and
\href{http://ngrep.sourceforge.net}{ngrep}.


\section{Log viewer}
\label{maintenance/log_viewer::doc}\label{maintenance/log_viewer:log-viewer}
Although all production IvozProvider installations mantained by
\href{https://www.irontec.com}{Irontec} include a \href{https://www.graylog.org/}{Graylog server}, \href{https://www.freedesktop.org/software/systemd/man/journalctl.html}{journalctl} is currently
the unique tool to inspect logs generated by different elements of the solution
in the past.


\subsection{Asterisk CLI}
\label{maintenance/log_viewer:asterisk-cli}
Asterisk CLI gives tons of realtime information too and are formatted beautifully
to detect possible configuration errors:

\noindent\sphinxincludegraphics{{asterisk_cli}.png}

You can access Asterisk CLI typing \emph{ast} in the shell.


\subsection{Kamailio realtime log viewing}
\label{maintenance/log_viewer:kamailio-realtime-log-viewing}
You can see Kamailio logs in realtime too typing \emph{kamtail-proxyusers} and
\emph{kamtail-proxytrunks} in the shell:

\noindent\sphinxincludegraphics{{kamtail}.png}



\renewcommand{\indexname}{Index}
\printindex
\end{document}
